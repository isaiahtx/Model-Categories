% Hirschorn model structure on Top
\documentclass{amsart}
\usepackage[margin=1.3333333in]{geometry}
\newcommand{\ol}{\overline}
\newcommand{\NN}{\mathbb{N}}
\newcommand{\CC}{\mathbb{C}}
\newcommand{\DD}{\mathbb{D}}
\newcommand{\RR}{\mathbb{R}}
\newcommand{\ZZ}{\mathbb{Z}}
\newcommand{\LL}{\mathbb{L}}
\newcommand{\abs}[1]{\left\lvert #1\right\rvert}
\newcommand{\wh}{\widehat}

\setlength{\marginparwidth}{1in}

\usepackage{todonotes}

%%hyperref for clickable links
\usepackage{hyperref}

%%%thmtools to correct autoref
\usepackage{thmtools}

%%%% biblatex

\usepackage{biblatex}
\addbibresource{refs.bib}

%%% lipsum for dummy text
\usepackage{lipsum}

%% AMS packages for math and theorems
\usepackage{amsmath, amsthm, amssymb}

%%% tikz packages for drawing and commutative diagrams
\usepackage{tikz}
\usepackage{tikz-cd}
\usepackage{tikz-3dplot}
\usepackage{quiver}

%%%% euscript for better calligraphic characters in math mode
\usepackage{eucal}[mathcal]

%%%% enumitem package for better list environments
\usepackage{enumitem}

\usepackage{subfiles}

%%% Paragraph spacing

% Paragraph spacing

%\usepackage{parskip}
%
%\setlength{\parskip}{0.5em}
%\setlength{\parindent}{1.5em}

%% comment these lines if you do not want to use bibtex for bibliography management.
%\usepackage{biblatex}
%\addbibresource{refs.bib}


%% theorems in usual style --- italicised text, bold header
\theoremstyle{plain}
\newtheorem{theorem}{Theorem}[section]
\newtheorem{corollary}[theorem]{Corollary}
\newtheorem{proposition}[theorem]{Proposition}
\newtheorem{lemma}[theorem]{Lemma}
\newtheorem*{thm*}{Theorem}

%% theorems in `definition' style --- regular text, bold header
\theoremstyle{definition}
\newtheorem{claim}[theorem]{Claim}
\newtheorem{remark}[theorem]{Remark}
\newtheorem{definition}[theorem]{Definition}
\newtheorem{exercise}[theorem]{Exercise}
\newtheorem{discussion}[theorem]{Discussion}
\newtheorem{notation}[theorem]{Notation}
\newtheorem{convention}[theorem]{Convention}
\newtheorem{conjecture}[theorem]{Conjecture}
\newtheorem{example}[theorem]{Example}

\DeclareMathOperator{\Ch}{Ch}
\newcommand{\Mod}{\mbf{Mod}}
\newcommand{\Top}{\mbf{Top}}
\newcommand{\SSet}{\mbf{SSet}}
\newcommand{\Set}{\mbf{Set}}
\newcommand{\Grp}{\mbf{Grp}}
\newcommand{\Ord}{\mbf{Ord}}
\newcommand{\RMod}{R-\mbf{Mod}}
\newcommand{\op}{\mathrm{op}}

\newcommand{\from}{\colon}
\newcommand{\sseq}{\subseteq}
\newcommand{\wt}{\widetilde}
\newcommand{\spseq}{\supseteq}
\newcommand{\brn}{\mathbb R^n}
\newcommand{\bRn}{\mathbb R^n}
\newcommand{\0}{\mathbf{0}}
\newcommand{\bR}{\mathbb{R}}
\newcommand{\cA}{\mathcal A}
\newcommand{\cB}{\mathcal B}
\newcommand{\cC}{\mathcal C}
\newcommand{\cD}{\mathcal D}
\newcommand{\id}{\mathrm{id}}
\newcommand{\cE}{\mathcal E}
\newcommand{\cF}{\mathcal F}
\newcommand{\cG}{\mathcal G}
\newcommand{\cH}{\mathcal H}
\newcommand{\cI}{\mathcal I}
\newcommand{\p}{{_\perp}}
\newcommand{\cJ}{\mathcal J}
\newcommand{\cK}{\mathcal K}
\newcommand{\cL}{\mathcal L}
\newcommand{\cM}{\mathcal M}
\newcommand{\cN}{\mathcal N}
\newcommand{\cO}{\mathcal O}
\newcommand{\cP}{\mathcal P}
\newcommand{\cQ}{\mathcal Q}
\newcommand{\into}{\hookrightarrow}
\newcommand{\onto}{\twoheadrightarrow}
\newcommand{\cR}{\mathcal R}
\newcommand{\cS}{\mathcal S}
\newcommand{\cT}{\mathcal T}
\newcommand{\cU}{\mathcal U}
\newcommand{\cV}{\mathcal V}
\newcommand{\cW}{\mathcal W}
\newcommand{\cX}{\mathcal X}
\newcommand{\cY}{\mathcal Y}
\newcommand{\cZ}{\mathcal Z}
\newcommand{\mbf}[1]{\mathbf{#1}}
\renewcommand{\ol}{\overline}
\newcommand{\ul}{\underline}
\newcommand{\bZ}{\mathbb{Z}}
\newcommand{\dx}{\,\mathrm dx}
\newcommand{\dt}{\,\mathrm dt}
\newcommand{\bC}{\mathbb{C}}
\newcommand{\bN}{\mathbb{N}}
\newcommand{\bQ}{\mathbb{Q}}
\newcommand{\vare}{\varepsilon}
\renewcommand{\(}{\left(}
\renewcommand{\)}{\right)}
\newcommand\defeq{\mathrel{\overset{\makebox[0pt]{\mbox{\normalfont\tiny def}}}{=}}}
\newcommand{\phantomreplace}[2]{\makebox[0pt][l]{#1}\hphantom{#2}}
\newcommand{\phantommathreplace}[2]{\makebox[0pt][l]{$\displaystyle #1$}\hphantom{#2}}
\makeatletter
\newcommand{\skipitems}[1]{%
  \addtocounter{\@enumctr}{#1}%
}
\makeatother

%%% hyperlinks/citations in document with prettier links
\hypersetup{%
	colorlinks,%
	linkcolor={red!60!black},%
	citecolor={red!60!black},%
	urlcolor={red!60!black}%
}

%%% Shortened operators
\def\on{\operatorname}
\def\scr{\EuScript}
\def\bb{\mathbb}
\def\sf{\mathsf}
\def\cal{\mathcal}

%% blackboard bolds
%\def\RR{\bb{R}}
%\def\CC{\bb{C}}
%\def\ZZ{\bb{Z}}
%\def\NN{\bb{N}}
%\def\QQ{\bb{Q}}

\newcommand{\Cof}{\mathcal C\mathrm{of}}
\newcommand{\Fib}{\mathcal F\mathrm{ib}}
\newcommand{\W}{\mathcal W}
\newcommand{\inj}{\text-\mathrm{inj}}
\newcommand{\proj}{\text-\mathrm{proj}}
\newcommand{\fib}{\text-\mathrm{fib}}
\newcommand{\cell}{\text-\mathrm{cell}}
\newcommand{\cof}{\text-\mathrm{cof}}
\DeclareMathOperator*{\colim}{colim}
\DeclareMathOperator{\Mor}{Mor}

%%% so walker can write comments in a different color.
\usepackage{xcolor}

\newcommand{\shorten}[1]{{\color{purple} #1}}
\newcommand{\remove}[1]{{\color{red} #1}}

\newcommand{\ww}[1]{{\color{blue} Walker: #1}}
\newcommand{\ii}[1]{{\color{teal} Isaiah Question: #1}}

%%% parentheses will denote suggested text
\newcommand{\sugs}[1]{{\color{blue} Suggested rewrite: 

#1}}

%%% to strike through text without deleting it
\usepackage[normalem]{ulem}

\title{Model Structures}

\author{Isaiah Dailey}
\date{\today}

\begin{document}
\maketitle


%\setcounter{tocdepth}{0}
\tableofcontents

This document follows Mark Hovey's \textit{Model Categories}, and its intention is to reproduce the proofs of several standard model categories in explicit detail.

\section{Preliminaries}

\subfile{sections/section1}

\section{The Model Structure on Topological Spaces}

\subfile{sections/section2}

\section{Simplicial Sets}

\subfile{sections/section3}

\section{The Model Structure on Simplicial Sets}

\subfile{sections/section4}

\textbf{Questions/Comments:}\begin{enumerate}
  \item I don't follow the proof of Lemma 2.1 in Goerss \& Jardine.
  
  I think I've figured it out, it is a consequence of the following lemma:
  \begin{lemma}
    Let $\cC$ be a category and $K\in\Set_\cC$ a presheaf on $\cC$. Then $K$ is the colimit of the diagram
    \[F:\cC\downarrow  K\to\Set_\cC\]
    (here $\cC\downarrow K$ is shorthand for the comma category $\cY\downarrow\iota$, where $\cY:\cC\to\Set_\cC$ is the Yoneda embedding and $\iota:\ast\to\Set_\cC$ picks out $K$)
    which assigns to each natural transformation $\omega:\cC(-,c)\Rightarrow K$ in $\cC\downarrow K$ the natural transformation $\cC(-,c)$, and sends each morphism in $\cC\downarrow K$
    % https://q.uiver.app/#q=WzAsMyxbMCwwLCJcXGNDKC0sYykiXSxbMiwwLCJcXGNDKC0sYycpIl0sWzEsMSwiSyJdLFswLDEsImZfXFxhc3QiLDAseyJsZXZlbCI6Mn1dLFsxLDIsIlxcdGF1IiwwLHsibGV2ZWwiOjJ9XSxbMCwyLCJcXHNpZ21hIiwyLHsibGV2ZWwiOjJ9XV0=
    \[\begin{tikzcd}
      {\cC(-,c)} && {\cC(-,c')} \\
      & K
      \arrow["{f_\ast}", Rightarrow, from=1-1, to=1-3]
      \arrow["\tau", Rightarrow, from=1-3, to=2-2]
      \arrow["\sigma"', Rightarrow, from=1-1, to=2-2]
    \end{tikzcd}\]
    to the arrow $f_\ast$ in $\Set_\cC$. Then $K$ is the colimit of $F$.
  \end{lemma}
  \begin{proof}
    First, define a cone $\eta$ under $F$ with nadir $K$ by defining $\eta_\sigma:\cC(-,c)\Rightarrow K$ to be simply $\sigma$. This clearly satisfies the naturality condition by how $\cC\downarrow K$ is defined. Now, suppose we are given another cone $\vare:F\Rightarrow\underline{X}$ under $F$. Then note given an object $c$ in $\cC$, by the Yoneda Lemma
    \[\Set(K(c),X(c))\cong\Set(\Set_\cC(\cC(-,c),K),\Set_\cC(\cC(-,c),X)).\]
    Define $\lambda:K\Rightarrow X$ by defining $\lambda_c$ to be the morphism $K(c)\to X(c)$ corresponding to the map $\Set_\cC(\cC(-,c),K)\to\Set_\cC(\cC(-,c),X)$ sending $\alpha\mapsto\vare_\alpha$. We claim this transformatio is natural. First, we claim that the map $\Set_\cC(\cC(-,c),K)\to\Set_\cC(\cC(-,c),X)$ is natural in $c$. Indeed, given $f:c\to c'$ in $\cC$, we want to show the following diagram commutes:
    % https://q.uiver.app/#q=WzAsNCxbMCwwLCJcXFNldF9cXGNDKFxcY0MoLSxjJyksSykiXSxbMiwwLCJcXFNldF9cXGNDKFxcY0MoLSxjJyksWCkiXSxbMiwyLCJcXFNldF9cXGNDKFxcY0MoLSxjKSxYKSJdLFswLDIsIlxcU2V0X1xcY0MoXFxjQygtLGMpLEspIl0sWzAsMSwiXFxhbHBoYVxcbWFwc3RvXFx2YXJlX1xcYWxwaGEiXSxbMSwyLCJcXFNldF9cXGNDKGZfXFxhc3QsWCkiXSxbMCwzLCJcXFNldF9cXGNDKGZfXFxhc3QsSykiLDJdLFszLDIsIlxcYmV0YVxcbWFwc3RvXFx2YXJlX1xcYmV0YSIsMl1d
    \[\begin{tikzcd}
      {\Set_\cC(\cC(-,c'),K)} && {\Set_\cC(\cC(-,c'),X)} \\
      \\
      {\Set_\cC(\cC(-,c),K)} && {\Set_\cC(\cC(-,c),X)}
      \arrow["{\alpha\mapsto\vare_\alpha}", from=1-1, to=1-3]
      \arrow["{\Set_\cC(f_\ast,X)}", from=1-3, to=3-3]
      \arrow["{\Set_\cC(f_\ast,K)}"', from=1-1, to=3-1]
      \arrow["{\beta\mapsto\vare_\beta}"', from=3-1, to=3-3]
    \end{tikzcd}\]
    Chasing an element $\alpha$ around the top of the diagram yields $\vare_\alpha\circ f_\ast$, while chasing $\alpha$ around the bottom of the diagram yields $\vare_{\alpha\circ f_\ast}$. Then $\vare_\alpha\circ f_\ast=\vare_{\alpha\circ f_\ast}$ by the naturality condition for $\vare$, which tells us the following diagram commutes:
    % https://q.uiver.app/#q=WzAsMyxbMCwwLCJcXGNDKC0sYykiXSxbMiwwLCJcXGNDKC0sYycpIl0sWzEsMSwiWCJdLFswLDEsImZfXFxhc3QiLDAseyJsZXZlbCI6Mn1dLFsxLDIsIlxcdmFyZV9cXGFscGhhIiwwLHsibGV2ZWwiOjJ9XSxbMCwyLCJcXHZhcmVfe1xcYWxwaGFcXGNpcmMgZl9cXGFzdH0iLDIseyJsZXZlbCI6Mn1dXQ==
    \[\begin{tikzcd}
      {\cC(-,c)} && {\cC(-,c')} \\
      & X
      \arrow["{f_\ast}", Rightarrow, from=1-1, to=1-3]
      \arrow["{\vare_\alpha}", Rightarrow, from=1-3, to=2-2]
      \arrow["{\vare_{\alpha\circ f_\ast}}"', Rightarrow, from=1-1, to=2-2]
    \end{tikzcd}\]
  \end{proof}
\end{enumerate}

\end{document}