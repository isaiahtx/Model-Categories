% Hirschorn model structure on Top

\documentclass{amsart}
% \usepackage[margin=1.3333333in]{geometry}
\usepackage[margin = 1in]{geometry}
\newcommand{\ol}{\overline}
\newcommand{\NN}{\mathbb{N}}
\newcommand{\CC}{\mathbb{C}}
\newcommand{\DD}{\mathbb{D}}
\newcommand{\RR}{\mathbb{R}}
\newcommand{\ZZ}{\mathbb{Z}}
\newcommand{\LL}{\mathbb{L}}
\newcommand{\abs}[1]{\left\lvert #1\right\rvert}
\newcommand{\wh}{\widehat}

%%hyperref for clickable links
\usepackage{hyperref}

%%%thmtools to correct autoref
\usepackage{thmtools}

%%%% biblatex

\usepackage{biblatex}
\addbibresource{refs.bib}

%%% lipsum for dummy text
\usepackage{lipsum}

%% AMS packages for math and theorems
\usepackage{amsmath, amsthm, amssymb}

%%% tikz packages for drawing and commutative diagrams
\usepackage{tikz}
\usepackage{tikz-cd}
\usepackage{tikz-3dplot}
\usepackage{quiver}

%%%% euscript for better calligraphic characters in math mode
\usepackage{eucal}[mathcal]

%%%% enumitem package for better list environments
\usepackage{enumitem}

%%% Paragraph spacing

% Paragraph spacing

%\usepackage{parskip}
%
%\setlength{\parskip}{0.5em}
%\setlength{\parindent}{1.5em}

%% comment these lines if you do not want to use bibtex for bibliography management.
%\usepackage{biblatex}
%\addbibresource{refs.bib}


%% theorems in usual style --- italicised text, bold header
\theoremstyle{plain}
\newtheorem{theorem}{Theorem}[section]
\newtheorem{corollary}[theorem]{Corollary}
\newtheorem{proposition}[theorem]{Proposition}
\newtheorem{lemma}[theorem]{Lemma}
\newtheorem*{thm*}{Theorem}

%% theorems in `definition' style --- regular text, bold header
\theoremstyle{definition}
\newtheorem{claim}[theorem]{Claim}
\newtheorem{remark}[theorem]{Remark}
\newtheorem{definition}[theorem]{Definition}
\newtheorem{exercise}[theorem]{Exercise}
\newtheorem{discussion}[theorem]{Discussion}
\newtheorem{notation}[theorem]{Notation}
\newtheorem{convention}[theorem]{Convention}
\newtheorem{conjecture}[theorem]{Conjecture}
\newtheorem{example}[theorem]{Example}

\DeclareMathOperator{\Ch}{Ch}
\newcommand{\Mod}{\mbf{Mod}}
\newcommand{\Top}{\mbf{Top}}
\newcommand{\Set}{\mbf{Set}}
\newcommand{\Grp}{\mbf{Grp}}
\newcommand{\Ord}{\mbf{Ord}}
\newcommand{\RMod}{R-\mbf{Mod}}

\newcommand{\from}{\colon}
\newcommand{\sseq}{\subseteq}
\newcommand{\wt}{\widetilde}
\newcommand{\spseq}{\supseteq}
\newcommand{\brn}{\mathbb R^n}
\newcommand{\bRn}{\mathbb R^n}
\newcommand{\0}{\mathbf{0}}
\newcommand{\bR}{\mathbb{R}}
\newcommand{\cA}{\mathcal A}
\newcommand{\cB}{\mathcal B}
\newcommand{\cC}{\mathcal C}
\newcommand{\cD}{\mathcal D}
\newcommand{\id}{\mathrm{id}}
\newcommand{\cE}{\mathcal E}
\newcommand{\cF}{\mathcal F}
\newcommand{\cG}{\mathcal G}
\newcommand{\cH}{\mathcal H}
\newcommand{\cI}{\mathcal I}
\newcommand{\p}{{_\perp}}
\newcommand{\cJ}{\mathcal J}
\newcommand{\cK}{\mathcal K}
\newcommand{\cL}{\mathcal L}
\newcommand{\cM}{\mathcal M}
\newcommand{\cN}{\mathcal N}
\newcommand{\cO}{\mathcal O}
\newcommand{\cP}{\mathcal P}
\newcommand{\cQ}{\mathcal Q}
\newcommand{\into}{\hookrightarrow}
\newcommand{\onto}{\twoheadrightarrow}
\newcommand{\cR}{\mathcal R}
\newcommand{\cS}{\mathcal S}
\newcommand{\cT}{\mathcal T}
\newcommand{\cU}{\mathcal U}
\newcommand{\cV}{\mathcal V}
\newcommand{\cW}{\mathcal W}
\newcommand{\cX}{\mathcal X}
\newcommand{\cY}{\mathcal Y}
\newcommand{\cZ}{\mathcal Z}
\newcommand{\mbf}[1]{\mathbf{#1}}
\renewcommand{\ol}{\overline}
\newcommand{\ul}{\underline}
\newcommand{\bZ}{\mathbb{Z}}
\newcommand{\dx}{\,\mathrm dx}
\newcommand{\dt}{\,\mathrm dt}
\newcommand{\bC}{\mathbb{C}}
\newcommand{\bN}{\mathbb{N}}
\newcommand{\bQ}{\mathbb{Q}}
\newcommand{\vare}{\varepsilon}
\renewcommand{\(}{\left(}
\renewcommand{\)}{\right)}
\newcommand\defeq{\mathrel{\overset{\makebox[0pt]{\mbox{\normalfont\tiny def}}}{=}}}
\newcommand{\phantomreplace}[2]{\makebox[0pt][l]{#1}\hphantom{#2}}
\newcommand{\phantommathreplace}[2]{\makebox[0pt][l]{$\displaystyle #1$}\hphantom{#2}}
\makeatletter
\newcommand{\skipitems}[1]{%
  \addtocounter{\@enumctr}{#1}%
}
\makeatother

%%% hyperlinks/citations in document with prettier links
\hypersetup{%
	colorlinks,%
	linkcolor={red!60!black},%
	citecolor={red!60!black},%
	urlcolor={red!60!black}%
}

%%% Shortened operators
\def\on{\operatorname}
\def\scr{\EuScript}
\def\bb{\mathbb}
\def\sf{\mathsf}
\def\cal{\mathcal}

%% blackboard bolds
%\def\RR{\bb{R}}
%\def\CC{\bb{C}}
%\def\ZZ{\bb{Z}}
%\def\NN{\bb{N}}
%\def\QQ{\bb{Q}}

\newcommand{\Cof}{\mathcal C\mathrm{of}}
\newcommand{\Fib}{\mathcal F\mathrm{ib}}
\newcommand{\W}{\mathcal W}
\newcommand{\inj}{\text-\mathrm{inj}}
\newcommand{\proj}{\text-\mathrm{proj}}
\newcommand{\fib}{\text-\mathrm{fib}}
\newcommand{\cell}{\text-\mathrm{cell}}
\newcommand{\cof}{\text-\mathrm{cof}}
\DeclareMathOperator*{\colim}{colim}
\DeclareMathOperator{\Mor}{Mor}

%%% so walker can write comments in a different color.
\usepackage{xcolor}

\newcommand{\shorten}[1]{{\color{purple} #1}}
\newcommand{\remove}[1]{{\color{red} #1}}

\newcommand{\ww}[1]{{\color{blue} Walker: #1}}
\newcommand{\ii}[1]{{\color{teal} Isaiah Question: #1}}

%%% parentheses will denote suggested text
\newcommand{\sugs}[1]{{\color{blue} Suggested rewrite: 

#1}}

%%% to strike through text without deleting it
\usepackage[normalem]{ulem}

\title{Model Structures}

\author{Isaiah Dailey}
\date{\today}

\begin{document}
\maketitle


%\setcounter{tocdepth}{0}
\tableofcontents

This document follows Mark Hovey's \textit{Model Categories}, and its intention is to reproduce the proofs of several standard model categories in explicit detail.

\section{Preliminaries}

We work with von Neumann ordinals, i.e.,\ an ordinal is a transitive set of ordinals (this definition is not circular, the empty set is an ordinal which we call ``$0$''). In the following discussion, let $\alpha$ and $\beta$ be ordinals. We write $\alpha+1$ to denote the successor ordinal $\alpha\cup\{\alpha\}$. We write $\alpha<\beta$ to mean $\alpha\in\beta$, and $\alpha\leq\beta$ denotes any of the equivalent conditions: (1) $\alpha<\beta$ or $\alpha=\beta$, (2) $\alpha\in\beta+1$, (3) $\alpha\sseq\beta$. Given a collection of ordinals $B$, we write $\sup B$ or $\sup_{\beta\in B}\beta$ to denote the ordinal $\bigcup_{\beta\in B}\beta$. We define the sum of ordinals $\alpha$ and $\beta$ recursively: $\alpha+0:=\alpha$, $\alpha+(\beta+1):=(\alpha+\beta)+1$, and $\alpha+\beta:=\sup_{\delta<\beta}(\alpha+\delta)$ when $\beta$ is a limit ordinal. Note that addition of ordinals is not commutative, but it is associative, and continuous in its right argument: given an ordinal $\alpha$ and a collection of ordinals $B$, $\alpha+\sup B=\sup_{\beta\in B}(\alpha+\beta)$. We say an ordinal $\lambda$ is a \textit{limit ordinal} if either of the following equivalent conditions hold: (1) $\lambda=\sup_{\beta<\lambda}\beta$ or (2) $\lambda\neq\beta+1$ for all ordinals $\beta$. Note that $0$ is a limit ordinal under our definition. We may regard an ordinal $\alpha$ as a poset category, in which case the colimit in $\alpha$ is given by the supremum. We let $\Ord$ denote the poset category of all (small) ordinals, so there exists a unique arrow $\alpha\to\beta$ if $\alpha\leq\beta$. Given a set $X$, we write $|X|$ to denote its \textit{cardinality}, i.e.,\ $|X|$ is the least ordinal $\alpha$ such that there exists a bijection between $\alpha$ and $X$. A cardinal number is an an ordinal which is the cardinality of some set $X$.

\begin{definition}[Hovey Definition 2.1.1]
  Suppose $\cC$ is a cocomplete category, and $\lambda$ is an ordinal. A \textit{$\lambda$-sequence} in $\cC$ is a colimit-preserving functor $X:\lambda\to\cC$, commonly written as
  \[X_0\to X_1\to\cdots\to X_\beta\to\cdots.\]
  Since $X$ preserves colimits, for all limit ordinals $\gamma<\lambda$, the arrows $X_\alpha\to X_\gamma$ for $\alpha<\gamma$ form a colimit cone under $\{X_\alpha\}_{\alpha<\gamma}$. We refer to the map $X_0\to \colim_{\beta<\lambda}X_\beta$ as the \textit{composition} of the $\lambda$-sequence. Given a collection $\cD$ of morphisms in $\cC$ such that every map $X_\beta\to X_{\beta+1}$ for $\beta+1<\lambda$ is in $\cD$, we refer to the composition $X_0\to\colim_{\beta<\lambda}X_\beta$ as a \textit{transfinite composition} of arrows in $\cD$.\footnote{To be more precise, there may be different (isomorphic) choices of colimit $\colim_{\beta<\gamma}X_\beta$, which give rise to different choices of composition $X_0\to\colim_{\beta<\gamma}X_\beta$. Thus, the composition of a $\lambda$-sequence is only unique up to composition by a unique isomorphism.}
\end{definition}

Of particular importance to us will be collections of arrows which are \textit{closed under transfinite composition}, i.e., collections $\cD$ for which given any ordinal $\lambda$ and $\lambda$-sequence $X$ of arrows in $\cD$, for any choice of colimit $\colim X$, the canonical map $X_0\to\colim X$ is also in $\cD$. We prove the following useful result about when a class of morphisms is closed under transfinite composition:

\begin{lemma}\label{condition_for_family_of_arrows_to_be_closed_under_transfinite_composition}
  Let $\cC$ be a category, and $\cD$ a collection of arrows in $\cC$ satisfying the following properties: $\cD$ is closed under composition with isomorphisms, and given an ordinal $\lambda$ and a $\lambda$-sequence $X:\lambda\to\cC$ of arrows in $\cD$ (so $X_\beta\to X_{\beta+1}$ belongs to $\cD$ for all $\beta+1<\lambda$), if we then get then get for free that $X_\alpha\to X_\beta$ belongs to $\cD$ for all $\alpha\leq\beta<\lambda$, then $\cD$ is closed under transfinite composition.
\end{lemma}
\begin{proof}
  Let $\lambda$ be an ordinal, and $X:\lambda\to\cC$ a $\lambda$-sequence of arrows in $\cD$. First, suppose $\lambda=\mu+1$ is a successor ordinal. Since we know that any transfinite composition of $X$ may be obtained from another by composing with an isomorphism and $\cD$ is closed under composition with isomorphisms, it suffices to show there exists \textit{some} transfinite composition of $X$ belonging to $\cD$. We know $\sup_{\beta<\lambda}\beta=\sup_{\beta<\mu+1}\beta=\mu$, and $X$ is colimit preserving, so that $X_\mu$ is a colimit of the diagram $X$ via the arrows $X_{\alpha}\to X_\mu$ for $\alpha<\lambda=\mu+1$. But we know in particular that $X_0\to X_\mu$ belongs to $\cD$, so we are done.
 
  Conversely, suppose $\lambda$ is a limit ordinal. Let $j:X\Rightarrow\ul{X_\lambda}$ be a colimit cone for $X$. We may use $j$ to extend $X$ to a $(\lambda+1)$-sequence in the obvious way (so for $\alpha<\lambda$, the structure map $X_\alpha\to X_\lambda$ is given by $j$ and the arrow $X_\lambda\to X_\lambda$ is the identity, as is necessary). Further note that $X$ is still a sequence of arrows in $\cD$, as given $\beta+1<\lambda+1$, so $\beta+1\leq\lambda$, it is not possible that $\beta+1=\lambda$ as $\lambda$ is a limit ordinal, in which case we know the map $X_\beta\to X_{\beta+1}$ belongs to $\cD$ as $\beta+1<\lambda$. Hence, unravelling definitions and applying the asserted property of $\cD$, we get for free that $j_0:X_0\to X_\lambda$ belongs to $\cD$.
\end{proof}

\begin{lemma}\label{stronger_characterization_of_closure_under_transfinite_composition}
  Given a cocomplete category $\cC$ and a collection $\cD$ of arrows in $\cC$, if $\cD$ is closed under transfinite composition, then given any limit ordinal $\lambda$ and $\lambda$-sequence $X:\lambda\to\cC$, for all $\alpha<\lambda$ the canonical map $X_\alpha\to\colim X$ belongs to $\cD$.
\end{lemma}
\begin{proof}[Proof Sketch]
  Let $\alpha<\lambda$, and fix a colimit cone $j:X\Rightarrow\ul{\colim X}$. Define $S:=\{\beta:\alpha\leq\beta\leq\lambda\}\sseq\lambda+1$. Define a map $\phi:S\to\Ord$ via transfinite recursion. Let $\phi(\alpha)=0$. Supposing $\phi(\beta)$ has been defined, let $\phi(\beta+1)=\phi(\beta)+1$. Finally, supposing $\alpha<\gamma\leq\lambda$ is a limit ordinal and $\phi(\beta)$ has been defined for $\alpha\leq\beta<\gamma$, define $\phi(\gamma)=\sup_{\alpha\leq\beta<\gamma}\phi(\beta)$. It is straightforward to verify that $\phi$ is order preserving, sends limit ordinals to limit ordinals, and satisfies $\alpha+\phi(\beta)=\beta$ for all $\alpha\leq\beta\leq\lambda$.
  
  Now, construct a $\phi(\lambda)$-sequence $Y:\phi(\lambda)\to\cC$ by $Y_\beta:=X_{\alpha+\beta}$, and given $\beta\leq\beta'<\phi(\lambda)$, define the map $Y_\beta\to Y_{\beta'}$ to be the arrow $X_{\alpha+\beta}\to X_{\alpha+\beta'}$ for $X$. Checking that $Y$ is functorial and colimit-preserving follows directly from the fact that $X$ is functorial and colimit-preserving. Then it can be seen that the $j_{\alpha+\beta}$'s for $0\leq\beta<\phi(\lambda)$ restrict to a colimit cone under $Y$. Since $Y$ is a $\phi(\lambda)$-sequence in $\cD$ and $\cD$ is closed under transfinite compositions, it follows that $j_\alpha\in\cD$, as desired.
\end{proof}

\begin{definition}[Hovey Definition 2.1.2]
  Let $\gamma$ be a cardinal. An ordinal $\alpha$ is \textit{$\gamma$-filtered} if it is a limit ordinal and, if $A\sseq\alpha$ and $|A|\leq\gamma$, then $\sup A<\alpha$.
\end{definition}

Given a cardinal $\gamma$, a $\gamma$-filtered category $\cC$ is one such that any diagram $\cD\to\cC$ has a cocone when $\cD$ has $<\gamma$ arrows. A catgory is just ``filtered'' if it is $\omega$-filtered, i.e., if every finite diagram in $\cC$ admits a cocone. Note that an ordinal $\alpha$ is $\gamma$-filtered precisely when it is $\gamma$-filtered as a category, and in particular every ordinal is $\omega$-filtered.

\begin{definition}[Hovey Definition 2.1.3]\label{2.1.3}
  Suppose $\cC$ is a comcomplete category, $\cD\sseq\Mor\cC$ is some collection of morphisms of $\cC$, $A$ is an object of $\cC$, and $\kappa$ is a cardinal. We say that $A$ is \textit{$\kappa$-small relative to $\cD$} if, for all $\kappa$-filtered ordinals $\lambda$ and all $\lambda$-sequences
  \[X_0\to X_1\to\cdots\to X_\beta\to\cdots\]
  such that each map $X_\beta\to X_{\beta+1}$ is in $\cD$ for $\beta+1<\lambda$, the canonical map of sets
  \[\colim_{\beta<\lambda}\cC(A,X_\beta)\to\cC(A,\colim_{\beta<\lambda}X_\beta)\]
  is an isomorphism. We say that $A$ is \textit{small relative to $\cD$} if it is $\kappa$-small relative to $\cD$ for some $\kappa$. We say that $A$ is \textit{small} if it is small relative to $\cC$ itself.
\end{definition}

\begin{definition}[Hovey Definition 2.1.4]
  Suppose $\cC$ is a cocomplete category, $\cD$ is a collection of morphisms of $\cC$, and $A$ is an object of $\cC$. We say that $A$ is \textit{finite relative to $\cD$} if $A$ is $\kappa$-small relative to $\cD$ for some finite cardinal $\kappa$. We say $A$ is \textit{finite} if it is finite relative to $\cC$ itself. In particular, since \textit{every} limit ordinal is $\kappa$-filtered for any finite cardinal $\kappa$, for an object $A$ to be finite relative to $\cD$, maps from $A$ must commute with colimits of \textit{arbitrary} $\lambda$-sequences for every limit ordinal $\lambda$.
\end{definition}

\begin{remark}\label{explicit_description_of_colimit_in_set}
Recall that given a small category $\cD$ and a functor $F:\cD\to\Set$, we may explicitly construct the colimit of $F$ as the set
\[\colim F:=\(\coprod_{d\in \cD}F(d)\)/\sim,\]
where the equivalence relation $\sim$ is \textbf{generated} by
\[((x\in F(d))\sim(x'\in F(d')))\quad\text{ if }\quad(\exists(f:d\to d')\text{ with }Ff(x)=x').\]
In particular, if $\cD$ is a filtered category then the resulting relation can be described as follows:
\begin{equation*}
  ((x\in F(d))\sim(x'\in F(d')))\quad\text{ iff }\quad(\exists\ d'',\,(f:d\to d''),\,(g:d'\to d'')\text{ with }Ff(x)=Fg(x')).
\end{equation*}
Then the colimit cone $\eta:F\Rightarrow\ul{\colim F}$ is defined by $\eta_d(x)=[x]$ for $d\in\cD$ and $x\in F(d)$, where $[x]$ denotes the equivalence class of $x$ in $\colim F$. Given a cone $\vare:F\Rightarrow\underline Y$ under $F$, the unique map $\colim F\to Y$ maps an equivalence class $[x]$ represented by an element $x\in F(d)$ to the element $\vare_d(x)$.
\end{remark}

Now we unravel what the ``canonical map'' of \autoref{2.1.3} is. Suppose we are given a cocomplete category $\cC$, an element $A\in\cC$, an ordinal $\lambda$, and a $\lambda$-sequence $X:\lambda\to\cC$. For $\alpha\leq\beta<\lambda$, let $\iota_{\alpha,\beta}$ be the map $X_\alpha\to X_\beta$. Let $\eta:X\Rightarrow\ul{\colim X}$ be the colimit cone. By whiskering the colimit cone along the functor $\cC(A,-)$, we get a cone $\cC(A,\eta):\{\cC(A,X_\beta)\}_{\beta<\lambda}\Rightarrow\ul{\cC(A,\colim X)}$. Then if we let $\vare:\{\cC(A,X_\beta)\}_{\beta<\lambda}\Rightarrow\ul{\colim_{\beta<\lambda}\cC(A,X_\beta)}$ be the colimit cone, the universal property of the colimit gives us the canonical map $\ell:\colim_{\beta<\lambda}\cC(A,X_\beta)\to\cC(A,\colim X)$, so that the following diagram commutes:
\[\begin{tikzcd}
  {\cC(A,X_0)} && {\cC(A,X_1)} && \cdots && {\cC(A,X_\beta)} && \cdots \\
  \\
  &&&& {\colim_{\beta<\lambda}\cC(A,X_\beta)} \\
  \\
  &&&& {\cC(A,\colim X)}
  \arrow["{(\iota_{0,1})_*}", from=1-1, to=1-3]
  \arrow["{(\iota_{1,2})_*}", from=1-3, to=1-5]
  \arrow[from=1-5, to=1-7]
  \arrow["{(\iota_{\beta,\beta+1})_*}", from=1-7, to=1-9]
  \arrow["{\vare_\beta}"', from=1-7, to=3-5]
  \arrow["{\vare_1}", from=1-3, to=3-5]
  \arrow["{\vare_0}"{pos=0.3}, from=1-1, to=3-5]
  \arrow["{(\eta_\beta)_*}", from=1-7, to=5-5]
  \arrow["\ell", dashed, from=3-5, to=5-5]
  \arrow["{(\eta_1)_*}"'{pos=0.4}, from=1-3, to=5-5]
  \arrow["{(\eta_0)_*}"', from=1-1, to=5-5]
\end{tikzcd}\]
In particular, by \autoref{explicit_description_of_colimit_in_set}, we know elements of $\colim_{\beta<\lambda}\cC(A,X_\beta)$ are equivalence classes of arrows $f:A\to X_\beta$ for $\beta<\lambda$ under the relation $[f:A\to X_\beta]=[g:A\to X_{\beta'}]$ iff there exists $\beta''\geq\beta,\beta'$ with $\iota_{\beta,\beta''}\circ f=\iota_{\beta',\beta''}\circ g$, and the map $\vare_\beta$ sends an arrow $f\in\cC(A,X_\beta)$ to the element $[f]$. Then it follows that $\ell([f:A\to X_\beta])=\eta_\beta\circ f$. Thus, this gives us the following result:

\begin{proposition}\label{nicer_description_of_smallness_conditions}
  Given a cocomplete category $\cC$, a collection $\cD$ of arrows in $\cC$, an object $A$ in $\cC$, and a cardinal $\kappa$, $A$ is $\kappa$-small relative to $\cD$, if, for all $\kappa$-filtered ordinals $\lambda$ and all $\lambda$-sequences $X:\lambda\to\cC$ such that the map $X_{\beta}\to X_{\beta+1}$ belongs to $\cD$ for all $\beta+1<\lambda$, given any colimit $\colim X$ for $X$, the following holds:
  \begin{enumerate}[label=(\roman*)]
    \item Given arrows $f:A\to X_\alpha$ and $g:A\to X_{\beta}$ in $\cC$, if $f$ and $g$ agree in the colimit (i.e., if the compositions $A\xrightarrow{f} X_\alpha\to\colim X$ and $A\xrightarrow{g} X_{\beta}\to \colim X$ are equal), then $f$ and $g$ are equal in some stage of the colimit (i.e., there exists $\gamma<\lambda$ with $\alpha,\beta\leq\gamma$ such that the compositions $A\xrightarrow{f} X_\alpha\to X_\gamma$ and $A\xrightarrow{g} X_{\beta}\to X_{\gamma}$ are equal).
    \item Any arrow $f:A\to\colim X$ factors through some stage of the colimit (i.e., there exists $\beta<\lambda$ and an arrow $\wt f:A\to X_\beta$ such that the composition $A\xrightarrow{\wt f}X_\beta\to\colim X$ equals $f$).
  \end{enumerate}
  In terms of the canonical map $\colim_{\beta<\lambda}\cC(A,X_\beta)\to\cC(A,\colim X)$, the first condition shows injectivity, while the second shows surjectivity.
\end{proposition}

We will use the characterization of smallness given by this remark whenever proving smallness arguments, as in the following example.

\begin{example}[Hovey 2.1.5]\label{2.1.5}
  Every set is small. Indeed, if $A$ is a set we claim that $A$ is $|A|$-small. To see this, suppose $\lambda$ is an $|A|$-filtered ordinal, and $X$ is a $\lambda$-sequence of sets. First of all, by \autoref{explicit_description_of_colimit_in_set}, the elements of $\colim X$ are equivalence classes of elements $a\in X_\alpha$ where $a\in X_\alpha$ and $b\in X_\beta$ represent the same element of $\colim X$ iff there exists $\alpha,\beta\leq\gamma<\lambda$ so that $a$ and $b$ are sent to the same elements by the maps $X_\alpha\to X_\gamma$ and $X_\beta\to X_\gamma$, respectively. Now, we show the conditions of \autoref{nicer_description_of_smallness_conditions}.
  
  First, we need to show that given $\alpha,\beta<\lambda$, if $f:A\to X_\alpha$ and $g:A\to X_{\beta}$ such that the compositions $\ol f:A\xrightarrow{f}X_\alpha\to \colim X$ and $\ol g:A\xrightarrow{g}X_{\beta}\to \colim X$ are equal, then $f$ and $g$ are equal in some stage of the colimit. For each $a\in A$, since $\ol f(a)=\ol f(g)$ in $\colim X$, by the above characterization of $\colim X$, there exists $\gamma_a<\lambda$ with $\alpha,\beta\leq\gamma_a$ such that $f(a)$ and $g(a)$ are sent to the same element in $X_{\gamma_a}$ by the maps $X_\alpha\to X_{\gamma_a}$ and $X_\beta\to X_{\gamma_a}$, respectively. Then let $\gamma:=\sup_{a\in A}\gamma_a$. Since $\left|\{\gamma_a\}_{a\in A}\right|\leq|A|$ and $\lambda$ is $|A|$-filtered, necessarily $\gamma<\lambda$. Then clearly the compositions $A\xrightarrow{f}X_\alpha\to X_\gamma$ and $A\xrightarrow{g}X_\beta\to X_\gamma$ agree for all $a\in A$.

  Secondly, we wish to show that given a map $f:A\to\colim X$, that $f$ factors through $X_\beta\to \colim X$ for some $\beta<\lambda$. For each $a\in A$, by the explicit description of $\colim X$, there exists some $\beta_a<\lambda$ and some $x_a\in X_{\beta_a}$ such that $f(a)=[x_a]$. Then let $\beta:=\sup_{a\in A}\beta_a$, so $\beta<\lambda$ as $X$ is $|A|$-filtered. Now define $\wt f:A\to X_\beta$ like so: for $a\in A$, define $\wt f(a)\in X_\beta$ to be the image of $x_a$ along the map $X_{\beta_a}\to X_\beta$. Then clearly the composition $f':A\xrightarrow{\wt f}X_\beta\to\colim X$ is equal to $f$, by unravelling definitions.
\end{example}

\begin{definition}[Hovey Definition 2.1.7]
  Let $I$ be a class of maps in a category $\cC$.\begin{enumerate}
    \item A map is \textit{$I$-injective} if it has the right lifting property w.r.t.\ every map in $I$. The class of $I$-injective maps is denoted $I\inj$ (or $I\p$).
    \item A map is \textit{$I$-projective} if it has the left lifting property w.r.t.\ every map in $I$. The class of $I$-projective maps is denoted $I\proj$ (or $\p I$).
    \item A map is an \textit{$I$-cofibration} if it has the left lifting property w.r.t.\ every $I$-injective map. The class of $I$-cofibrations is the class $(I\inj)\proj$ and is denoted $I\cof$ (or $\p(I\p)$).
    \item A map is an \textit{$I$-fibration} if it has the right lifting property w.r.t.\ every $I$-projective map. The class of $I$-fibrations is the class $(I\proj)\inj$ and is denoted $I\fib$ (or $(\p I)\p$).
  \end{enumerate}
\end{definition}

The following is asserted in Hovey on pg.\ 30 following Definition 2.1.7, but not proven. We provide a proof.

\begin{lemma}\label{useful_LP_properties}
  Given classes $A$ and $B$ of maps in a category $\cC$ with $A\sseq B$, we have $A\sseq {\p(A\p)}$, $A\sseq (\p A)\p$, $(\p(A\p))\p=A\p$, $\p((\p A)\p)={\p A}$, $A\p\spseq B\p$, $\p A\spseq {\p B}$, ${\p(A\p)}\sseq {\p(B\p)}$, and $(\p A)\p\sseq (\p B)\p$.
\end{lemma}
\begin{proof}
  Each of these amount to unravelling definitions and are entirely straightforward.
\end{proof}

\begin{definition}[Hovey Definition 2.1.9]
  Let $I$ be a set of maps in a cocomplete category $\cC$. A \textit{relative $I$-cell complex} is a transfinite composition of pushouts of elements of $I$. That is, if $f:A\to B$ is a relative $I$-cell complex, then there is an ordinal $\lambda$ and a $\lambda$-sequence $X:\lambda\to\cC$ such that $f$ is the composition of $X$ and such that, for each $\beta$ such that $\beta+1<\lambda$, there is a pushout square
  \[\begin{tikzcd}
    {C_\beta} & {X_\beta} \\
    {D_\beta} & {X_{\beta+1}}
    \arrow[from=1-1, to=1-2]
    \arrow[from=1-2, to=2-2]
    \arrow[from=2-1, to=2-2]
    \arrow["\ulcorner"{anchor=center, pos=0.125, rotate=180}, draw=none, from=2-2, to=1-1]
    \arrow["{g_\beta}"', from=1-1, to=2-1]
  \end{tikzcd}\]
  with $g_\beta\in I$. We denote the collection of relative $I$-cell complexes by $I\cell$. We say that $A\in\cC$ is an \textit{$I$-cell complex} if the map $0\to A$ is a relative $I$-cell complex.
\end{definition}

\begin{lemma}\label{I-cell_closed_under_composition_with_isomorphisms}
  Let $\cC$ be a category and $I$ a class of morphisms in $\cC$. Then $I\cell$ is closed under composition with isomorphisms.
\end{lemma}
\begin{proof}[Proof Sketch]
  Suppose that $f:B\to C$ is an element of $I\cell$, and $h:A\to B$ and $g:C\to D$ are isomorphisms in $\cC$. We wish to show $f\circ h$ and $g\circ f$ are also elements of $I\cell$. Since $f\in I\cell$, there exists an ordinal $\lambda$, a $\lambda$-sequence $X$ with $X_0=B$, and a colimit cone $\eta:X\Rightarrow\underline C$, such that $\eta_0=f$. 
  
  First of all, construct a new cone $\eta':X\Rightarrow\underline D$ under $X$ where $\eta'_\beta:=g\circ\eta_\beta$. It is straightforward to verify that $\eta'$ is a colimit cone for $X$ since $\eta$ is a colimit cone and $g$ is an isomorphism. Thus, $g\circ f=g\circ\eta_0=\eta_0'\in I\cell$, as $\eta_0'$ is the composition of a sequence of pushouts of elements of $I$.

  On the other hand, we may construct a new $\lambda$-sequence $X'$ by defining $X'_0=A$, $X_\beta'=X_\beta$ for all $0<\beta<\lambda$, the map $X_0'\to X_\beta'$ for $0<\beta<\lambda$ to be the composition
  \[\begin{tikzcd}
    A & {B=X_0} & {X_\beta},
    \arrow["h", from=1-1, to=1-2]
    \arrow[from=1-2, to=1-3]
  \end{tikzcd}\]
  and the composition $X'_\alpha\to X'_\beta$ to simply be the same map $X_\alpha\to X_\beta$ for $0<\alpha\leq \beta<\lambda$. It is straightforward to verify that defines a $\lambda$-sequence, and that we may define a colimit cone $\eta':X'\Rightarrow\underline C$ by $\eta'_0=\eta_0\circ h=f\circ h$, and $\eta'_\beta=\eta_\beta$ for $0<\beta<\lambda$. Furthermore, clearly for all $1<\beta+1<\lambda$, we have the arrow $X_\beta'\to X_{\beta+1}'$ is a pushout of a map in $I$. Thus, in order to show $f\circ h\in I\cell$, it remains to show that the arrow $X_0'=A\to X_1=X_1'$ is a pushout of a map in $I$. Indeed, we know $B=X_0\to X_1$ is a pushout of a map $k:P\to Q$ in $I$, and it can be easily verified the diagram on the right is a pushout diagram as the left diagram is a pushout diagram and $h$ is an isomorphism
  \[\begin{tikzcd}[row sep=small,column sep=small]
    P && {X_0} && P & {X_0} & {X_0'} \\
    &&& \leadsto &&& {X_0} \\
    Q && {X_1} && Q && {X_1'}
    \arrow[from=1-3, to=3-3]
    \arrow[from=3-1, to=3-3]
    \arrow[from=1-1, to=1-3]
    \arrow["k"', from=1-1, to=3-1]
    \arrow["\ulcorner"{anchor=center, pos=0.125, rotate=180}, draw=none, from=3-3, to=1-1]
    \arrow[from=1-5, to=1-6]
    \arrow["h", from=1-7, to=2-7]
    \arrow[from=2-7, to=3-7]
    \arrow[from=1-5, to=3-5]
    \arrow[from=3-5, to=3-7]
    \arrow["{h^{-1}}", from=1-6, to=1-7]
    \arrow["\ulcorner"{anchor=center, pos=0.125, rotate=180}, draw=none, from=3-7, to=1-5]
  \end{tikzcd}\qedhere\]
\end{proof}

\begin{definition}\label{saturated}
  Let $\cC$ be a category and $I$ a collection of morphisms in $\cC$. Then if $I$ is closed under transfinite composition, pushouts, and retracts then we say $I$ is \textit{saturated}.
\end{definition}

\begin{lemma}
  Suppose $I$ is a class of maps in a cocomplete category $\cC$. Then $\p I$ is saturated. 
\end{lemma}
\begin{proof}
  \color{red}TODO.
\end{proof}

This yields the following Corollary:

\begin{corollary}[Hovey 2.1.10]\label{2.1.10}
  Given a cocomplete category $\cC$ and a class of maps $I$ in $\cC$, $I\cell\sseq{\p(I\p)}$.
\end{corollary}

\begin{theorem}[Small Object Argument, Hovey 2.1.14]\label{2.1.14}
  Suppose $\cC$ is a cocomplete categroy, and $I$ is a set of maps in $\cC$. Suppose the domains of the maps of $I$ are small relative to $I\cell$. Then there is a functorial factorization $(\gamma,\delta)$ on $\cC$ such that for all morphisms $f\in\cC$, the map $\gamma(f)$ is in $I\cell$ and the map $\delta(f)$ is in $I\inj$.
\end{theorem}
\begin{proof}
  \color{red}TODO.
\end{proof}

\begin{corollary}[Hovey 2.1.15]\label{2.1.15}
  Suppose that $I$ is a set of maps in a cocomplete category $\cC$. Suppose as well that the domains of $I$ are small relative to $I\cell$. Then given $f:A\to B$ in $\p(I\p)$, there is a $g:A\to C$ in $I\cell$ such that $f$ is a retract of $g$ by a map which fixes $A$.
\end{corollary}
\begin{proof}
  \color{red}TODO
\end{proof}

\begin{definition}[Hovey Definition 2.1.17]\label{2.1.17}
  Suppose $\cC$ is a model category. We say that $\cC$ is \textit{cofibrantly generated} if there are sets $I$ and $J$ of maps such that:\begin{enumerate}[label=\arabic*.,noitemsep,topsep=0pt]
    \item The domains of the maps of $I$ are small relative to $I\cell$;
    \item The domains of the maps of $J$ are small relative to $J\cell$;
    \item The class of fibrations is $J\p$; and
    \item The class of trivial fibrations is $I\p$.
  \end{enumerate}
  We refer to $I$ as the set of \textit{generating cofibrations} and to $J$ as the set of \textit{generating trivial cofibrations}. A cofibrantly generated model category is \textit{finitely generated} if we can choose the sets $I$ and $J$ above so that the domains and codomains of $I$ and $J$ are finite relative to $I\cell$.
\end{definition}

\begin{proposition}[Hovey Proposition 2.1.18]\label{2.1.18}
  Suppose $\cC$ is a cofibrantly generated model category, with generating cofibrations $I$ and generating trivial fibrations $J$.\begin{enumerate}[label=(\alph*),noitemsep,topsep=0pt]
    \item The cofibrations form the class ${\p(I\p)}$.
    \item Every cofibration is a retract of a relative $I$-cell complex.
    \item The domains of $I$ are small relative to the cofibrations.
    \item The trivial cofibrations form the class ${\p(J\p)}$.
    \item Every trivial cofibration is a retract of a relative $J$-cell complex.
    \item The domains of $J$ are small relative to the trivial cofibrations.
  \end{enumerate}
  If $\cC$ is fibrantly generated, then the domains and codomains of $I$ and $J$ are finite relative to the cofibrations.
\end{proposition}
\begin{proof}
  \color{red}TODO.
\end{proof}

\begin{theorem}[Hovey Theorem 2.1.19]\label{2.1.19}
  Suppose $\cC$ is a complete \& cocomplete category. Suppose $\cW$ is a subcategory of $\cC$, and $I$ and $J$ are sets of maps of $\cC$. Then there is a cofibrantly generated model structure on $\cC$ with $I$ as the set of generating cofibrations, $J$ as the set of generating trivial fibrations, and $\cW$ as the subcategory of weak equivalences if and only if the following conditions are satisfied.\begin{enumerate}[label=\arabic*.,noitemsep,topsep=0pt]
    \item The subcategory $\cW$ has the 2-of-3 property and is closed under retracts.
    \item The domains of $I$ are small relative to $I\cell$.
    \item The domains of $J$ are small relative to $J\cell$.
    \item $J\cell\sseq\cW\cap {\p(I\p)}$.
    \item $I\p\sseq\cW\cap J\p$.
    \item Either $\cW\cap {\p(I\p)}\sseq {\p(J\p)}$ or $\cW\cap J\p\sseq I\p$.
  \end{enumerate}
\end{theorem}
\begin{proof}
  \color{red}TODO.
\end{proof}

\section{Topological Spaces}

A map $f:X\to Y$ in $\Top$ is an \textit{inclusion} if it is continuous, injective, and for all $U\sseq X$ open, there is some $V\sseq Y$ open such that $f^{-1}(V)=U$. If $f$ is a closed inclusion and every point in $Y\setminus f(X)$ is closed, then we call $f$ a \textit{closed $T_1$ inclusion}. We will let $\cT$ denote the class of closed $T_1$ inclusions in $\Top$.

The symbol $D^n$ will denote the unit disk in $\bR^n$, and the symbol $S^{n-1}$ will denote the unit sphere in $\bR^n$, so that we have the boundary inclusions $S^{n-1}\into D^n$. In particular, for $n=0$ we let $D^0=\{0\}$ and $S^{-1}=\emptyset$.

Recall: If $F:\cJ\to\Top$ is a functor, where $\cJ$ is a small category, the limit of $F$ is obtained by taking the limit in the category of sets, and then topologizing it with the \textit{initial topology}, where if $\eta:\underline{\lim F}\Rightarrow F$ is the limit cone, then the topology on $\lim F$ is that with subbasis given by sets of the form $\eta_j^{-1}(U)$ where $j\in\cJ$ and $U\sseq F_j$ is open. Similarly, the colimit of $F$ is obtained by taking the colimit $\colim F$ in the category of sets and endowing it with the \textit{final topology}, where a set $U\sseq\colim F$ is open if and only if $\vare_j^{-1}(U)$ is open in $F_j$ for all $j\in\cJ$, where $\vare:F\Rightarrow\underline{\colim F}$ is the colimit cone (equivalently, a set $C\sseq\colim F$ is closed if and only if $\vare_j^{-1}(C)$ is closed in $F_j$ for all $j\in\cJ$).

% Source: https://ncatlab.org/nlab/show/locally+compact+topological+space, Munkres
Given a space $X$, we construct a functor $(-)^X:\Top\to\Top$ as follows: Given a space $Y$, define $Y^X$ to be the space whose underlying set is the set $\Top(X,Y)$ of continuous maps $X\to Y$, and the topology on $Y^X$ is the \textit{compact-open topology}, i.e., the topology with subbasis given by the sets of the form 
\[S(K,U):=\{f\in\Top(X,Y):f(K)\sseq U\}\]
for $K\sseq X$ compact and $U\sseq Z$ open. Given a continuous map $f:Y\to Z$, define the induced map $f_*:Y^X\to Z^X$ by $f_*(g):=f\circ g$. Unravelling definitions, we have that given $f:Y\to Z$ continuous, $f_*^{-1}(S(K,U))=S(K,f^{-1}(U))$ for all $K\sseq X$ compact and $U\sseq Z$ open, so that $f_*$ is continuous. Furthermore, $(-)^X$ is clearly functorial, by associativity and unitality of function composition.

Given a topological space $X$, we say that $X$ is \textit{locally compact} if for all points $x\in X$ and open neighborhoods $U$ of $x$, there exists an open set $V\sseq X$ with $x\in V$, $\ol V\sseq U$, and $\ol V$ compact. We claim that $(-)^X$ is right adjoint to $-\times X$ when $X$ is locally compact and Hausdorff. 

%First, we will prove the following Lemma:
%
%% sources: https://proofwiki.org/wiki/Equivalence_of_Definitions_of_Locally_Compact_Hausdorff_Space, https://math.stackexchange.com/a/1329871/930175
%\begin{lemma}
  %Let $X$ be a locally compact Hausdorff space. Then given a point $x\in X$ and an open set $U\sseq X$ containing $x$, there exists an open set $V\sseq X$ with $x\in V\sseq\ol V\sseq U$ with $\ol V$ compact.
%\end{lemma}
%\begin{proof}
  %Since $X$ is locally compact, there exists an open set $W\sseq X$ and a compact set $K\sseq X$ such that $x\in W\sseq K$. Then $U\cap K$ is open in $K$ by definition of the subspace topology, so that $K\setminus U$ is closed in $K$. Note furthermore that $K$ is Hausdorff, as given $a,b\in K$ not equal, since $X$ is Hausdorff there exist disjoint open neighborhoods $U_a,U_b\sseq X$ of $a$ and $b$, respectively, and $U_a\cap K$ and $U_b\cap K$ are disjoint open neighborhoods of $a$ and $b$ in $K$, as desired. Let $y\in U\cap K$. Then for each $z\in K\setminus U$, there exist disjoint open subsets $U_z,V_y\sseq K$ with $z\in U_z$ and $y\in V_y$. Let $\cF:=\{U_z:z\in K\setminus U\}$.
%\end{proof}

\begin{proposition}\label{locally_compact_hausdorff_spaces_are_exponentiable}
  If $X$ is a locally compact Hausdorff space, then functor $-\times X$ is left adjoint to $(-)^X$ (so that in particular $-\times X$ preserves colimits).
\end{proposition}
\begin{proof}
  We start by constructing the counit and unit of the adjunction. Given a space $Z$, define the counit $\vare_Z:X\times Z^X\to Z$ to be the evaluation function, taking a pair $(x,f)\mapsto f(x)$. First, we claim $\vare_Z$ is continuous. Suppose we are given an open set $V\sseq Z$ and a point $(x,f)\in \vare_Z^{-1}(U)$ (so $f(x)\in V$). Since $f$ is continuous and $X$ is locally compact, there exists an open set $U\sseq X$ containing $x$ such that $x\in U\sseq\ol U\sseq f^{-1}(V)$ with $\ol U$ compact. Then consider the open set $U\times S(\ol U,V)$ in $X\times Y^X$. First of all, $(x,f)\in U\times S(\ol U,V)$, as $x\in U$ and $\ol U\sseq f^{-1}(V)$, so that $f(\ol U)\sseq V$ meaning $f\in S(\ol U,V)$. Furthermore, given $(y,g)\in U\times S(\ol U,V)$, we have $\vare_Z(y,g)=g(y)\in g(U)\sseq g(\ol U)\sseq V$, so $U\times S(\ol U,V)$ is an open neighborhood of $x$ contained in $\vare_Z^{-1}(V)$, as desired. Hence, $\vare_Z$ is continuous. It remains to show naturality. Given a map $f:Z\to W$, we wish to show the following diagram commutes:
  \[\begin{tikzcd}
    {X\times Z^X} & Z \\
    {X\times W^X} & W
    \arrow["{\vare_Z}", from=1-1, to=1-2]
    \arrow["{\id_X\times f_*}"', from=1-1, to=2-1]
    \arrow["{\vare_W}", from=2-1, to=2-2]
    \arrow["f", from=1-2, to=2-2]
  \end{tikzcd}\]
  Indeed, chasing an element $(x,g)$ around the diagram yields:
  \[\begin{tikzcd}
    {(x,g)} & {g(x)} \\
    {(x,f\circ g)} & {f(g(x))}
    \arrow[maps to, from=1-1, to=1-2]
    \arrow[maps to, from=1-2, to=2-2]
    \arrow[maps to, from=1-1, to=2-1]
    \arrow[maps to, from=2-1, to=2-2]
  \end{tikzcd}\]
  so it does indeed commute.

  Now we wish to define the unit $\eta_Y:Y\to(Y\times X)^X$. Given $y\in Y$, define $\eta_Y(y)\in (Y\times X)^X$ by $\eta_Y(y)(x):=(y,x)$. First of all, for it to be true that $\eta_Y(y)\in(X\times Y)^X$, it must be true that $\eta_Y(y)$ is continuous. Indeed, this is clear as $\eta_Y$ is obtained as the product map $y\times\id_X:X\to Y\times X$, where $y$ represents the constant function on $y$ (which is obviously continuous). Furthermore, $\eta_Y$ itself is continuous: given $K\sseq X$ compact and $U\sseq Y\times X$ open, we wish to show that $\eta_Y^{-1}(S(K,U))$ is open in $Y$. It suffices to show that given $y\in\eta_Y^{-1}(S(K,U))$, there exists an open neighborhood $W$ of $y$ that is mapped by $\eta_Y$ into $S(K,U)$. Since $y\in\eta_Y^{-1}(S(K,U))$, $\eta_Y(y)(K)=\{y\}\times K\sseq U$. Then $U\cap (Y\times K)$ is an open set in the subspace $Y\times K$ containing the slice $\{y\}\times K$. By definition of the product topology, for each $k\in K$, there exist open sets $W_k\sseq Y$ and $V_k\sseq K$ such that $(y,k)\in W_k\times V_k\sseq U\cap(Y\times K)$. Then the $V_k$'s form an open cover of $K$, which is compact, so that there exist $k_1,\ldots,k_n\in K$ with $V_{k_1}\cup\cdots\cup V_{k_n}=K$. Hence if we define $W:=W_{k_1}\cap\cdots\cap W_{k_n}$, then $\{y\}\times K\sseq W\times K\sseq U\cap(Y\times K)$, and $W$ is open in $Y$ as it is a finite intersection of open sets. Then for all $w\in W$, $\eta_Y(w)(K)=\{w\}\times K\sseq W\times K\sseq U$. Hence, indeed $\eta_Y$ is continuous. It remains to show naturality. Given a map $f:Y\to W$, we wish to show the following diagram commutes:
  % https://q.uiver.app/?q=WzAsNCxbMCwwLCJZIl0sWzAsMSwiVyJdLFsxLDEsIihXXFx0aW1lcyBYKV5YIl0sWzEsMCwiKFlcXHRpbWVzICBYKV5YIl0sWzAsMSwiZiIsMl0sWzEsMiwiXFxldGFfVyJdLFswLDMsIlxcZXRhX1kiXSxbMywyLCIoZlxcdGltZXNcXGlkX1gpXyoiXV0=
  \[\begin{tikzcd}
    Y & {(Y\times  X)^X} \\
    W & {(W\times X)^X}
    \arrow["f"', from=1-1, to=2-1]
    \arrow["{\eta_W}", from=2-1, to=2-2]
    \arrow["{\eta_Y}", from=1-1, to=1-2]
    \arrow["{(f\times\id_X)_*}", from=1-2, to=2-2]
  \end{tikzcd}\]
  Indeed, chasing an element $y$ around the top of the diagram yields the function obtained as the composition $x\mapsto (y,x)\mapsto f\times\id_X(y,x)=(f(y),x)$, while chasing around the bottom of the diagram more directly yields the function $x\mapsto (f(y),x)$.


  Now that we have constructed the unit and counit, it remains to verify the counit-unit equations, i.e., that for each $Y\in\Top$ that $\vare_{Y\times X}\circ(\eta_Y\times\id_X)=\id_{Y\times X}$ and $(\vare_Y)_*\circ\eta_{Y^X}=\id_{Y^X}$. First of all, given $(y,x)\in Y\times X$, we have
  \[(\vare_{Y\times X}\circ (\eta_Y\times\id_X))(y,x)=\vare_{X\times Y}(\eta_Y(y),x)=\eta_Y(y)(x)=(y,x).\]
  On the other hand, given $f\in Y^X$, we have
  \[(\vare_Y)_*(\eta_{Y^X}(f))=(\vare_Y)_*([x\mapsto (f,x)])=[x\mapsto (f,x)\mapsto\vare_Y(f,x)=f(x)]=f.\]
  Hence, indeed $\vare$ and $\eta$ form the counit and unit for the adjoint pair $(-\times X,(-)^X)$.
\end{proof}

Now that we have gotten some topological preliminaries out of the way, we are ready to define the model structure.

\begin{definition}
  A map $f:X\to Y$ in $\Top$ is called a \textit{weak equivalence} if
  \[\pi_n(f,x):\pi_n(X,x)\to\pi_n(Y,f(x))\]
  is an isomorphism for all $n\geq0$ and for all $x\in X$. We will write $\cW$ to refer to the class of all weak equivalences in $\Top$.

  Define the set of maps $I'$ to consist of all the boundary inclusion $S^{n-1}\into D^n$ for all $n\geq0$, and define the set $J$ to consist of all the inclusions $D^n\into D^n\times I$ mapping $x\mapsto(x,0)$ for $n\geq0$. Then a map $f$ will be called a \textit{cofibration} if it is in $I'\cof={\p( I'\p )}$, and a \textit{fibration} if it is in $J\inj=J\p$.

  A map in $I'\cell$ is usually called a \textit{relative cell complex}; a relative CW-complex is a special case of a relative cell complex, where, in particular, the cells can be attached in order of their dimension. Note that in particular maps of $J$ are relative CW complexes, hence are relative $I'$-cell complexes. 
  %Thus $J\cof\sseq I'\cof$.
  A fibration is often known as a \textit{Serre fibration} in the literature.
\end{definition}

\begin{theorem}[Hovey Theorem 2.4.19]\label{2.4.19}
  There is a finitely generated model structure on $\Top$ with $I'$ as the set of generating cofibrations, $J$ as the set of generating trivial cofibrations, and the cofibrations, fibrations, and weak equivalences as above. Every object of $\Top$ is fibrant, and the cofibrant objects are retracts of relative cell complexes.
\end{theorem}
\begin{proof}
  We will apply \autoref{2.1.19} to get that there is a cofibrantly generated model structure on $\Top$ with $I'$ as the set of generating cofibrations, $J$ as the set of generating trivial fibrations, and $\cW$ as the subcategory of weak equivalences. The six requirements outlined in the theorem will be verified like so:
  \begin{enumerate}[label=\arabic*.,noitemsep,topsep=0pt]
    \item $\cW$ is a subcategory of $\cC$ which has the 2-of-3 property and is closed under retracts: \autoref{2.4.4}.
    \item The domains of $I'$ are small relative to $I'\cell$: \autoref{domains_of_I'/J_small_rel_I'-cell/J-cell}.
    \item The domains of $J$ are small relative to $J\cell$: \autoref{domains_of_I'/J_small_rel_I'-cell/J-cell}.
    \item $J\cell\sseq\cW\cap {\p( I'\p )}$: In \autoref{2.4.9}, we will show ${\p({J}\p)}\sseq\cW\cap {\p( I'\p )}$, and by \autoref{2.1.10} $J\cell\sseq {\p({J}\p)}$.
    \item $ I'\p \sseq\cW\cap J\p$: \autoref{2.4.10}
    \item $\cW\cap J\p\sseq  I'\p $: \autoref{2.4.12}
  \end{enumerate}
  It will follow by the definition of a cofibrantly generated model structure (\autoref{2.1.17}) that the fibrations in this model structure are given by $J\p$, which is precisely how we defined it. By \autoref{2.1.18}, the class of cofibrations will be given by ${\p( I'\p )}$, which is likewise exactly how we defined them.

  In \autoref{2.4.2}, we will show that compact spaces are finite relative to the class $\cT$ of closed $T_1$ inclusions. Hence, this model structure will be finitely generated, as the domains and codomains of $I'$ and $J$ are all compact, and by the reasoning given above we will have shown $I'\cell\sseq\cT$.
  
  We will show that every object of $\Top$ is fibrant in \autoref{2.4.14}. 
\end{proof}

\begin{lemma}\label{lambda_sequence_of_inclusions_makes_everything_an_inclusion}
  Let $\lambda$ be an ordinal, and $X$ a $\lambda$-sequence in $\Top$. 
  Then:
  \begin{enumerate}[label=(\roman*)]
    \item If $X$ is a $\lambda$-sequence of injections, then $X_\alpha\to X_\beta$ is an injective for all $\alpha\leq\beta<\lambda$.
    \item If $X$ is a $\lambda$-sequence of inclusions, then the map $X_\alpha\to X_\beta$ is an inclusion for all $\alpha\leq\beta<\lambda$.
    \item If $X$ is a $\lambda$-sequence of closed $T_1$ inclusions, then the map $X_\alpha\to X_\beta$ is a closed $T_1$ inclusion for all $\alpha\leq\beta<\lambda$.
  \end{enumerate}
\end{lemma}
\begin{proof}
  In what follows, given $\alpha\leq\beta<\lambda$, let $\iota_{\alpha,\beta}$ denote the map $X_\alpha\to X_\beta$.
  \begin{enumerate}[label=(\roman*),listparindent=\parindent,parsep=0pt]
    \item Let $\alpha<\lambda$. We perform a proof by transfinite induction on $\beta$ for $\alpha\leq\beta<\lambda$ that $\iota_{\alpha,\beta}:X_\alpha\to X_\beta$ is injective. For the zero case, clearly $\iota_{\alpha,\alpha}=\id_{X_\alpha}$ is injective. Supposing $\iota_{\alpha,\beta}$ is injective for some $\alpha<\beta+1<\lambda$, we have $\iota_{\alpha,\beta+1}=\iota_{\beta,\beta+1}\circ\iota_{\alpha,\beta}$ is a composition of injections, and is therefore clearly injective itself. Finally, suppose $\gamma$ is a limit ordinal with $\alpha\leq\gamma<\lambda$ such that $\iota_{\alpha,\beta}$ is injective for all $\alpha\leq\beta<\gamma$. We claim $\iota_{\alpha,\gamma}$ is injective. Since $X_\gamma$ is colimit preserving and $\gamma$ is a limit ordinal, $X_\gamma$ is the colimit of the diagram $\{X_\beta\}_{\beta<\gamma}$ via the maps $\iota_{\beta,\gamma}$, so that in particular by \autoref{explicit_description_of_colimit_in_set} and the fact that the forgetful functor $\Top\to\Set$ preserves colimits, given $a,b\in X_\alpha$ with $\iota_{\alpha,\gamma}(a)=\iota_{\alpha,\gamma}(b)$, there exists some $\beta<\gamma$ with $\iota_{\alpha,\beta}(a)=\iota_{\alpha,\beta}(b)$, and $\iota_{\alpha,\beta}$ is injective for all $\beta<\gamma$, so it must have been true $a=b$ in $X_\alpha$. 
    \item By part(i), we know that $\iota_{\alpha,\beta}$ is injective for $\alpha\leq\beta<\lambda$. Thus it suffices to prove the following statement: For all $\alpha<\lambda$ and $U\sseq X_\alpha$, for all $\alpha\leq\beta<\lambda$, there exists $U_\beta\sseq X_\beta$ with $U_\alpha=U$ such that for all $\alpha\leq\beta'\leq\beta<\lambda$, $\iota_{\beta',\beta}^{-1}(U_\beta)=U_{\beta'}$. We prove this by transfinite recursion on $\alpha\leq\beta<\lambda$.
    
    The zero case has been taken care of: $U_\alpha=U$. For the sucessor case, given $\alpha<\beta+1<\lambda$, supposing $U_\beta$ has been defined with the desired properties, since $\iota_{\beta,\beta+1}$ is an inclusion, there exists $U_{\beta+1}\sseq X_{\beta+1}$ with $\iota_{\beta,\beta+1}^{-1}(U_{\beta+1})=U_\beta$. Then given $\alpha\leq\beta'\leq\beta+1$, we have
    \[\iota_{\beta',\beta+1}^{-1}(U_{\beta+1})=(\iota_{\beta,\beta+1}\circ\iota_{\beta',\beta})^{-1}(U_{\beta+1})=\iota_{\beta',\beta}^{-1}(\iota_{\beta,\beta+1}^{-1}(U_{\beta+1}))=\iota_{\beta',\beta}^{-1}(U_\beta)=U_{\beta'}.\]
    Finally, the limit case. Suppose $\gamma$ is a limit ordinal with $\alpha<\gamma\leq\lambda$, and suppose $U_\beta$ has been constructed with the desired properties for $\alpha\leq\beta<\gamma$. We wish to define $U_\gamma$. Since $X$ is colimit preserving and $\gamma=\sup_{\alpha\leq\beta<\gamma}\beta$, the maps $\iota_{\beta,\gamma}$ for $\alpha\leq\beta<\gamma$ form a colimit cone for the diagram $\{X_\beta\}_{\alpha\leq\beta<\gamma}$. Let $S=\{0,1\}$ be the Sierpinski space whose open sets are $\{\emptyset,\{1\},\{0,1\}\}$. For $\alpha\leq\beta<\gamma$, define a map $s_\beta:X_\beta\to S$ mapping everything in $U_\beta$ to $1$ and every other point to $0$. Each $s_\beta$ is clearly continuous, as $s_\beta^{-1}(1)=U_\beta$. Furthermore, we claim the $s_\beta$'s form a cone under the diagram $\{X_\beta\}_{\alpha\leq\beta<\gamma}$, i.e., that given $\alpha\leq\beta'\leq\beta<\gamma$, the following diagram commutes
    \[\begin{tikzcd}
      {X_{\beta'} } && {X_\beta} \\
      & S
      \arrow["{\iota_{\beta',\beta}}", from=1-1, to=1-3]
      \arrow["{s_\beta}", from=1-3, to=2-2]
      \arrow["{s_{\beta'}}"', from=1-1, to=2-2]
    \end{tikzcd}\]
    To see this, let $x\in X_{\beta'}$. If $x\in U_{\beta'}=\iota_{\beta',\beta}^{-1}(U_\beta)$, then $\iota_{\beta',\beta}(x)\in U_\beta$, so $s_\beta(\iota_{\beta',\beta}(x))=1=s_{\beta'}(x)$. Conversely, if $x\in X_{\beta'}\setminus U_{\beta'}=X_{\beta'}\setminus\iota^{-1}_{\beta',\beta}(U_\beta)$, then $x\notin \iota_{\beta',\beta}^{-1}(U_\beta)$, so $\iota_{\beta',\beta}(x)\notin U_{\beta}$, meaning $s_\beta(\iota_{\beta',\beta}(x))=0=s_{\beta'}(0)$. Hence, the $s_\beta$'s do indeed form a cone under $\{X_\beta\}_{\alpha\leq\beta<\gamma}$, so by universal property of the colimit there exists a unique map $\ell:X_\gamma\to S$ such that $s_\beta=\ell\circ\iota_{\beta,\gamma}$ for all $\alpha\leq\beta<\gamma$. Define $U_\gamma:=\ell^{-1}(1)$, which is open as $\{1\}$ is open in $S$. It remains to show that for all $\alpha\leq\beta\leq\gamma$ that $\iota_{\beta,\gamma}^{-1}(U_\gamma)=U_\beta$. Indeed, we have
    \[\iota_{\beta,\gamma}^{-1}(U_\gamma)=\iota_{\beta,\gamma}^{-1}(\ell^{-1}(1))=(\ell\circ\iota_{\beta,\gamma})^{-1}(1)=s_\beta^{-1}(1)=U_\beta.\]
    \item By part (ii), we know that $\iota_{\alpha,\beta}$ is an inclusion for $\alpha\leq\beta<\lambda$. Fix $\alpha<\lambda$. We perform transfinite induction on $\alpha\leq\beta<\lambda$ to show that $\iota_{\alpha,\beta}$ is a closed $T_1$ inclusion, assuming it is already an inclusion. For the zero case, clearly $\iota_{\alpha,\alpha}=\id_{X_\alpha}$ is closed, and vacuosuly very point in $X_\alpha\setminus\iota_{\alpha,\alpha}(X_\alpha)=\emptyset$ is a closed point. For the successor case, supposing $\iota_{\alpha,\beta}:X_\alpha\to X_\beta$ is a closed $T_1$ inclusion, we wish to show that $\iota_{\alpha,\beta+1}:X_\alpha\to X_{\beta+1}$ is a closed $T_1$ inclusion. Since $\iota_{\alpha,\beta+1}=\iota_{\beta,\beta+1}\circ\iota_{\alpha,\beta}$ is a composition of closed $T_1$ inclusions, it is clearly closed. It remains to show that every point in $X_{\beta+1}\setminus\iota_{\alpha,\beta+1}(X_\alpha)$ is closed in $X_{\beta+1}$. Indeed, let $x\in X_{\beta+1}\setminus\iota_{\alpha,\beta+1}(X_\alpha)$. First, if $x\in X_{\beta+1}\setminus\iota_{\beta,\beta+1}(X_\beta)$, we are done, as $\iota_{\beta,\beta+1}$ is a closed $T_1$ inclusion. Hence, we may assume that $x\in\iota_{\beta,\beta+1}(X_\beta)$, so there exists some $y\in X_\beta$ such that $\iota_{\beta,\beta+1}(y)=x$. Since $\iota_{\beta,\beta+1}$ is closed, in order to show $x$ is a closed point in $X_{\beta+1}$, it suffices to show that $y$ is a closed point in $X_\beta$. Since $\iota_{\alpha,\beta}$ is a closed $T_1$ inclusion, it further suffices to show that $y$ is not in the image of $\iota_{\alpha,\beta}$. Suppose for the sake of a contradiction that there existed $z\in X_\alpha$ with $\iota_{\alpha,\beta}(z)=y$. Then we would have
    \[\iota_{\alpha,\beta+1}(z)=\iota_{\beta,\beta+1}(\iota_{\alpha,\beta}(z))=\iota_{\beta,\beta+1}(y)=x,\]
    a contradiction of the fact that $x\in X_{\beta+1}\setminus\iota_{\alpha,\beta+1}(X_\alpha)$. Hence, it must have been true that $y$ is not in the image of $\iota_{\alpha,\beta}$ in the first place, the desired result. Finally, the limit case. Suppose $\gamma$ is a limit ordinal with $\alpha<\gamma\leq\lambda$ such that $\iota_{\alpha,\beta}$ is a closed $T_1$ inclusion for all $\alpha\leq\beta<\gamma$. Then we wish to show $\iota_{\alpha,\gamma}$ is a closed $T_1$ inclusion.

    First, we show $\iota_{\alpha,\gamma}$ is closed. Let $C\sseq X_\alpha$ be closed. Since $\gamma=\sup_{\alpha\leq\beta<\gamma}\beta$ and $X$ is colimit-preserving, $X_\gamma$ is the colimit of the $X_\beta$'s for $\alpha\leq\beta<\gamma$ via the maps $\iota_{\beta,\gamma}$, and the topology on $X_\gamma$ is the final topology induced by these maps. Hence, in order to show $\iota_{\alpha,\gamma}(C)$ is closed in $X_\gamma$, it suffices to show that $\iota_{\beta,\gamma}^{-1}(\iota_{\alpha,\gamma}(C))$ is closed in $X_\beta$ for all $\alpha\leq\beta<\gamma$. It further suffices to show that $\iota_{\beta,\gamma}^{-1}(\iota_{\alpha,\gamma}(C))=\iota_{\alpha,\beta}(C)$, as $\iota_{\alpha,\beta}$ is closed. First, suppose $x\in\iota_{\beta,\gamma}^{-1}(\iota_{\alpha,\gamma}(C))$, so $\iota_{\beta,\gamma}(x)=\iota_{\alpha,\gamma}(c)$ for some $c\in C$. Then since the forgetful functor $\Top\to\Set$ preserves colimits, by the explicit description of the colimit in $\Set$ (\autoref{explicit_description_of_colimit_in_set}), there exists $\mu$ with $\alpha,\beta\leq\mu<\gamma$ such that $\iota_{\beta,\mu}(x)=\iota_{\alpha,\mu}(c)$. But $\iota_{\alpha,\mu}=\iota_{\beta,\mu}\circ\iota_{\alpha,\beta}$, and $\iota_{\beta,\mu}$ is injective (by (i)) so $x=\iota_{\alpha,\beta}(c)$, meaning $x\in\iota_{\alpha,\beta}(C)$, as desired. Conversely, suppose we are given $c\in C$, then we wish to show $\iota_{\alpha,\beta}(c)\in\iota_{\beta,\gamma}^{-1}(\iota_{\alpha,\gamma}(C))$, i.e., that $\iota_{\beta,\gamma}(\iota_{\alpha,\beta}(c))\in\iota_{\alpha,\gamma(C)}$. This follows immediately as $\iota_{\beta,\gamma}\circ\iota_{\alpha,\beta}=\iota_{\alpha,\gamma}$.

    Lastly, we show that for all $x\in X_\gamma\setminus\iota_{\alpha,\gamma}(X_\alpha)$ that $x$ is a closed point in $X_\gamma$. Again by the description of the colimit in $\Set$ (\autoref{explicit_description_of_colimit_in_set}), the fact that the forgetful functor $\Top\to\Set$ preserves colimits, and that $X$ preserves colimits, we know that every point in $X_\gamma$ is in the image of some $\iota_{\beta,\gamma}$ for some $\alpha\leq\beta<\gamma$. Hence, there exists some $\alpha<\beta<\gamma$ and a point $y\in X_\beta$ with $\iota_{\beta,\gamma}(y)=x$. By the preceeding paragraph, $\iota_{\beta,\gamma}$ is closed, so in order to show $x$ is a closed point in $X_\gamma$ it suffices to show that $y$ is a closed point in $X_\beta$. It further suffices to show that $y\in X_\beta\setminus\iota_{\alpha,\beta}(X_\alpha)$, as $\iota_{\alpha,\beta}$ is a closed $T_1$ inclusion. Suppose for the sake of a contradiction that there existed some $z\in X_\alpha$ such that $\iota_{\alpha,\beta}(z)=y$. Then we would have
    \[\iota_{\alpha,\gamma}(z)=\iota_{\beta,\gamma}(\iota_{\alpha,\beta}(z))=\iota_{\beta,\gamma}(y)=x,\] 
    a contradiction of the fact that $x\in X_\gamma\setminus\iota_{\alpha,\gamma}(X_\alpha)$. Hence, $y$ must not have been in the image of $\iota_{\alpha,\beta}$ in the first place, as desired.
    \qedhere
  \end{enumerate}
\end{proof}

This result, by \autoref{condition_for_family_of_arrows_to_be_closed_under_transfinite_composition} and \autoref{stronger_characterization_of_closure_under_transfinite_composition}, gives the following corollaries:

\begin{corollary}\label{inclusions_closed_under_transfinite_composition}
  The class of injective maps (resp.\ inclusions, closed $T_1$ inclusions) in $\Top$ is closed under transfinite composition.
\end{corollary}

\begin{corollary}\label{colimit_legs_of_lambda_sequence_of_inclusions_are_inclusions}
  Let $\lambda$ be an ordinal, and $X$ be a $\lambda$-sequence in $\Top$. 
  Then:
  \begin{enumerate}[label=(\roman*),noitemsep]
    \item If $X$ is a $\lambda$-sequence of injections, then the canonical map $X_\alpha\to\colim X$ is an injection for all $\alpha<\lambda$.
    \item If $X$ is a $\lambda$-sequence of inclusions, then the canonical map $X_\alpha\to\colim X$ is an inclusion for all $\alpha<\lambda$.
    \item If $X$ is a $\lambda$-sequence of closed $T_1$ inclusions, then the canonical map $X_\alpha\to\colim X$ is a closed $T_1$ inclusion for all $\alpha<\lambda$.
  \end{enumerate} 
\end{corollary}

%\begin{proof}
  %\begin{enumerate}[label=(\roman*),listparindent=\parindent,parsep=0pt]
    %\item Suppose we are given $\alpha<\lambda$ and $a,b\in X_\alpha$ with $j_\alpha(a)=j_\alpha(b)$. Then by the fact that the forgetful functor $\Top\to\Set$ preserves colimits and \autoref{explicit_description_of_colimit_in_set}, there exists $\beta\geq\alpha$ such that $\iota_{\alpha,\beta}(a)=\iota_{\alpha,\beta}(b)$. But $\iota_{\alpha,\beta}$ is injective for all $\alpha\leq\beta<\lambda$ by \autoref{lambda_sequence_of_inclusions_makes_everything_an_inclusion}(i), so it must have been true $a=b$ in the first place. 
    %\item Suppose we are given $\alpha<\lambda$ and an open set $U\sseq X_\alpha$. We wish to show there exists $V\sseq\colim X$ open with $j_\alpha^{-1}(V)=U$. By \autoref{lambda_sequence_of_inclusions_makes_everything_an_inclusion}(ii), for each $\alpha\leq\beta<\lambda$, there exists $V_\beta\sseq X_\beta$ open with $\iota_{\alpha,\beta}^{-1}(V_\beta)=U$. Now, since $\lambda=\sup_{\alpha\leq\beta<\lambda}\beta$ and $X$ is colimit preserving, $X_\lambda$ is a colimit for the ``subdiagram'' $\{X_\beta\}_{\alpha\leq\beta<\lambda}$. Let $S=\{0,1\}$ be the Sierpinski space, whose open sets are $\{\emptyset,\{1\},\{0,1\}\}$. Then for $\alpha\leq\beta<\lambda$, define $s_\beta:X_\beta\to S$ mapping $V_\beta$ to $1$ and every point in $X_\beta\setminus V_\beta$ to $0$. As in the proof of \autoref{lambda_sequence_of_inclusions_makes_everything_an_inclusion}(ii), by the universal property of the colimit, the $s_\beta$'s induce a (continuous) map $\ell:\colim X\to S$ such that $s_\beta=\ell\circ j_\beta$ for all $\alpha\leq\beta<\lambda$. Then define $V:=\ell^{-1}(1)$, which is open as $\{1\}$ is open in $S$. Then in particular,
    %\[j_\alpha^{-1}(V)=j_\alpha^{-1}(\ell^{-1}(1))=(\ell\circ j_\alpha)^{-1}(1)=s_\alpha^{-1}(1)=V_\alpha=U,\]
    %\item\color{red}TODO.
  %\end{enumerate}
%
  %In particular, if $X$ is a $\lambda$-sequence of inclusions, then we have shown that for all $\alpha<\lambda$, $j_\alpha:X_\alpha\to\colim X$ is injective, and for all $U\sseq X_\alpha$ open, there exists $V\sseq\colim X$ open with $j_\alpha^{-1}(V)=U$. Thus, $j_\alpha$ is an inclusion.
%\end{proof}

\begin{lemma}[Hovey 2.4.1]\label{2.4.1}
  Every topological space is small relative to the inclusions.
\end{lemma}
\begin{proof}
  We claim that every topological space $A$ is $|A|$-small relative to the inclusions. We use the characterization of smallness afforded by \autoref{nicer_description_of_smallness_conditions}. Let $\lambda$ be an $|A|$-filtered ordinal, and let $X:\lambda\to\Top$ be a $\lambda$-sequence so that $X_\beta\to X_{\beta+1}$ is an inclusion for all $\beta+1<\lambda$. Recall that the forgetful functor $\Top\to\Set$ is forgetful, so elements of $\colim X$ are equivalence classes of elements $a\in X_\alpha$ for $\alpha<\lambda$, where $a\in X_\alpha$ and $b\in X_\beta$ represent the same equivalence class iff there exists $\alpha,\beta\leq\gamma<\lambda$ so that $a$ and $b$ are sent to the same element by the maps $X_\alpha\to X_\gamma$ and $X_\beta\to X_\gamma$, respectively.

  First, suppose $f:A\to X_\alpha$ and $g:A\to X_\beta$ are continuous maps such that the compositions $A\xrightarrow fX_\alpha\to\colim X$ and $A\xrightarrow gX_\beta\to \colim X$ are equal. Then the same proof given in \autoref{2.1.5} works to show that $f$ and $g$ are equal in some stage of the colimit, as desired.

  Conversely, suppose we are given a (continuous) map $f:A\to\colim X$. As in the proof of \autoref{2.1.5}, we may find some $\beta<\lambda$ and a map of sets $\wt f:A\to X_\beta$ such that the composition $A\xrightarrow{\wt f}X_\beta\xrightarrow j\colim X$ is equal to $f$ (note we have given the canonical map $X_\beta\to\colim X$ the name $j$). It remains to show that $\wt f$ is continuous. Let $U\sseq X_\beta$ be open. Since $j$ is an inclusion (\autoref{colimit_legs_of_lambda_sequence_of_inclusions_are_inclusions}), there exists $V\sseq\colim X_\beta$ open such that $j^{-1}(V)=U$. Then ${\wt f}^{-1}(U)=\wt f^{-1}(j^{-1}(V))=(j\circ\wt f)^{-1}(V)=f^{-1}(V)$, and $f$ is continuous, so $\wt f^{-1}(U)=f^{-1}(V)$ is open. Thus $\wt f$ is continuous, as desired.
\end{proof}

\begin{proposition}[Hovey 2.4.2]\label{2.4.2}
  Compact topological spaces are finite relative to the class $\cT$ of closed $T_1$ inclusions.
\end{proposition}
\begin{proof}
  We use the characterization of smallness afforded by \autoref{nicer_description_of_smallness_conditions}. Let $\lambda$ be a limit ordinal, and let $X:\lambda\to\Top$ be a $\lambda$-sequence so that $X_\beta\to X_{\beta+1}$ is a closed $T_1$ inclusion for all $\beta+1<\lambda$. Let $j:X\Rightarrow\ul{\colim X}$ is a colimit cone for $X$. Recall that the forgetful functor $\Top\to\Set$ is forgetful, so by \autoref{explicit_description_of_colimit_in_set} elements of $\colim X$ are equivalence classes of elements $a\in X_\alpha$ for $\alpha<\lambda$, where $a\in X_\alpha$ and $b\in X_\beta$ represent the same equivalence class iff there exists $\alpha,\beta\leq\gamma<\lambda$ so that $a$ and $b$ are sent to the same element by the maps $X_\alpha\to X_\gamma$ and $X_\beta\to X_\gamma$, respectively.

  First, we show condition (i) of \autoref{nicer_description_of_smallness_conditions}. Let $j:X\Rightarrow\ul{\colim X}$ be a colimt cone for $X$, and suppose we are given maps $f:A\to X_\alpha$ and $g:A\to X_\beta$ such that $j_\alpha\circ f=j_\beta\circ g$. WLOG, suppose $\alpha\leq\beta$. Then
  \[j_\beta\circ\iota_{\alpha,\beta}\circ f=j_\alpha\circ f=j_{\beta}\circ g,\]
  and $j_\beta$ is injective (\autoref{colimit_legs_of_lambda_sequence_of_inclusions_are_inclusions}) and therefore a monomorphism in $\Top$, so $\iota_{\alpha,\beta}\circ f=g$, meaning $f$ and $g$ do indeed agree in some stage of the colimit, as desired.

  Now we show condition (ii) of \autoref{nicer_description_of_smallness_conditions}. Let $f:A\to\colim X$ be a continuous map. In order to show $f$ factors through some $X_\alpha$, we first claim it is sufficient for there to be some $\alpha<\lambda$ with $f(A)\sseq j_\alpha(X_\alpha)$. Given an ordinal $\alpha<\lambda$, for each $a\in A$, there exists $\wt f(a)\in X_\alpha$ such that $j_\alpha(\wt f(a))=f(a)$. Thus we have defined a function $\wt f:A\to X_\alpha$ such that $j_\alpha\circ\wt f=f$. It remains to show that $\wt f$ is continuous. Indeed, we know $j_\alpha$ is an inclusion (\autoref{colimit_legs_of_lambda_sequence_of_inclusions_are_inclusions}), so given $U\sseq X_\alpha$ open, there exists $V\sseq\colim X$ open with $j_\alpha^{-1}(V)=U$, in which case
  \[\wt f^{-1}(U)=\wt f^{-1}(j_\alpha^{-1}(V))=(j_\alpha\circ\wt f)^{-1}(V)=f^{-1}(V),\]
  which is open as $f$ is continuous. Hence, $\wt f$ is continuous, as desired.

  Now, suppose for the sake of a contradiction that for all $\alpha<\lambda$, $f(A)\not\sseq j_\alpha(X_\alpha)$. Thus we may construct a \textbf{strictly increasing} sequence $\{\alpha_n\}_{n=0}^\infty\sseq\lambda$ such that for $n>0$, there exists $x_n\in j_{\alpha_n}(X_{\alpha_n})\setminus j_{\alpha_{n-1}}(X_{\alpha_{n-1}})$ with $x_n\in f(A)$. Thus for each $n>0$, there exists $y_n\in X_{\alpha_n}$ such that $j_{\alpha_n}(y_n)=x_n$. Note in particular that given $0\leq m<n$, $y_n$ is not in the image of $\iota_{\alpha_m,\alpha_n}$. Suppose for the sake of a contradiction that $y_n=\iota_{\alpha_m,\alpha_n}(z)$ for some $z\in X_{\alpha_m}$ and $0\leq m<n$. Then we know $j_{\alpha_m}(z)=j_{\alpha_n}(\iota_{\alpha_m,\alpha_n}(z))=j_{\alpha_n}(y_n)=x_n$, and
  \[x_n\in j_{\alpha_n}(X_{\alpha_n})\setminus j_{\alpha_{n-1}}(X_{\alpha_{n-1}})\spseq j_{\alpha_n}(X_{\alpha_n})\setminus j_{\alpha_{n-1}}(\iota_{\alpha_m,\alpha_{n-1}}(X_{\alpha_m}))=j_{\alpha_n}(X_{\alpha_n})\setminus j_{\alpha_{m}}(X_{\alpha_m}).\]
  Hence we reach a contradiction, as $j_{\alpha_m}(z)=x_m$ but $x_n$ is not in the image of $j_{\alpha_m}$. Let $\mu:=\sup_{n=1}^\infty\alpha_n$. Clearly $\mu\leq\lambda$; if $\mu=\lambda$, define $X_\mu:=\colim X$, $j_\mu:=\id_{X_\mu}$, and for $\alpha<\lambda$ define $\iota_{\alpha,\mu}:=j_\alpha$. Let $K:=\{\iota_{\alpha_n,\mu}(y_n)\}_{n=1}^\infty\sseq X_\mu$. We claim every subset of $K$ is closed in $X_\mu$. Since $X$ is colimit preserving and $\mu=\sup_{n=1}^\infty\alpha_n$, the topology on $X_\mu$ is the final topology induced by the maps $\iota_{\alpha_n,\mu}:X_{\alpha_n}\to X_\mu$ for $n=1,2,\ldots$. Thus, given a subset $C\sseq K$, in order to show that $C$ is closed in $X_\mu$, it is sufficient (and necessary) for $\iota_{\alpha_n,\mu}^{-1}(C)$ to be closed in $X_{\alpha_n}$ for $n=1,2,\ldots$. Let $n>0$. Given $y\in\iota_{\alpha_n,\mu}^{-1}(C)$, then $\iota_{\alpha_n,\mu}(y)\in C\sseq K$, so that in particular $\iota_{\alpha_n,\mu}(y)=\iota_{\alpha_m,\mu}(y_m)$ for some $m=1,2,\ldots$. We claim $m\leq n$. Suppose for the sake of a contradiction that $m>n$, then we would have 
  \[\iota_{\alpha_m,\mu}(y_m)=\iota_{\alpha_n,\mu}(y)=\iota_{\alpha_m,\mu}(\iota_{\alpha_n,\alpha_m}(y)),\]
  and $\iota_{\alpha_m,\mu}$ is injective (by either \autoref{lambda_sequence_of_inclusions_makes_everything_an_inclusion} if $\mu<\lambda$ or by \autoref{colimit_legs_of_lambda_sequence_of_inclusions_are_inclusions} if $\mu=\lambda$, in which case recall we defined $\iota_{\alpha_m,\mu}=j_{\alpha_m}$), thus $y_m=\iota_{\alpha_n,\alpha_m}(y)$, meaning $y_m$ is in the image of $\iota_{\alpha_n,\alpha_m}$ for $m>n$, a contradiction, as we showed earlier this is impossible. Thus it must have been true that $m\leq n$ in the first place, so 
  \[\iota_{\alpha_n,\mu}(y)\in\{\iota_{\alpha_m,\mu}(y_m)\}_{m=1}^n\implies y\in\iota_{\alpha_n,\mu}^{-1}(\{\iota_{\alpha_m,\mu}(y_m)\}_{m=1}^n).\]
  We further claim $\iota_{\alpha_n,\mu}^{-1}(\{\iota_{\alpha_m,\mu}(y_m)\}_{m=1}^n)=\{\iota_{\alpha_m,\alpha_n}(y_m)\}_{m=1}^n$. To see the inclusion $\sseq$, suppose $z\in X_{\alpha_n}$ with $\iota_{\alpha_n,\mu}(z)=\iota_{\alpha_m,\mu}(y_m)$ for some $m\leq n$. Then $\iota_{\alpha_n,\mu}(z)=\iota_{\alpha_n,\mu}(\iota_{\alpha_m,\alpha_n}(y_m))$ and $\iota_{\alpha_n,\mu}$ is injective (\autoref{lambda_sequence_of_inclusions_makes_everything_an_inclusion} if $\mu<\lambda$ and \autoref{colimit_legs_of_lambda_sequence_of_inclusions_are_inclusions} if $\mu=\lambda$), so $z=\iota_{\alpha_m,\alpha_n}(y_m)$, as desired. To see the opposite inclusion, given $m\leq n$, we have $\iota_{\alpha_n,\mu}(\iota_{\alpha_m,\alpha_n}(y_m))=\iota_{\alpha_m,\mu}(y_m)$, so $\iota_{\alpha_m,\alpha_n}(y_m)\in \iota_{\alpha_n,\mu}^{-1}(\{\iota_{\alpha_m,\mu}(y_m)\}_{m=1}^n)$, as desired. Thus, we have shown $y\in\{\iota_{\alpha_m,\alpha_n}(y_m)\}_{m=1}^n$. Recall our choice of $y\in\iota_{\alpha_n,\mu}^{-1}(C)$ was arbitrary, so $\iota_{\alpha_n,\mu}^{-1}(C)$ is contained in $\{\iota_{\alpha_m,\alpha_n}(y_m)\}_{m=1}^n$. Thus, because $\{\iota_{\alpha_m,\alpha_n}(y_m)\}_{m=1}^n$ is finite, in order to show $\iota_{\alpha_n,\mu}^{-1}(C)$ is closed in $X_{\alpha_n}$, it suffices to show that $\iota_{\alpha_m,\alpha_n}(y_m)$ is a closed point in $X_{\alpha_n}$ for $m=1,\ldots,n$. As we have shown above, $y_m$ is not in the image of $\iota_{\alpha_0,\alpha_m}$ for any $m\geq1$, and $\iota_{\alpha_0,\alpha_m}$ is a closed $T_1$ inclusion (\autoref{lambda_sequence_of_inclusions_makes_everything_an_inclusion}), so $y_m$ is a closed point of $X_{\alpha_m}$ for $m=1,\ldots,n$. Then since $\iota_{\alpha_m,\alpha_n}$ is closed (again by \autoref{lambda_sequence_of_inclusions_makes_everything_an_inclusion}), $\iota_{\alpha_m,\alpha_n}(y_m)$ is closed in $X_{\alpha_n}$ for $m=1,\ldots,n$, precisely the desired result.

  Now, we have shown that every subset of $K$ is closed in $X_{\mu}$. Then $j_\mu:X_\mu\to\colim X$ is a closed and injective (this follows by \autoref{colimit_legs_of_lambda_sequence_of_inclusions_are_inclusions} if $\mu<\lambda$, and if $\mu=\lambda$, $X_\mu=\colim X$, in which case $j_\mu$ is the identity), so every subset of $S:=j_\mu(K)$ is closed in $\colim X$. Note that
  \[S=\{j_\mu(\iota_{\alpha_n,\mu}(y_n))\}_{n=1}^\infty=\{j_{\alpha_n}(y_n)\}_{n=1}^\infty=\{x_n\}_{n=1}^\infty\sseq f(A),\]
  Then for $n=1,2,\ldots$, define $U_n:=f(A)\setminus (S\setminus\{x_n\})$. Each $U_n$ is open in $f(A)$ (as $S\setminus\{x_n\}$ is a subset of $S$ and is therefore closed in $\colim X$, thus in $f(A)$), and the collection $\{U_n\}_{n=1}^\infty$ forms an infinite open cover of $f(A)$. Finally, this open cover has no finite subcover, as $U_n$ is the only element of the cover containing $x_n$ for $n=1,2,\ldots$. Hence we reach a contradiction, as $f$ is continuous and $A$ is compact, so $f(A)$ is compact, but we have found an infinite open cover of $f(A)$ which has no finite subcover. Thus, there must have existed soem $\alpha<\lambda$ with $f(A)\sseq j_\alpha(X_\alpha)$ in the first place, in which case, as we have shown, this implies $f$ factors through $X_\alpha$ via a continuous map, as desired.
\end{proof}

\begin{proposition}[Hovey 2.4.5 \& 2.4.6]\label{2.4.5-6}
  The class of injective maps (resp.\ inclusions, closed $T_1$ inclusions) in $\Top$ is saturated.
\end{proposition}
\begin{proof}
  We know these three classes are closed under transfinite compositions (\autoref{inclusions_closed_under_transfinite_composition}), so it suffices to show these classes are closed under pushouts and retracts. In what follows, fix a pushout diagram and a retract diagram of the following form: 
  \[\begin{tikzcd}
    A & C && A & B & A \\
    B & D && C & D & C
    \arrow["i"', from=1-1, to=2-1]
    \arrow["g", from=2-1, to=2-2]
    \arrow["f", from=1-1, to=1-2]
    \arrow["j", from=1-2, to=2-2]
    \arrow["\ulcorner"{anchor=center, pos=0.125, rotate=180}, draw=none, from=2-2, to=1-1]
    \arrow["f"', from=1-4, to=1-5]
    \arrow["g"', from=1-5, to=1-6]
    \arrow["j"', from=1-4, to=2-4]
    \arrow["h", from=2-4, to=2-5]
    \arrow["k", from=2-5, to=2-6]
    \arrow["i", from=1-5, to=2-5]
    \arrow["j", from=1-6, to=2-6]
    \arrow[curve={height=-18pt}, Rightarrow, no head, from=1-4, to=1-6]
    \arrow[curve={height=18pt}, Rightarrow, no head, from=2-4, to=2-6]
  \end{tikzcd}\]
  \begin{enumerate}[label=(\roman*),listparindent=\parindent,parsep=0pt]
    \item First, consider the pushout diagram, and suppose $i$ is injective. We wish to show $j$ is injective. Suppose for the sake of a contradiction there existed distinct points $c_1,c_2\in C$ such that $j(c_1)=j(c_2)$. Define $ h:C\to\{1,2,3\}$ mapping $c_1\mapsto 1$, $c_2\mapsto 2$, and every other point in $C$ maps to $3$. Define $k:B\to\{1,2,3\}$ to map every point in $i(f^{-1}(c_1))$ to $1$, every point in $i(f^{-1}(c_2))$ to $2$, and every other point to $3$. We may give $\{1,2,3\}$ the indiscrete topology so that $ h$ and $k$ are continuous. Then clearly $ h\circ f= k\circ i$, so there exists a map $\ell:D\to\{1,2,3\}$ such that $\ell\circ j= h$ and $\ell\circ g= k$. But we reach a contradiction, as $\ell(j(c_1))=\ell(j(c_2))$, but $ h(c_1)\neq h(c_2)$.
    
    Now, consider the retract diagram, and suppose $i$ is injective. We wish to show $j$ is injective. Suppose $a_1,a_2\in A$ such that $j(a_1)=j(a_2)$. Then $h(j(a_1))=h(j(a_2))$, and $h\circ j=i\circ f$, so $i(f(a_1))=i(f(a_2))$. Note that since $g\circ f=\id_A$, necessarily $f$ is injective, and we are assuming $i$ is injective, so $i(f(a_1))=i(f(a_2))\implies a_1=a_2$.
    \item First, consider the pushout diagram, and suppose $i$ is an inclusion. We wish to show $j$ is an inclusion. We know $j$ is injective by (i). It remains to show that given $U\sseq C$ open, there exists $V\sseq D$ open with $j^{-1}(V)=U$. Given $U\sseq C$ open, $f^{-1}(U)$ is open in $A$ as $f$ is continuous, and $i$ is an inclusion, so there exists $W\sseq B$ open with $i^{-1}(W)=f^{-1}(U)$. Now let $S=\{0,1\}$ be the Sierpinski space with open sets $\{\emptyset,\{1\},\{0,1\}\}$. Define $ h:C\to S$ to map every point of $U$ to $1$, and every other point to $0$. Define $ k:B\to S$ to map every point of $W$ to $1$, and every other point to $0$. Clearly $ h$ and $ k$ are continuous. We claim $ h\circ f= k\circ i$. Indeed, let $a\in A$. If $a\in f^{-1}(U)$, then $ h(f(a))=1$, as $f(a)\in U$, while $ k(i(a))=1$, as $a\in U=i^{-1}(W)$, so $i(a)\in W$. Conversely, if $a\notin f^{-1}(U)$, then $ h(f(a))=0$ as $f(a)\notin U$, while $ k(i(a))=0$, as $a\notin i^{-1}(W)$ meaning $i(a)\notin W$. Hence, there exists a (unique) continuous map $\ell:D\to S$ with $\ell\circ j=h$ and $\ell\circ g=k$. Define $V:=\ell^{-1}(1)$, which is open in $D$ as $\{1\}$ is open in $S$. Then finally, we claim $j^{-1}(V)=U$. Indeed,
    \[j^{-1}(V)=j^{-1}(\ell^{-1}(1))=(\ell\circ j)^{-1}(1)=h^{-1}(1)=U.\]
    Thus $j$ is an inclusion, as desired.
    
    Now, consider the retract diagram, and suppose $i$ is an inclusion. We wish to show $j$ is an inclusion. We know $j$ is injective by (i). It remains to show that given $U\sseq A$ open, there exists $V\sseq C$ open with $j^{-1}(V)=U$. Then since $g$ is continuous, $g^{-1}(U)$ is open in $B$, and $i$ is an inclusion, so there exists $W\sseq D$ open with $i^{-1}(W)=g^{-1}(U)$. Then 
    \[j^{-1}(h^{-1}(W))=(h\circ j)^{-1}(W)=(i\circ f)^{-1}(W)= f^{-1}(i^{-1}(W))= f^{-1}(g^{-1}(U))=(g\circ f)^{-1}(U)=U,\]
    and $h^{-1}(W)$ is open as $h$ is continuous and $W$ is open, so we are done, as if we set $V:=h^{-1}(W)$, then we have shown $V$ is open in $C$ and $j^{-1}(V)=U$, as desired.

    \item First, consider the pushout diagram, and suppose $i$ is a closed $T_1$ inclusion. We wish to show $j$ is a closed $T_1$ inclusion. We know $j$ is an inclusion by (ii). It remains to show $j$ is closed, and every point of $D\setminus j(C)$ is closed in $D$. First, we show closedness. Let $V\sseq C$ be a closed set, we want to show $j(C)$ is closed in $D$. By definition of the colimit topology on $D$ (which is the final topology on $D$ induced by $j$ and $g$ by the discussion at the beginning of this chapter), in order to show $j(V)$ is closed in $D$ it suffices to show that $j^{-1}(j(V))$ is closed in $C$ and $g^{-1}(j(V))$ is closed in $B$. Since $j$ is injective by (i), $j^{-1}(j(V))=V$, which we have defined to be closed in $A$. Now, to show $g^{-1}(j(V))$ is closed, since $i$ is a closed map and $f$ is continuous, it suffices to show $i(f^{-1}(V))=g^{-1}(j(V))$. First of all, let $b\in i(f^{-1}(V))$, then $b=i(a)$ for some $a\in A$ with $f(a)\in V$. Then $g(b)=g(i(a))=j(f(a))\in j(V)$, so $b\in g^{-1}(j(V))$. Conversely, let $b\in g^{-1}(j(V))$, so $g(b)=j(c)$ for some $c\in V$. Then by the explicit description of the colimit in $\Set$ and the fact that the forgetful functor $\Top\to\Set$ preserves colimits, there exists $a\in A$ such that $f(a)=c$ and $i(a)=b$. Then in particular, $f(a)=c\in V$, so $a\in f^{-1}(V)$, so $b\in i(f^{-1}(V))$ as desired. Hence, $j$ is indeed a closed map. It remains to show that for all $d\in D\setminus j(C)$, $d$ is a closed point in $D$. Given $d\in D\setminus j(C)$, by the explicit characterization of the colimit topology, it suffices to show that $j^{-1}(d)$ and $g^{-1}(d)$ are closed in $C$ and $B$, respectively. First of all, $j^{-1}(d)=\emptyset$, which is closed. Now, since $d$ is not in the image of $j$, $d$ is not in the image of $g\circ i=j\circ f$, so $g^{-1}(d)\sseq B\setminus i(A)$. Thus since $i$ is a closed $T_1$ inclusion, in order to show $g^{-1}(d)$ is closed, it suffices to show that $g^{-1}(d)$ is a singleton. Suppose for the sake of a contradiction there existed distinct points $x,y\in B$ with $g(x)=g(y)=d$. Define $h:C\to\{1,2,3\}$ to map every point to $3$. Define $k:B\to\{1,2,3\}$ to map $x\mapsto 1$, $y\mapsto 2$, and every other point in $B$ maps to $3$. Endue $\{1,2,3\}$ with the indiscrete topology so that $h$ and $k$ are continuous. Note $h\circ f=k\circ i$: given $a\in A$, since $x,y\in g^{-1}(d)\sseq B\setminus i(A)$, $i(a)$ does not equal $x$ or $y$, thus $k(i(a))=3=h(f(a))$. Hence by the definition of the colimit, there exists a map $\ell:D\to X$ such that $\ell\circ g=k$ and $\ell\circ j=h$. Then we reach a contradiction, as $k(x)\neq k(y)$ but $\ell(g(x))=\ell(g(y))=\ell(d)$. Hence, $g^{-1}(d)\notin i(A)$ must have been a singleton in the first place, meaning $g^{-1}(d)$ is closed in $B$ as desired.
    
    Now, consider the retract diagram, and suppose $i$ is a closed $T_1$ inclusion. We wish to show $j$ is a closed $T_1$ inclusion. We know $j$ is an inclusion by (ii). It remains to show $j$ is closed, and every point of $C\setminus j(A)$ is closed in $C$. First, we show closedness. Let $V\sseq A$ be closed. First, we claim $j(V)=h^{-1}(i(g^{-1}(V)))$. Given $c\in j(V)$, $c=j(a)$ for some $a\in V$, in which we have $h(c)=h(j(a))=i(f(a))$, and $g(f(a))=a\in V$, so $f(a)\in g^{-1}(V)$, meaning $h(c)\in i(g^{-1}(V))$, so $c\in h^{-1}(i(g^{-1}(V)))$. Conversely, given $c\in h^{-1}(i(g^{-1}(V)))$, so $h(c)=i(b)$ for some $b\in B$ with $g(b)\in V$. Then
    \[c=k(h(c))=k(i(b))=j(g(b))\in j(V),\]
    as desired. Thus, we have shown $j(V)=h^{-1}(i(g^{-1}(V)))$.Note that since $V\sseq A$ is closed and $g$ is continuous, $g^{-1}(V)$ is closed in $B$. Since $i$ is a closed map, $i(g^{-1}(V))$ is closed in $D$. Finally, since $h$ is continuous, $h^{-1}(i(g^{-1}(V)))=j(V)$ is closed in $C$, as desired. It remains to show that for all $c\in C\setminus j(A)$ that $c$ is a closed point in $C$. Given $c\in C\setminus j(A)$, note that $h(c)\notin i(B)$, as if $h(c)=i(b)$ for some $b\in B$, then we would have $c=k(h(c))=k(i(b))=j(g(b))$, yet we chose $c$ not in the image of $j$. Hence, since $i$ is a closed $T_1$ inclusion, and $h(c)$ is not in the image of $i$, $h(c)$ is a closed point in $D$. Then note that since $k\circ h=\id_C$, $h$ is injective, so $c=h^{-1}(h(c))$ is a closed point, as it is the preimage of the closed set $\{h(c)\}$ along the continuous map $h$.\qedhere
  \end{enumerate}
\end{proof}

\begin{lemma}[Hovey 2.4.8]\label{2.4.8}
  $\cW\cap\cT$ is closed under transfinite compositions, where $\cT$ denotes the class of closed $T_1$ inclusions.
\end{lemma}
\begin{proof}
  Let $\lambda$ be an ordinal, and let $X:\lambda\to\Top$ be a $\lambda$-sequence such that for all $\beta+1<\lambda$, the map $X_{\beta}\to X_{\beta+1}$ belongs to $\cW\cap\cT$. Let $j:X\to\ul{X_\lambda}$ be a colimit cone for $X$. By \autoref{inclusions_closed_under_transfinite_composition}, we know that $j_0:X\to X_\lambda$ is a closed $T_1$ inclusion, so it remains to show that $\pi_n(j_0,x_0):\pi_n(X_0,x_0)\to\pi_n(X_\lambda,j_0(x_0))$ is an isomorphism for all $n\geq0$ and $x_0\in X_0$.

  First we show surjectivity. Suppose we are given $x_0\in X_0$ and a continuous map $f:(S^n,\ast)\to(X_\lambda,j_0(x_0))$. Since $S^n$ is compact, by \autoref{}
\end{proof}

\begin{proposition}\label{domains_of_I'/J_small_rel_I'-cell/J-cell}
  The domains of $I'$ (resp.\ $J$) are small relative to $I'\cell$.
\end{proposition}
\begin{proof}
  By \autoref{2.4.1}, every space is small relative to the inclusions, and in particular every space is small relative to the class $\cT$ of closed $T_1$ inclusions. Hence, it suffices to show that $J\cell,I'\cell\sseq\cT$. We showed above in \autoref{2.4.5-6} that $\cT$ is closed under transfinite composition and pushouts, and clearly every map in $I'$ and $J$ is a closed $T_1$ inclusion, so the desired result follows.
\end{proof}

\begin{lemma}[Hovey Lemma 2.4.4]\label{2.4.4}
  The weak equivalences in $\Top$ are closed under retracts and satisfy 2-of-3 axiom (so that in particular the weak equivalences form a subcategory, as clearly identities are weak equivalences).
\end{lemma}
\begin{proof}
  First we show that weak equivalences satisfy 2-of-3. Let $f:X\to Y$ and $g:Y\to Z$ be continuous functions of topological spaces. 
  
  First of all, suppose $f$ and $g$ are both weak equivalences. Then by functoriality of $\pi_n$, since $\pi_n(f,x)$ and $\pi_n(g,f(x))$ are isomorphisms for all $x\in X$, $\pi_n(g\circ f,x)=\pi_n(g,f(x))\circ\pi_n(f,x)$ is likewise an isomorphism for all $x\in X$, so that $g\circ f$ is a weak equivalence.

  Now, suppose that $g\circ f$ and $g$ are weak equivalences. Pick a point $x\in X$. We wish to show that $\pi_n(f,x):\pi_n(X,x)\to\pi_n(Y,f(x))$ is an isomorphism for all $n\geq0$. We know that $\pi_n(g\circ f,x)$ is an isomorphism, and $\pi_n(g,f(x))$ is an isomorphism, say with inverse, $\varphi$, so that
  \[\varphi\circ\pi_n(g\circ f,x)=\varphi\circ\pi_n(g,f(x))\circ\pi_n(f,x)=\pi_n(f,x)\]
  is an isomorphism, as it is a composition of isomorphisms.

  Now, suppose that $g\circ f$ and $f$ are weak equivalences. Pick a point $y\in Y$. Since $\pi_0(f)$ is an isomorphism, there exists a point $x\in X$ such that $f(x)$ belongs to the path component containing $y$, so that there exists some $\alpha:I\to Y$ with $\alpha(0)=f(x)$ and $\alpha(1)=f(y)$. Then consider the following diagram
  % https://q.uiver.app/?q=WzAsNCxbMCwwLCJcXHBpX24oWSx5KSJdLFsxLDAsIlxccGlfbihaLGcoeSkpIl0sWzAsMSwiXFxwaV9uKFksZih4KSkiXSxbMSwxLCJcXHBpX24oWixnKGYoeCkpKSJdLFswLDEsIlxccGlfbihnLHkpIl0sWzAsMl0sWzIsMywiXFxwaV9uKGcsZih4KSkiLDJdLFsxLDNdXQ==
  \[\begin{tikzcd}
    {\pi_n(Y,y)} & {\pi_n(Z,g(y))} \\
    {\pi_n(Y,f(x))} & {\pi_n(Z,g(f(x)))}
    \arrow["{\pi_n(g,y)}", from=1-1, to=1-2]
    \arrow[from=1-1, to=2-1]
    \arrow["{\pi_n(g,f(x))}", from=2-1, to=2-2]
    \arrow[from=1-2, to=2-2]
  \end{tikzcd}\]
  where the left arrow is the isomorphism given by conjugation by the path $\alpha$, and the right arrow is the isomorphism given by conjugation by the path $g\circ\alpha$. It is tedious yet straightforward to verify that the diagram commutes.
  Furthermore, we know that $\pi_n(f,x)$ and $\pi_n(g\circ f,x)=\pi_n(g,f(x))\circ\pi_n(f,x)$ are isomorphisms for all $n$, so that if we denote the inverse of $\pi_n(f,x)$ by $\varphi$, then
  \[\pi_n(g\circ f,x)\circ\varphi=\pi_n(g,f(x))\circ\pi_n(f,x)\circ\varphi=\pi_n(g,f(x))\]
  is an isomorphism, as it is given as a composition of isomorphisms. Hence, the top arrow must likewise be an isomorphism, precisely the desired result.

  The fact that weak equivalences in $\Top$ are closed under retracts is entirely straightforward and follows from the fact that the functors $\pi_n$ preserve retract diagrams and that the class of isomorphisms in any category is closed under retracts.
\end{proof}

\begin{proposition}[Hovey 2.4.9]\label{2.4.9}
  ${\p({J}\p)}\sseq\cW\cap {\p( I'\p )}$.
\end{proposition}
\begin{proof}
  First, in order to show ${\p(J\p)}\sseq{\p( I'\p )}$, It suffices to show that $J\sseq I'\cell$, as by \autoref{2.1.10} we would have $J\sseq{\p( I'\p )}$, and
  \[J\sseq {\p( I'\p )}\implies{\p(J\p)}\sseq{\p((\p( I'\p ))\p)}={\p( I'\p )},\]
  where the implication and equality both follow from \autoref{useful_LP_properties} which gives that
  \[A\sseq B\implies {\p(A\p)}\sseq{\p(B\p)}\quad\text{ and }\quad(\p({A}\p))\p=A\p.\]
  Now, to show $J\sseq I'\cell$, first consider the composition $j_n:D^n\into S^n\into D^{n+1}$, where the first map is the pushout
  \[\begin{tikzcd}
    {S^{n-1}} & {D^n} \\
    {D^n} & {S^n}
    \arrow[hook, from=1-1, to=1-2]
    \arrow[hook, from=1-1, to=2-1]
    \arrow[from=2-1, to=2-2]
    \arrow[from=1-2, to=2-2]
    \arrow["\ulcorner"{anchor=center, pos=0.125, rotate=180}, draw=none, from=2-2, to=1-1]
  \end{tikzcd}\]
  obtained by gluing two copies of $D^n$ along their boundary, and the second map map is simply the inclusion $S^n\into D^{n+1}$, which can be written as the pushout
  \[\begin{tikzcd}
    {S^n} & {S^n} \\
    {D^{n+1}} & {D^{n+1}}
    \arrow[Rightarrow, no head, from=1-1, to=1-2]
    \arrow[hook, from=1-1, to=2-1]
    \arrow[Rightarrow, no head, from=2-1, to=2-2]
    \arrow[hook, from=1-2, to=2-2]
    \arrow["\ulcorner"{anchor=center, pos=0.125, rotate=180}, draw=none, from=2-2, to=1-1]
  \end{tikzcd}\]
  It can be seen that $j_n$ includes $D^n$ as a hemisphere of $S^n=\partial D^{n+1}\sseq D^{n+1}$. Note that $D^{n}\times I$ is homeomorphic to $D^{n+1}$ (``smooth out'' the sharp edges of the cylinder) via some homeomorphism $h_n:D^{n+1}\to D^n\times I$, and in particular, we may define $h_n$ so that $h_n(j_n(D^n))= D^n\times\{0\}\sseq D^n\times I$ by squashing the hemisphere $j_n(D^n)$ to be one of the faces of the cylinder $D^n\times I$, in which case $h_n\circ j_n:D^n\to D^n\times I$ is precisely the inclusion $D^n\into D^n\times I$ sending $x\mapsto (x,0)$, and since $j_n\in I'\cell$, $h_n\circ j_n\in I'\cell$ by \autoref{I-cell_closed_under_composition_with_isomorphisms}.

  Now, we claim that $\p(J\p)\sseq\cW$. First note that by \autoref{2.1.15} and \autoref{domains_of_I'/J_small_rel_I'-cell/J-cell}, every map in $\p(J\p)$ is a retract of an element of $J\cell$. Furthermore, we know that $\cW$ is closed under retracts (\autoref{2.4.4}), so that it suffices to show that $J\cell\sseq\cW$. We claim it suffices to show that pushouts of maps in $J$ are weak equivalences. Supposing we had shown this, we would have that pushouts of maps in $J$ are weak equivalences and $T_1$ inclusions, as $J\sseq\cT$ and $\cT$ is saturated by \autoref{2.4.5-6}. Then by \autoref{2.4.8}, we would have that $J\cell\sseq\cW\cap\cT$, precisely the desired result.
  
  Now, let $\cS$ be the class of \textit{inclusions of a deformation retract}, i.e., those \textbf{injective} maps $i:A\to B$ such that there exists a homotopy $H:B\times I\to B$ with $H(i(a),t)=i(a)$ for all $a\in A$, $H(b,0)=b$ for all $b\in B$, and $H(b,1)=i(r(b))$ for all $b\in B$ for some map $r:B\to A$\footnote{Hovey has a typo here, namely, he does not specify that $i$ must be injective. Without this specification, his assertion fails. For example, take $A=\bR^2$, $B=\bR$, $i(x,y)=x$, $H(b,t)=b$, and $r(b)=(b,0)$. Then $i$ is an inclusion of a deformation retract according to Hovey's ``definition,'' but $i$ is not injective and $r$ is not a retract.}. We will show the following:
  \begin{enumerate}[listparindent=\parindent,parsep=0pt]
    \item $\cS\sseq\cW$.
    
    It suffices to show that if $i:A\to B$ belongs to $\cS$, then $i$ is a homotopy equivalence. Indeed, given $i:A\to B$, let $H:B\times I\to B$ and $r:B\to A$ be a homotopy and retract satisfying the conditions above. Then in particular, $H$ is a homotopy between $\id_B$ (at time $t=0$) and $i\circ r$ (at time $t=1$). It remains to show that $r\circ i=\id_A$. First of all, note that since $H(b,1)=i(r(b))$ for all $b\in B$, we have $H(i(a),1)=i(r(i(a)))$. Yet, we also know that $H(i(a),t)=i(a)$ for all $t\in I$, so $i(r(i(a)))=i(a)$, and $i$ is injective so $r(i(a))=a$.

    \item $J\sseq\cS$.
    
    For $n\geq0$, let $j_n:D^n\into D^n\times I$ denote the inclusion of $D^n$ as the subset $D^n\times\{0\}$. Define a deformation retract $H:D^n\times I\times I\to D^n\times I$ by $(x,s,t)\mapsto(x,s(1-t))$. Then indeed we have $H(j_n(x),t)=H(x,0,t)=(x,0)=j_n(x)$ for all $x\in D^n$, $H(x,t,0)=(x,t(1-0))=(x,t)$ for all $(x,t)\in D^n\times I$, and $H(x,t,1)=(x,t(1-1))=(x,0)=j_n(r(x))$ for all $(x,t)\in D^n\times I$, where $r:D^n\times I\to D^n$ is the projection onto time zero sending $(x,t)\mapsto(x,0)$. Finally, $j_n$ is clearly injective. Thus, indeed $J\sseq\cS$.

    \item $\cS$ is closed under pushouts.
    
    Suppose we are given a pushout diagram
    \[\begin{tikzcd}
      A & C \\
      B & D
      \arrow["i"', from=1-1, to=2-1]
      \arrow["g"', from=2-1, to=2-2]
      \arrow["f", from=1-1, to=1-2]
      \arrow["j", from=1-2, to=2-2]
      \arrow["\ulcorner"{anchor=center, pos=0.125, rotate=180}, draw=none, from=2-2, to=1-1]
    \end{tikzcd}\]
    where $i\in\cS$. Then we wish to show $j$ in $\cS$. First, we know $j$ is injective by \autoref{2.4.5-6}. Now, we look to construct $H$ and $r$. Let $K:B\times I\to B$ and $r':B\to A$ be maps satisfying the conditions for $i$ to be an inclusion of a deformation retract.

    We wish to define a homotopy $H:D\times I\to D$. Then $I$ is a locally compact Hausdorff space (in particular, it is compact and Hausdorff), so that the functor $-\times I:\Top\to\Top$ preserves colimits (\autoref{locally_compact_hausdorff_spaces_are_exponentiable}), meaning the following is a pushout diagram:
    % https://q.uiver.app/?q=WzAsNCxbMCwwLCJBXFx0aW1lcyBJICJdLFsxLDAsIkNcXHRpbWVzIEkiXSxbMSwxLCJEXFx0aW1lcyBJIl0sWzAsMSwiQlxcdGltZXMgSSJdLFswLDEsImZcXHRpbWVzXFxpZF9JIl0sWzEsMiwialxcdGltZXNcXGlkX0kiXSxbMiwwLCIiLDAseyJzdHlsZSI6eyJuYW1lIjoiY29ybmVyLWludmVyc2UifX1dLFswLDMsImlcXHRpbWVzXFxpZF9JIiwyXSxbMywyLCJnXFx0aW1lc1xcaWRfSSIsMl1d
    \[\begin{tikzcd}
      {A\times I } & {C\times I} \\
      {B\times I} & {D\times I}
      \arrow["{f\times\id_I}", from=1-1, to=1-2]
      \arrow["{j\times\id_I}", from=1-2, to=2-2]
      \arrow["\ulcorner"{anchor=center, pos=0.125, rotate=180}, draw=none, from=2-2, to=1-1]
      \arrow["{i\times\id_I}"', from=1-1, to=2-1]
      \arrow["{g\times\id_I}"', from=2-1, to=2-2]
    \end{tikzcd}\]
    Then by the universal property of the pushout, there is a map $H:D\times I\to D$ (the dashed line) such that the following diagram commutes
    \[\begin{tikzcd}
      {A\times I } & {C\times I} \\
      {B\times I} & {D\times I} & C \\
      & B & D
      \arrow["{f\times\id_I}", from=1-1, to=1-2]
      \arrow["{j\times\id_I}", from=1-2, to=2-2]
      \arrow["\ulcorner"{anchor=center, pos=0.125, rotate=180}, draw=none, from=2-2, to=1-1]
      \arrow["{i\times\id_I}"', from=1-1, to=2-1]
      \arrow["{g\times\id_I}"', from=2-1, to=2-2]
      \arrow["{\pi_1}", curve={height=-12pt}, two heads, from=1-2, to=2-3]
      \arrow["K"', curve={height=12pt}, from=2-1, to=3-2]
      \arrow["g"', from=3-2, to=3-3]
      \arrow["j", from=2-3, to=3-3]
      \arrow["H", dashed, from=2-2, to=3-3]
    \end{tikzcd}\]
    Now, note $r'\circ i=\id_A$. Indeed, given $a\in A$, we have $i(r'(i(a)))=K(i(a),t)=i(a)$ and $i$ is injective, so that $r'(i(a))=a$, as desired. Hence, there exists a unique map $r:D\to C$ (the dashed line) such that the following diagram commutes:
    % https://q.uiver.app/?q=WzAsNixbMCwwLCJBIl0sWzEsMCwiQyJdLFsxLDEsIkQiXSxbMCwxLCJCIl0sWzIsMiwiQyJdLFsxLDIsIkEiXSxbMCwxLCJmIl0sWzEsMiwiaiJdLFsyLDAsIiIsMCx7InN0eWxlIjp7Im5hbWUiOiJjb3JuZXItaW52ZXJzZSJ9fV0sWzAsMywiaSIsMl0sWzMsMiwiZyIsMl0sWzMsNSwiciIsMix7ImN1cnZlIjoyfV0sWzUsNCwiZiIsMl0sWzIsNCwiIiwxLHsic3R5bGUiOnsiYm9keSI6eyJuYW1lIjoiZGFzaGVkIn19fV0sWzEsNCwiIiwyLHsiY3VydmUiOi0yLCJsZXZlbCI6Miwic3R5bGUiOnsiaGVhZCI6eyJuYW1lIjoibm9uZSJ9fX1dXQ==
    \[\begin{tikzcd}
      A & C \\
      B & D \\
      & A & C
      \arrow["f", from=1-1, to=1-2]
      \arrow["j", from=1-2, to=2-2]
      \arrow["\ulcorner"{anchor=center, pos=0.125, rotate=180}, draw=none, from=2-2, to=1-1]
      \arrow["i"', from=1-1, to=2-1]
      \arrow["g"', from=2-1, to=2-2]
      \arrow["r'"', curve={height=12pt}, from=2-1, to=3-2]
      \arrow["f"', from=3-2, to=3-3]
      \arrow["r", dashed, from=2-2, to=3-3]
      \arrow[curve={height=-20pt}, Rightarrow, no head, from=1-2, to=3-3]
    \end{tikzcd}\]
    Now we claim that our constructions $H$ and $r$ endue $j$ with the structure of an inclusion of a deformation retract, as desired. First $c\in C$, we wish to show $H(j(c),t)=j(c)$ for all $t$. Indeed, we have 
    \[H(j(c),t)=H(j\times\id_I(c,t))=j(\pi_1(c,t))=j(c).\]
    Given $d\in D$, we want to show $H(d,0)=d$. By the explicit description of the colimit in $\Top$, we know that every element of $D$ is in the image of either $j$ or $g$. If $d=j(c)$ for some $c$, then we have just shown $H(d,0)=H(j(c),0)=j(c)=d$, as desired. On the other hand, if $d=g(b)$ for some $b\in B$ we have 
    \[H(d,0)=H(g\times\id_I(b,0))=g(K(b,0))=g(b)=d.\] 
    Finally, we claim that $H(d,1)=j(r(d))$ for all $d\in D$. If $d=j(c)$ for some $c\in C$, then we have 
    \[H(d,1)=H(j(c),1)=j(c)=j(r(j(c)))=j(r(d)),\] 
    as desired. On the other hand, if $d=g(b)$ for some $b\in B$, then
    \[H(d,1)=H(g\times\id_I(b,1))=g(K(b,1))=g(i(r'(b)))=j(f(r'(b)))=j(r(g(b)))=j(r(d)).\qedhere\]
  \end{enumerate}
\end{proof}

\begin{proposition}[Hovey 2.4.10]\label{2.4.10}
  $I'\p\sseq\cW\cap J\p$
\end{proposition}
\begin{proof}
  First, by \autoref{2.4.9} we know $\p(J\p)\sseq\p(I'\p)$, and this implies $I'\p\sseq J\p$, as by \autoref{useful_LP_properties} we have
  \[\p(J\p)\sseq\p(I'\p)\implies J\p=(\p(J\p))\p\spseq(\p(I'\p))\p=I'\p.\]
  Thus, it suffices to show that $I'\p\sseq\cW$. Now, suppose $p:(X,x_0)\to (Y,p(x_0))$ is in $I'\p$. We wish to show that the map $\pi_n(p,x_0):\pi_n(X,x_0)\to\pi_n(Y,p(x_0))$ is an isomorphism for all $n$.
  
  First we show that $\pi_n(p,x_0)$ is surjective. Let $g:(S^n,\ast)\to(Y,p(x_0))$ be a map. Then we have the following commutative diagram
  \[\begin{tikzcd}
    \ast & X \\
    {S^n} & Y
    \arrow[from=1-1, to=2-1]
    \arrow["g", from=2-1, to=2-2]
    \arrow[from=1-1, to=1-2]
    \arrow["p", from=1-2, to=2-2]
  \end{tikzcd}\]
  where the top arrow picks out $x_0$. Note that the map $\ast\to S^n$ may be realized as a pushout of the diagram $D^n\leftarrow S^{n-1}\rightarrow\ast$, so that $\ast\to S^n$ belongs to $I'\cell$, and therefore $\p(I'\p)$ by \autoref{2.1.10}, and $p\in I'\p$, so $\ast\to S^n$ has the left lifting property against $p$. Thus, the above diagram has a lift $f:(S^n,\ast)\to (X,x_0)$ such that $p\circ f=g$, so that $\pi_n(p,x_0)([f])=[p\circ f]=[g]$, as desired.

  Finally, we show that $\pi_n(p,x_0)$ is injective. Suppose we have two maps $f,g:(S^n,\ast)\to(X,x_0)$ such that $p\circ f$ and $p\circ g$ represent the same element of $\pi_n(Y,p(x_0))$. Then there is a homotopy $H:S^n\times I\to Y$ such that for all $s\in S^n$ and $t\in I$, $H(s,0)=p(f(s))$, $H(s,1)=p(g(s))$, and $H(\ast,t)=p(x_0)$. By the universal property of the quotient, $H$ induces a map $\ol H:S^n\wedge I_+:=(S^n\times I)/(\ast\times I)$ sending the equivalence class $[s,t]\mapsto H(s,t)$. Hence, the following diagram commutes:
  \[\begin{tikzcd}
    {S^n\vee S^n} & X \\
    {S^n\wedge I_+} & Y
    \arrow[hook, from=1-1, to=2-1]
    \arrow["{\ol H}", from=2-1, to=2-2]
    \arrow["{f\vee g}", from=1-1, to=1-2]
    \arrow["p"', from=1-2, to=2-2]
  \end{tikzcd}\]
  where the left arrow is an element of $I'\cell$, as it may be obtained by attaching an $n+1$ cell to $S^n\vee S^n$ (when $n=0$, the attachning map is obvious; when $n>0$, the attaching map is the quotient map $S^n\onto S^n\vee S^n$ obtained by collapsing the equator). Thus, by similar reasoning to above there exists a lift $\ol K: S^n\wedge I_+\to X$. Then if we define $K$ to be the composition $S^n\times I\onto S^n\wedge I_+\xrightarrow{\ol K} X$, this gives us the desired homotopy between $f$ and $g$: given $s\in S^n$ and $t\in I$, we have $K(s,0)=\ol K([s,0])=f(s)$, $K(s,1)=\ol K([s,1])=g(s)$, and $K(\ast,t)=\ol K([\ast,t])=\ol K([\ast,0])=(\ast)=x_0$.
\end{proof}

In what follows, given continuous maps $p:X\to Y$ and $i:A\to B$, let $Q(i,p):X^B\to P(i,p):=X^A\times_{Y^A}Y^B$ denote the map obtained by the universal property of the fiber product (pullback) via the maps $i^\ast\circ p_\ast:X^B\to Y^A$ and $p_\ast\circ i^\ast:X^B\to Y^A$.

\begin{lemma}[Hovey 2.4.11]\label{2.4.11}
  Suppose $p:X\to Y$ is a map. Then $p\in I'\p$ if and only if the map $Q(i,p):X^B\to P(i,p):=X^A\times_{Y^A} Y^B$ is surjective for all maps $i:A\to B$ in $I'$. In particular, if $Q(i,p)\in\cW\cap J\p$ for all $i\in I'$, then $p\in I'\p$.
\end{lemma}
\begin{proof}
  \color{red}TODO.
\end{proof}

\begin{lemma}[Hovey 2.4.13]\label{2.4.13}
  Suppose $p:X\to Y$ belongs to $J\p$ and $i:S^{n-1}\into D^n$ belongs to $I'$. Then the map $Q(i,p)$ belongs to $J\p$.
\end{lemma}
\begin{proof}
  \color{red}TODO.
\end{proof}

\begin{corollary}[Hovey 2.4.14]\label{2.4.14}
  Every topological space is fibrant, i.e., given a space $X$, the unique map $X\to\ast$ is an element of $J\p$. In particular, the map $Y^{D^n}\to Y^{S^{n-1}}$ belongs to $J\p$ for all $n\geq0$.
\end{corollary}
\begin{proof}
  \color{red}TODO.
\end{proof}

\begin{lemma}[Hovey 2.4.15]\label{2.4.15}
  If $p:X\to Y$ is a weak equivalence (belongs to $\cW$), then $p^{D^n}:X^{D^n}\to Y^{D^n}$ is also a weak equivalence.
\end{lemma}
\begin{proof}
  \color{red}TODO.
\end{proof}

\begin{lemma}[Hovey 2.4.16]\label{2.4.16}
  Suppose $p:X\to Y$ belongs to $J\p$, and $x\in X$. Let $F:= p^{-1}(p(x))$, and $i:F\into X$ denote the inclusion. Then there is a long exact sequences
  % https://q.uiver.app/?q=WzAsOCxbMCwxLCJcXHBpX24oRix4KSJdLFsxLDEsIlxccGlfbihYLHgpIl0sWzIsMSwiXFxwaV9uKFkscCh4KSkiXSxbMCwyLCJcXHBpX3tuLTF9KEYseCkiXSxbMSwwLCJcXGNkb3RzIl0sWzIsMCwiXFxwaV97bisxfShZLHAoeCkpIl0sWzEsMiwiXFxjZG90cyJdLFsyLDIsIlxccGlfMChZLHAoeCkpIl0sWzAsMSwiXFxwaV9uKGkseCkiLDJdLFsxLDIsIlxccGlfbihwLHgpIl0sWzIsMywiZF9cXGFzdCJdLFs0LDVdLFs1LDAsImRfXFxhc3QiLDJdLFszLDYsIlxccGlfe24tMX0oaSx4KSIsMl0sWzYsNywiXFxwaV8wKHAseCkiLDJdXQ==
  \[\begin{tikzcd}
    & \cdots & {\pi_{n+1}(Y,p(x))} \\
    {\pi_n(F,x)} & {\pi_n(X,x)} & {\pi_n(Y,p(x))} \\
    {\pi_{n-1}(F,x)} & \cdots & {\pi_0(Y,p(x))}
    \arrow["{\pi_n(i,x)}"', from=2-1, to=2-2]
    \arrow["{\pi_n(p,x)}", from=2-2, to=2-3]
    \arrow["{d_\ast}", from=2-3, to=3-1]
    \arrow[from=1-2, to=1-3]
    \arrow["{d_\ast}"', from=1-3, to=2-1]
    \arrow["{\pi_{n-1}(i,x)}"', from=3-1, to=3-2]
    \arrow["{\pi_0(p,x)}"', from=3-2, to=3-3]
  \end{tikzcd}\]
  which is natural with respect to commutative squares
  \[\begin{tikzcd}[column sep=small,row sep=small]
    X & {X'} \\
    Y & {Y'}
    \arrow["{p'}", from=1-2, to=2-2]
    \arrow[from=1-1, to=1-2]
    \arrow["p"', from=1-1, to=2-1]
    \arrow[from=2-1, to=2-2]
  \end{tikzcd}\]
  where $p,p'\in J\p$. Here $d_\ast$ is a group homomorphism $\pi_n(Y,p(x))\to\pi_{n-1}(F,x)$ when $n>1$.
\end{lemma}
\begin{proof}
  \color{red}TODO.
\end{proof}

\begin{lemma}[Hovey 2.4.17]\label{2.4.17}
  Suppose $p:X\to Y$ is a weak equivalence. Then $p^{S^n}:X^{S^n}\to Y^{S^n}$ is a weak equivalence for all $n\geq-1$, where $S^{-1}=\emptyset$.
\end{lemma}
\begin{proof}
  \color{red}TODO.
\end{proof}

\begin{proposition}[Hobey 2.4.18]\label{2.4.18}
  Suppose we have a pullback square
  % https://q.uiver.app/?q=WzAsNCxbMCwwLCJXIl0sWzAsMSwiWiJdLFsxLDEsIlkiXSxbMSwwLCJYIl0sWzAsMSwicSIsMl0sWzEsMiwiZyJdLFswLDMsImYiXSxbMywyLCJwIl0sWzAsMiwiIiwxLHsic3R5bGUiOnsibmFtZSI6ImNvcm5lci1pbnZlcnNlIn19XV0=
  \[\begin{tikzcd}
    W & X \\
    Z & Y
    \arrow["q"', from=1-1, to=2-1]
    \arrow["g", from=2-1, to=2-2]
    \arrow["f", from=1-1, to=1-2]
    \arrow["p", from=1-2, to=2-2]
    \arrow["\ulcorner"{anchor=center, pos=0.125}, draw=none, from=1-1, to=2-2]
  \end{tikzcd}\]
  in $\Top$, where $p\in J\p$ and $g$ is a weak equivalence. Then $f$ is a weak equivalence.
\end{proposition}
\begin{proof}
  \color{red}TODO.
\end{proof}

\begin{proposition}[Hovey 2.4.12]\label{2.4.12}
  $\cW\cap J\p\sseq I'\p $
\end{proposition}
\begin{proof}
  Let $p:X\to Y$ belong to $\cW\cap J\p$. By \autoref{2.4.11}, it suffices to show the map $Q(i,p)$ is a trivial fibration.\color{red}FINISH.
\end{proof}

\textbf{Questions/Comments:}\begin{enumerate}
  \item I am getting down and dirty fiddling with the details, but I feel I don't have much big-picture understanding.\
\end{enumerate}

\end{document}