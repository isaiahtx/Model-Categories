\documentclass{amsart}
% \usepackage[margin=1.3333333in]{geometry}
\usepackage[margin = 1in]{geometry}
\newcommand{\ol}{\overline}
\newcommand{\NN}{\mathbb{N}}
\newcommand{\CC}{\mathbb{C}}
\newcommand{\DD}{\mathbb{D}}
\newcommand{\RR}{\mathbb{R}}
\newcommand{\ZZ}{\mathbb{Z}}
\newcommand{\LL}{\mathbb{L}}
\newcommand{\abs}[1]{\left\lvert #1\right\rvert}
\newcommand{\wh}{\widehat}

%%hyperref for clickable links
\usepackage{hyperref}

%%%thmtools to correct autoref
\usepackage{thmtools}

%%%% biblatex

\usepackage{biblatex}
\addbibresource{refs.bib}

%%% lipsum for dummy text
\usepackage{lipsum}

%% AMS packages for math and theorems
\usepackage{amsmath, amsthm, amssymb}

%%% tikz packages for drawing and commutative diagrams
\usepackage{tikz}
\usepackage{tikz-cd}
\usepackage{tikz-3dplot}
\usepackage{quiver}

%%%% euscript for better calligraphic characters in math mode
\usepackage{eucal}[mathcal]

%%%% enumitem package for better list environments
\usepackage{enumitem}

%%% Paragraph spacing

% Paragraph spacing

%\usepackage{parskip}
%
%\setlength{\parskip}{0.5em}
%\setlength{\parindent}{1.5em}

%% comment these lines if you do not want to use bibtex for bibliography management.
%\usepackage{biblatex}
%\addbibresource{refs.bib}


%% theorems in usual style --- italicised text, bold header
\theoremstyle{plain}
\newtheorem{theorem}{Theorem}[section]
\newtheorem{corollary}[theorem]{Corollary}
\newtheorem{proposition}[theorem]{Proposition}
\newtheorem{lemma}[theorem]{Lemma}
\newtheorem*{thm*}{Theorem}

%% theorems in `definition' style --- regular text, bold header
\theoremstyle{definition}
\newtheorem{claim}[theorem]{Claim}
\newtheorem{remark}[theorem]{Remark}
\newtheorem{definition}[theorem]{Definition}
\newtheorem{exercise}[theorem]{Exercise}
\newtheorem{discussion}[theorem]{Discussion}
\newtheorem{notation}[theorem]{Notation}
\newtheorem{convention}[theorem]{Convention}
\newtheorem{conjecture}[theorem]{Conjecture}
\newtheorem{example}[theorem]{Example}

\DeclareMathOperator{\Ch}{Ch}
\newcommand{\Mod}{\mbf{Mod}}
\newcommand{\Top}{\mbf{Top}}
\newcommand{\Grp}{\mbf{Grp}}
\newcommand{\RMod}{R-\mbf{Mod}}

\newcommand{\from}{\colon}
\newcommand{\sseq}{\subseteq}
\newcommand{\wt}{\widetilde}
\newcommand{\spseq}{\supseteq}
\newcommand{\brn}{\mathbb R^n}
\newcommand{\bRn}{\mathbb R^n}
\newcommand{\0}{\mathbf{0}}
\newcommand{\bR}{\mathbb{R}}
\newcommand{\cA}{\mathcal A}
\newcommand{\cB}{\mathcal B}
\newcommand{\cC}{\mathcal C}
\newcommand{\cD}{\mathcal D}
\newcommand{\cE}{\mathcal E}
\newcommand{\cF}{\mathcal F}
\newcommand{\cG}{\mathcal G}
\newcommand{\cH}{\mathcal H}
\newcommand{\cI}{\mathcal I}
\newcommand{\cJ}{\mathcal J}
\newcommand{\cK}{\mathcal K}
\newcommand{\cL}{\mathcal L}
\newcommand{\cM}{\mathcal M}
\newcommand{\cN}{\mathcal N}
\newcommand{\cO}{\mathcal O}
\newcommand{\cP}{\mathcal P}
\newcommand{\cQ}{\mathcal Q}
\newcommand{\into}{\hookrightarrow}
\newcommand{\onto}{\twoheadrightarrow}
\newcommand{\cR}{\mathcal R}
\newcommand{\cS}{\mathcal S}
\newcommand{\cT}{\mathcal T}
\newcommand{\cU}{\mathcal U}
\newcommand{\cV}{\mathcal V}
\newcommand{\cW}{\mathcal W}
\newcommand{\cX}{\mathcal X}
\newcommand{\cY}{\mathcal Y}
\newcommand{\cZ}{\mathcal Z}
\newcommand{\mbf}[1]{\mathbf{#1}}
\renewcommand{\ol}{\overline}
\newcommand{\ul}{\underline}
\newcommand{\bZ}{\mathbb{Z}}
\newcommand{\dx}{\,\mathrm dx}
\newcommand{\dt}{\,\mathrm dt}
\newcommand{\bC}{\mathbb{C}}
\newcommand{\bN}{\mathbb{N}}
\newcommand{\bQ}{\mathbb{Q}}
\newcommand{\vare}{\varepsilon}
\renewcommand{\(}{\left(}
\renewcommand{\)}{\right)}
\newcommand\defeq{\mathrel{\overset{\makebox[0pt]{\mbox{\normalfont\tiny def}}}{=}}}
\newcommand{\phantomreplace}[2]{\makebox[0pt][l]{#1}\hphantom{#2}}
\newcommand{\phantommathreplace}[2]{\makebox[0pt][l]{$\displaystyle #1$}\hphantom{#2}}
\makeatletter
\newcommand{\skipitems}[1]{%
  \addtocounter{\@enumctr}{#1}%
}
\makeatother

%%% hyperlinks/citations in document with prettier links
\hypersetup{%
	colorlinks,%
	linkcolor={red!60!black},%
	citecolor={red!60!black},%
	urlcolor={red!60!black}%
}

%%% Shortened operators
\def\on{\operatorname}
\def\scr{\EuScript}
\def\bb{\mathbb}
\def\sf{\mathsf}
\def\cal{\mathcal}

%% blackboard bolds
%\def\RR{\bb{R}}
%\def\CC{\bb{C}}
%\def\ZZ{\bb{Z}}
%\def\NN{\bb{N}}
%\def\QQ{\bb{Q}}

% categories
\def\Cat{\on{Cat}}
\def\Set{\on{Set}}
\def\Grp{\on{Grp}}
\def\Ab{\on{Ab}}
\def\Vect{\on{Vect}}
\def\Hom{\on{Hom}}
\def\Fun{\on{Fun}}
\def\cC{\scr{C}}
\def\dD{\scr{D}}
\def\eE{\scr{E}}
\def\aA{\scr{A}}
\def\bB{\scr{B}}

\newcommand{\Cof}{\mathcal C\mathrm{of}}
\newcommand{\Fib}{\mathcal F\mathrm{ib}}
\newcommand{\W}{\mathcal W}
\newcommand{\inj}{\text-\mathrm{inj}}
\newcommand{\proj}{\text-\mathrm{proj}}
\newcommand{\fib}{\text-\mathrm{fib}}
\newcommand{\cell}{\text-\mathrm{cell}}
\newcommand{\cof}{\text-\mathrm{cof}}
\DeclareMathOperator*{\colim}{colim}
\DeclareMathOperator{\Mor}{Mor}

%%% so walker can write comments in a different color.
\usepackage{xcolor}

\newcommand{\shorten}[1]{{\color{purple} #1}}
\newcommand{\remove}[1]{{\color{red} #1}}

\newcommand{\ww}[1]{{\color{blue} Walker: #1}}
\newcommand{\ii}[1]{{\color{teal} Isaiah Question: #1}}

%%% parentheses will denote suggested text
\newcommand{\sugs}[1]{{\color{blue} Suggested rewrite: 

#1}}

%%% to strike through text without deleting it
\usepackage[normalem]{ulem}

\title{Model Structures}

\author{}
\date{\today}

\begin{document}
\maketitle


%\setcounter{tocdepth}{0}
\tableofcontents

\section{Preliminaries}

\begin{definition}[Hovey Definition 2.1.1]
  Suppose $\cC$ is a cocomplete category, and $\lambda$ is an ordinal. A \textit{$\lambda$-sequence} in $\cC$ is a colimit-preserving functor $X:\lambda\to\cC$, commonly written as
  \[X_0\to X_1\to\cdots\to X_\beta\to\cdots.\]
  Since $X$ preserves colimits, for all limit ordinals $\gamma<\lambda$, the induced map
  \[\colim_{\beta<\lambda}X_\beta\to X_\gamma\]
  is an isomorphism. We refer to the map $X_0\to \colim_{\beta<\lambda}X_\beta$ as the \textit{composition} of the $\lambda$-sequence. Given a collection $\cD$ of morphisms in $\cC$ such that every map $X_\beta\to X_{\beta+1}$ for $\beta+1<\lambda$ is in $\cD$, we refer to the composition $X_0\to\colim_{\beta<\lambda}X_\beta$ as a \textit{transfinite composition} of maps in $\cD$.
\end{definition}

\begin{definition}[Hovey Definition 2.1.2]
  Let $\gamma$ be a cardinal. An ordinal $\alpha$ is \textit{$\gamma$-filtered} if it is a limit ordinal and, if $A\sseq\alpha$ and $|A|\leq\gamma$, then $\sup A<\alpha$.
\end{definition}

\begin{definition}
  Suppose $\cC$ is a comcomplete category, $\cD\sseq\Mor\cC$ is some collection of morphisms of $\cC$, $A$ is an object of $\cC$, and $\kappa$ is a cardinal. We say that $A$ is \textit{$\kappa$-small relative to $\cD$} if, for all $\kappa$-filtered ordinals $\lambda$ and all $\lambda$-sequences
  \[X_0\to X_1\to\cdots\to X_\beta\to\cdots\]
  such that each map $X_\beta\to X_{\beta+1}$ is in $\cD$ for $\beta+1<\lambda$, the map of sets
  \[\colim_{\beta<\lambda}\cC(A,X_\beta)\to\cC(A,\colim_{\beta<\lambda}X_\beta)\]
  is an isomorphism. We say that $A$ is \textit{small relative to $\cD$} if it is $\kappa$-small relative to $\cD$ for some $\kappa$. We say that $A$ is \textit{small} if it is small relative to $\cC$ itself.
\end{definition}

\begin{definition}[Hovey Definition 2.1.7]
  Let $I$ be a class of maps in a category $\cC$.\begin{enumerate}
    \item A map is \textit{$I$-injective} if it has the right lifting property w.r.t.\ every map in $I$. The class of $I$-injective maps is denoted $I\inj$ (or $I_\perp$).
    \item A map is \textit{$I$-projective} if it has the left lifting property w.r.t.\ every map in $I$. The class of $I$-projective maps is denoted $I\proj$ (or $_\perp I$).
    \item A map is an \textit{$I$-cofibration} if it has the left lifting property w.r.t.\ every $I$-injective map. The class of $I$-cofibrations is the class $(I\inj)\proj$ and is denoted $I\cof$ (or $_\perp(I_\perp)$).
    \item A map is an \textit{$I$-fibration} if it has the right lifting property w.r.t.\ every $I$-projective map. The class of $I$-fibrations is the class $(I\proj)\inj$ and is denoted $I\fib$ (or $(_\perp I)_\perp$).
  \end{enumerate}
\end{definition}

\begin{definition}[Hovey Definition 2.1.9]
  Let $I$ be a set of maps in a cocomplete category $\cC$. A \textit{relative $I$-cell complex} is a transfinite composition of pushouts of elements of $I$. That is, if $f:A\to B$ is a relative $I$-cell complex, then there is an ordinal $\lambda$ and a $\lambda$-sequence $X:\lambda\to\cC$ such that $f$ is the composition of $X$ and such that, for each $\beta$ such that $\beta+1<\lambda$, there is a pushout square
  \[\begin{tikzcd}
    {C_\beta} & {X_\beta} \\
    {D_\beta} & {X_{\beta+1}}
    \arrow[from=1-1, to=1-2]
    \arrow[from=1-2, to=2-2]
    \arrow[from=2-1, to=2-2]
    \arrow["\ulcorner"{anchor=center, pos=0.125, rotate=180}, draw=none, from=2-2, to=1-1]
    \arrow["{g_\beta}"', from=1-1, to=2-1]
  \end{tikzcd}\]
  with $g_\beta\in I$. We denote the collection of relative $I$-cell complexes by $I\cell$. We say that $A\in\cC$ is an \textit{$I$-cell complex} if the map $0\to A$ is a relative $I$-cell complex.
\end{definition}

\begin{lemma}[Hovey 2.1.10]\label{2.1.10}
  Suppose $I$ is a class of maps in a category $\cC$ with all small colimits. Then $I\cell\sseq I\cof$.
\end{lemma}

\begin{definition}[Hovey Definition 2.1.17]\label{2.1.17}
  Suppose $\cC$ is a model category. We say that $\cC$ is \textit{cofibrantly generated} if there are sets $I$ and $J$ of maps such that:\begin{enumerate}[label=\arabic*.,noitemsep,topsep=0pt]
    \item The domains of the maps of $I$ are small relative to $I\cell$;
    \item The domains of the maps of $J$ are small relative to $J\cell$;
    \item The class of fibrations is $J\inj$; and
    \item The class of trivial fibrations is $I\inj$.
  \end{enumerate}
  We refer to $I$ as the set of \textit{generating cofibrations} and to $J$ as the set of \textit{generating trivial cofibrations}. A cofibrantly generated model category is \textit{finitely generated} if we can choose the sets $I$ and $J$ above so that the domains and codomains of $I$ and $J$ are finite relative to $I\cell$.
\end{definition}
\begin{proof}
  TODO
\end{proof}

\begin{proposition}[Hovey Proposition 2.1.18]\label{2.1.18}
  Suppose $\cC$ is a cofibrantly generated model category, with generating cofibrations $I$ and generating trivial fibrations $J$.\begin{enumerate}[label=(\alph*),noitemsep,topsep=0pt]
    \item The cofibrations form the class $I\cof$.
    \item Every cofibration is a retract of a relative $I$-cell complex.
    \item The domains of $I$ are small relative to the cofibrations.
    \item The trivial cofibrations form the class $J\cof$.
    \item Every trivial cofibration is a retract of a relative $J$-cell complex.
    \item The domains of $J$ are small relative to the trivial cofibrations.
  \end{enumerate}
  If $\cC$ is fibrantly generated, then the domains and codomains of $I$ and $J$ are finite relative to the cofibrations.
\end{proposition}

\begin{theorem}[Hovey Theorem 2.1.19]\label{2.1.19}
  Suppose $\cC$ is a complete \& cocomplete category. Suppose $\cW$ is a subcategory of $\cC$, and $I$ and $J$ are sets of maps of $\cC$. Then there is a cofibrantly generated model structure on $\cC$ with $I$ as the set of generating cofibrations, $J$ as the set of generating trivial fibrations, and $\cW$ as the subcategory of weak equivalences if and only if the following conditions are satisfied.\begin{enumerate}[label=\arabic*.,noitemsep,topsep=0pt]
    \item The subcategory $\cW$ has the 2-of-3 property and is closed under retracts.
    \item The domains of $I$ are small relative to $I\cell$.
    \item The domains of $J$ are small relative to $J\cell$.
    \item $J\cell\sseq\cW\cap I\cof$.
    \item $I\inj\sseq\cW\cap J\inj$.
    \item Either $\cW\cap I\cof\sseq J\cof$ or $\cW\cap J\inj\sseq I\inj$.
  \end{enumerate}
\end{theorem}
\begin{proof}
  TODO
\end{proof}

\begin{definition}\label{saturated}
  Let $\cC$ be a category and $I$ a collection of morphisms in $\cC$. Then if $I$ is closed under transfinite composition, pushouts, and retracts then we say $I$ is \textit{saturated}.
\end{definition}

%\begin{proposition}
%  Let $I$ be a set of maps in a cocomplete category $\cC$. Then $_\perp I$ is saturated.
%\end{proposition}
%\begin{proof}
%  Let $X$ be a $\lambda$ sequence in $\cC$ for some ordinal $\lambda$ such that the map $X_\beta\to X_{\beta+1}$ is in $_\perp I$ for $\beta+1<\lambda$. Then we wish to show that the map $X_0\to X_\lambda:=\colim_{\beta<\lambda}X_\beta$ also is in $_\perp I$. Suppose we are given a lifting problem of the form
%  % https://q.uiver.app/?q=WzAsNCxbMCwwLCJYXzAiXSxbMCwxLCJYX1xcbGFtYmRhIl0sWzEsMSwiQiJdLFsxLDAsIkEiXSxbMCwxXSxbMSwyXSxbMCwzXSxbMywyLCJmIl1d
%  \[\begin{tikzcd}
%    {X_0} & A \\
%    {X_\lambda} & B
%    \arrow[from=1-1, to=2-1]
%    \arrow[from=2-1, to=2-2]
%    \arrow[from=1-1, to=1-2]
%    \arrow["f", from=1-2, to=2-2]
%  \end{tikzcd}\]
%  where $f\in I$.
%\end{proof}

%\section{Chain complexes of modules over a ring}
%
%
%\begin{definition}
  %Let $R$ be a ring. Given an $R$-module $M$, define $S^n(M),D^n(M)\in\Ch(R)$ by
  %\[S^n(M)_k:=\begin{cases}
    %M & k=n \\
    %0 & \text{else }
  %\end{cases}\quad\quad\quad\text{ and }\quad\quad\quad D^n(M):=\begin{cases}
    %M & k=n,n-1 \\
    %0 & \text{else},
  %\end{cases}\]
  %where the differential in $d_n$ in $D^n(M)$ is the identity. We denote $S^n(R)$ by simply $S^n$ and $D^n(R)$ by simply $D^n$. Define $I:=\bigcup_{n\in\bZ}\Ch(R)(S^{n-1},D^n)$, and define $J:=\bigcup_{n\in\bZ}\Ch(R)(0,D^n)$.
%
  %Now, define $\Fib_R:=J\inj$, $\Cof_R:=I\cof$, and define $\W_R$ to be the class of quasi-isomorphisms (i.e., those maps $f:A_\bullet\to B_\bullet$ such that $H_n(f):H_n(A_\bullet)\to H_n(B_\bullet)$ is an isomorphism for all $n\in\bZ$).
%\end{definition}

\section{Topological Spaces}

An injective map $f:X\to Y$ in $\Top$ is an \textit{inclusion} if $U$ is open in $X$ if and only if there is a $V$ open in $Y$ such that $f^{-1}(V)=U$. If $f$ is a closed inclusion and every point in $Y\setminus f(X)$ is closed, then we call $f$ a \textit{closed $T_1$ inclusion}. We will let $\cT$ denote the class of closed $T_1$ inclusions in $\Top$.

The symbol $D^n$ will denote the unit disk in $\bR^n$, and the symbol $S^{n-1}$ will denote the unit sphere in $\bR^n$, so that we have the boundary inclusions $S^{n-1}\into D^n$. In particular, for $n=0$ we let $D^0=\{0\}$ and $S^{-1}=\emptyset$.

\begin{definition}
  A map $f:X\to Y$ in $\Top$ is called a \textit{weak equivalence} if
  \[\pi_n(f,x):\pi_n(X,x)\to\pi_n(Y,f(x))\]
  is an isomorphism for all $n\geq0$ and for all $x\in X$.

  Define the set of maps $I'$ to consist of all the boundary inclusion $S^{n-1}\into D^n$ for all $n\geq0$, and define the set $J$ to consist of all the inclusions $D^n\into D^n\times I$ mapping $x\mapsto(x,0)$ for $n\geq0$. Then a map $f$ will be called a \textit{cofibration} if it is in $I'\cof={_\perp({I'}_\perp)}$, and a \textit{fibration} if it is in $J\inj=J_\perp$.

  A map in $I'\cell$ is usually called a \textit{relative cell complex}; a relative CW-complex is a special case of a relative cell complex, where, in particular, the cells can be attached in order of their dimension. Note that in particular maps of $J$ are relative CW complexes, hence are relative $I$-cell complexes. 
  %Thus $J\cof\sseq I'\cof$.
  A fibration is often known as a \textit{Serre fibration} in the literature.
\end{definition}

\begin{theorem}[Hovey Theorem 2.4.19]\label{2.4.19}
  There is a finitely generated model structure on $\Top$ with $I'$ as the set of generating cofibrations, $J$ as the set of generating trivial cofibrations, and the cofibrations, fibrations, and weak equivalences as above. Every object of $\Top$ is fibrant, and the cofibrant objects are retracts of relative cell complexes.
\end{theorem}
\begin{proof}
  We will apply \autoref{2.1.19} to get that there is a cofibrantly generated model structure on $\Top$ with $I'$ as the set of generating cofibrations, $J$ as the set of generating trivial fibrations, and $\cW$ as the subcategory of weak equivalences. The six requirements outlined in the theorem will be verified like so:
  \begin{enumerate}[label=\arabic*.,noitemsep,topsep=0pt]
    \item $\cW$ is a subcategory of $\cC$ which has the 2-of-3 property and is closed under retracts: \autoref{2.4.4}.
    \item The domains of $I'$ are small relative to $I'\cell$: In {\color{red}Hovey 2.4.1}, we will show that every space is small relative to the inclusions, and in particular every space is small relative to the class $\cT$ of closed $T_1$ inclusions. Hence, it will suffice to show that $I'\cell\sseq\cT$. In \autoref{2.4.5-6}, we will show that $\cT$ is saturated, and clearly every map in $I'$ is a closed $T_1$ inclusion, so the desired result follows.
    \item The domains of $J$ are small relative to $J\cell$: By the same argument given above, this will follow by {\color{red}Hovey 2.4.1}, \autoref{2.4.5-6}, and the fact that $J\sseq\cT$.
    \item $J\cell\sseq\cW\cap I'\cof$: In {\color{red}Hovey 2.4.9}, we will show $J\cof\sseq\cW\cap I'\cof$, and by \autoref{2.1.10} $J\cell\sseq J\cof$.
    \item $I'\inj\sseq\cW\cap J\inj$: {\color{red}Hovey 2.4.10}
    \item $\cW\cap J\inj\sseq I'\inj$: {\color{red}Hovey 2.4.12}
  \end{enumerate}
  It will follow by the definition of a cofibrantly generated model structure (\autoref{2.1.17}) that the fibrations in this model structure are given by $J\inj$, which is precisely how we defined it. By \autoref{2.1.18}, the class of cofibrations will be given by $I'\cof$, which is likewise exactly how we defined them.

  In {\color{red}Hovey 2.4.2}, we will show that compact spaces are finite relative to the class $\cT$ of closed $T_1$ inclusions. Hence, this model structure will be finitely generated, as the domains and codomains of $I'$ and $J$ are all compact, and by the reasoning given above we will have shown $I'\cell\sseq\cT$.
  
  Finally, we will show that every object of $\Top$ is fibrant in {\color{red}Hovey 2.4.14}, and that the cofibrant objects are retracts of relative cell complexes in \autoref{}.
\end{proof}

\begin{proposition}[Hovey 2.4.5 \& 2.4.6]\label{2.4.5-6}
  The class of closed $T_1$ inclusions is saturated.
\end{proposition}
\begin{proof}
  \color{red}TODO.
\end{proof}

\begin{lemma}[Hovey Lemma 2.4.4]\label{2.4.4}
  The weak equivalences in $\Top$ are closed under retracts and satisfy 2-of-3 axiom (so that in particular the weak equivalences form a subcategory, as clearly identities are weak equivalences).
\end{lemma}
\begin{proof}
  First we show that weak equivalences satisfy 2-of-3. Let $f:X\to Y$ and $g:Y\to Z$ be continuous functions of topological spaces. 
  
  First of all, suppose $f$ and $g$ are both weak equivalences. Then by functoriality of $\pi_n$, since $\pi_n(f,x)$ and $\pi_n(g,f(x))$ are isomorphisms for all $x\in X$, $\pi_n(g\circ f,x)=\pi_n(g,f(x))\circ\pi_n(f,x)$ is likewise an isomorphism for all $x\in X$, so that $g\circ f$ is a weak equivalence.

  Now, suppose that $g\circ f$ and $g$ are weak equivalences. Pick a point $x\in X$. We wish to show that $\pi_n(f,x):\pi_n(X,x)\to\pi_n(Y,f(x))$ is an isomorphism for all $n\geq0$. We know that $\pi_n(g\circ f,x)$ is an isomorphism, and $\pi_n(g,f(x))$ is an isomorphism, say with inverse, $\varphi$, so that
  \[\varphi\circ\pi_n(g\circ f,x)=\varphi\circ\pi_n(g,f(x))\circ\pi_n(f,x)=\pi_n(f,x)\]
  is an isomorphism, as it is a composition of isomorphisms.

  Now, suppose that $g\circ f$ and $f$ are weak equivalences. Pick a point $y\in Y$. Since $\pi_0(f)$ is an isomorphism, there exists a point $x\in X$ such that $f(x)$ belongs to the path component containing $y$, so that there exists some $\alpha:I\to Y$ with $\alpha(0)=f(x)$ and $\alpha(1)=f(y)$. Then consider the following diagram
  % https://q.uiver.app/?q=WzAsNCxbMCwwLCJcXHBpX24oWSx5KSJdLFsxLDAsIlxccGlfbihaLGcoeSkpIl0sWzAsMSwiXFxwaV9uKFksZih4KSkiXSxbMSwxLCJcXHBpX24oWixnKGYoeCkpKSJdLFswLDEsIlxccGlfbihnLHkpIl0sWzAsMl0sWzIsMywiXFxwaV9uKGcsZih4KSkiLDJdLFsxLDNdXQ==
  \[\begin{tikzcd}
    {\pi_n(Y,y)} & {\pi_n(Z,g(y))} \\
    {\pi_n(Y,f(x))} & {\pi_n(Z,g(f(x)))}
    \arrow["{\pi_n(g,y)}", from=1-1, to=1-2]
    \arrow[from=1-1, to=2-1]
    \arrow["{\pi_n(g,f(x))}", from=2-1, to=2-2]
    \arrow[from=1-2, to=2-2]
  \end{tikzcd}\]
  where the left arrow is the isomorphism given by conjugation by the path $\alpha$, and the right arrow is the isomorphism given by conjugation by the path $g\circ\alpha$. It is tedious yet straightforward to verify that the diagram commutes.
  Furthermore, we know that $\pi_n(f,x)$ and $\pi_n(g\circ f,x)=\pi_n(g,f(x))\circ\pi_n(f,x)$ are isomorphisms for all $n$, so that if we denote the inverse of $\pi_n(f,x)$ by $\varphi$, then
  \[\pi_n(g\circ f,x)\circ\varphi=\pi_n(g,f(x))\circ\pi_n(f,x)\circ\varphi=\pi_n(g,f(x))\]
  is an isomorphism, as it is given as a composition of isomorphisms. Hence, the top arrow must likewise be an isomorphism, precisely the desired result.

  The fact that weak equivalences in $\Top$ are closed under retracts is entirely straightforward and follows from the fact that the class of isomorphisms in any category is closed under retracts.
\end{proof}

Questions:\begin{enumerate}
  \item What is an example of a relative cell complex that is not a CW complex?
\end{enumerate}

\end{document}