\documentclass{amsart}
% \usepackage[margin=1.3333333in]{geometry}
\usepackage[margin = 1in]{geometry}
\newcommand{\ol}{\overline}
\newcommand{\NN}{\mathbb{N}}
\newcommand{\CC}{\mathbb{C}}
\newcommand{\DD}{\mathbb{D}}
\newcommand{\RR}{\mathbb{R}}
\newcommand{\ZZ}{\mathbb{Z}}
\newcommand{\LL}{\mathbb{L}}
\newcommand{\abs}[1]{\left\lvert #1\right\rvert}
\newcommand{\wh}{\widehat}

%%hyperref for clickable links
\usepackage{hyperref}

%%%thmtools to correct autoref
\usepackage{thmtools}

%%%% biblatex

\usepackage{biblatex}
\addbibresource{refs.bib}

%%% lipsum for dummy text
\usepackage{lipsum}

%% AMS packages for math and theorems
\usepackage{amsmath, amsthm, amssymb}

%%% tikz packages for drawing and commutative diagrams
\usepackage{tikz}
\usepackage{tikz-cd}
\usepackage{tikz-3dplot}
\usepackage{quiver}

%%%% euscript for better calligraphic characters in math mode
\usepackage{eucal}[mathcal]

%%%% enumitem package for better list environments
\usepackage{enumitem}

%%% Paragraph spacing

% Paragraph spacing

%\usepackage{parskip}
%
%\setlength{\parskip}{0.5em}
%\setlength{\parindent}{1.5em}

%% comment these lines if you do not want to use bibtex for bibliography management.
%\usepackage{biblatex}
%\addbibresource{refs.bib}


%% theorems in usual style --- italicised text, bold header
\theoremstyle{plain}
\newtheorem{theorem}{Theorem}[section]
\newtheorem{corollary}[theorem]{Corollary}
\newtheorem{proposition}[theorem]{Proposition}
\newtheorem{lemma}[theorem]{Lemma}
\newtheorem*{thm*}{Theorem}

%% theorems in `definition' style --- regular text, bold header
\theoremstyle{definition}
\newtheorem{claim}[theorem]{Claim}
\newtheorem{remark}[theorem]{Remark}
\newtheorem{definition}[theorem]{Definition}
\newtheorem{exercise}[theorem]{Exercise}
\newtheorem{discussion}[theorem]{Discussion}
\newtheorem{notation}[theorem]{Notation}
\newtheorem{convention}[theorem]{Convention}
\newtheorem{conjecture}[theorem]{Conjecture}
\newtheorem{example}[theorem]{Example}

\DeclareMathOperator{\Ch}{Ch}
\newcommand{\Mod}{\mbf{Mod}}
\newcommand{\Top}{\mbf{Top}}
\newcommand{\Grp}{\mbf{Grp}}
\newcommand{\RMod}{R-\mbf{Mod}}

\newcommand{\from}{\colon}
\newcommand{\sseq}{\subseteq}
\newcommand{\wt}{\widetilde}
\newcommand{\spseq}{\supseteq}
\newcommand{\brn}{\mathbb R^n}
\newcommand{\bRn}{\mathbb R^n}
\newcommand{\0}{\mathbf{0}}
\newcommand{\bR}{\mathbb{R}}
\newcommand{\cA}{\mathcal A}
\newcommand{\cB}{\mathcal B}
\newcommand{\cC}{\mathcal C}
\newcommand{\cD}{\mathcal D}
\newcommand{\id}{\mathrm{id}}
\newcommand{\cE}{\mathcal E}
\newcommand{\cF}{\mathcal F}
\newcommand{\cG}{\mathcal G}
\newcommand{\cH}{\mathcal H}
\newcommand{\cI}{\mathcal I}
\newcommand{\p}{{_\perp}}
\newcommand{\cJ}{\mathcal J}
\newcommand{\cK}{\mathcal K}
\newcommand{\cL}{\mathcal L}
\newcommand{\cM}{\mathcal M}
\newcommand{\cN}{\mathcal N}
\newcommand{\cO}{\mathcal O}
\newcommand{\cP}{\mathcal P}
\newcommand{\cQ}{\mathcal Q}
\newcommand{\into}{\hookrightarrow}
\newcommand{\onto}{\twoheadrightarrow}
\newcommand{\cR}{\mathcal R}
\newcommand{\cS}{\mathcal S}
\newcommand{\cT}{\mathcal T}
\newcommand{\cU}{\mathcal U}
\newcommand{\cV}{\mathcal V}
\newcommand{\cW}{\mathcal W}
\newcommand{\cX}{\mathcal X}
\newcommand{\cY}{\mathcal Y}
\newcommand{\cZ}{\mathcal Z}
\newcommand{\mbf}[1]{\mathbf{#1}}
\renewcommand{\ol}{\overline}
\newcommand{\ul}{\underline}
\newcommand{\bZ}{\mathbb{Z}}
\newcommand{\dx}{\,\mathrm dx}
\newcommand{\dt}{\,\mathrm dt}
\newcommand{\bC}{\mathbb{C}}
\newcommand{\bN}{\mathbb{N}}
\newcommand{\bQ}{\mathbb{Q}}
\newcommand{\vare}{\varepsilon}
\renewcommand{\(}{\left(}
\renewcommand{\)}{\right)}
\newcommand\defeq{\mathrel{\overset{\makebox[0pt]{\mbox{\normalfont\tiny def}}}{=}}}
\newcommand{\phantomreplace}[2]{\makebox[0pt][l]{#1}\hphantom{#2}}
\newcommand{\phantommathreplace}[2]{\makebox[0pt][l]{$\displaystyle #1$}\hphantom{#2}}
\makeatletter
\newcommand{\skipitems}[1]{%
  \addtocounter{\@enumctr}{#1}%
}
\makeatother

%%% hyperlinks/citations in document with prettier links
\hypersetup{%
	colorlinks,%
	linkcolor={red!60!black},%
	citecolor={red!60!black},%
	urlcolor={red!60!black}%
}

%%% Shortened operators
\def\on{\operatorname}
\def\scr{\EuScript}
\def\bb{\mathbb}
\def\sf{\mathsf}
\def\cal{\mathcal}

%% blackboard bolds
%\def\RR{\bb{R}}
%\def\CC{\bb{C}}
%\def\ZZ{\bb{Z}}
%\def\NN{\bb{N}}
%\def\QQ{\bb{Q}}

% categories
\def\Cat{\on{Cat}}
\def\Set{\on{Set}}
\def\Grp{\on{Grp}}
\def\Ab{\on{Ab}}
\def\Vect{\on{Vect}}
\def\Hom{\on{Hom}}
\def\Fun{\on{Fun}}
\def\cC{\scr{C}}
\def\dD{\scr{D}}
\def\eE{\scr{E}}
\def\aA{\scr{A}}
\def\bB{\scr{B}}

\newcommand{\Cof}{\mathcal C\mathrm{of}}
\newcommand{\Fib}{\mathcal F\mathrm{ib}}
\newcommand{\W}{\mathcal W}
\newcommand{\inj}{\text-\mathrm{inj}}
\newcommand{\proj}{\text-\mathrm{proj}}
\newcommand{\fib}{\text-\mathrm{fib}}
\newcommand{\cell}{\text-\mathrm{cell}}
\newcommand{\cof}{\text-\mathrm{cof}}
\DeclareMathOperator*{\colim}{colim}
\DeclareMathOperator{\Mor}{Mor}

%%% so walker can write comments in a different color.
\usepackage{xcolor}

\newcommand{\shorten}[1]{{\color{purple} #1}}
\newcommand{\remove}[1]{{\color{red} #1}}

\newcommand{\ww}[1]{{\color{blue} Walker: #1}}
\newcommand{\ii}[1]{{\color{teal} Isaiah Question: #1}}

%%% parentheses will denote suggested text
\newcommand{\sugs}[1]{{\color{blue} Suggested rewrite: 

#1}}

%%% to strike through text without deleting it
\usepackage[normalem]{ulem}

\title{Model Structures}

\author{}
\date{\today}

\begin{document}
\maketitle


%\setcounter{tocdepth}{0}
\tableofcontents

\section{Preliminaries}

\begin{definition}[Hovey Definition 2.1.1]
  Suppose $\cC$ is a cocomplete category, and $\lambda$ is an ordinal. A \textit{$\lambda$-sequence} in $\cC$ is a colimit-preserving functor $X:\lambda\to\cC$, commonly written as
  \[X_0\to X_1\to\cdots\to X_\beta\to\cdots.\]
  Since $X$ preserves colimits, for all limit ordinals $\gamma<\lambda$, the induced map
  \[\colim_{\beta<\gamma}X_\beta\to X_\gamma\]
  is an isomorphism. We refer to the map $X_0\to \colim_{\beta<\lambda}X_\beta$ as the \textit{composition} of the $\lambda$-sequence. Given a collection $\cD$ of morphisms in $\cC$ such that every map $X_\beta\to X_{\beta+1}$ for $\beta+1<\lambda$ is in $\cD$, we refer to the composition $X_0\to\colim_{\beta<\lambda}X_\beta$ as a \textit{transfinite composition} of maps in $\cD$.\footnote{To be more precise, there may be different (isomorphic) choices of colimit $\colim_{\beta<\gamma}X_\beta$, which give rise to different choices of composition $X_0\to\colim_{\beta<\gamma}X_\beta$. Thus, the composition of a $\lambda$-sequence is only unique up to composition by a unique isomorphism.}
\end{definition}

\begin{definition}[Hovey Definition 2.1.2]
  Let $\gamma$ be a cardinal. An ordinal $\alpha$ is \textit{$\gamma$-filtered} if it is a limit ordinal and, if $A\sseq\alpha$ and $|A|\leq\gamma$, then $\sup A<\alpha$.
\end{definition}

Given a cardinal $\gamma$, a $\gamma$-filtered category is one such that any diagram $\cD\to\cC$ has a cocone where $\cD$ has $<\gamma$ arrows. A catgory is just ``filtered'' if it is $\omega$-filtered, i.e., if every finite diagram in $\cC$ admits a cocone. Note that an ordinal $\alpha$ is $\gamma$-filtered precisely when it is $\gamma$-filtered as a category, and in particular every ordinal is $\omega$-filtered.

\begin{definition}[Hovey Definition 2.1.3]
  Suppose $\cC$ is a comcomplete category, $\cD\sseq\Mor\cC$ is some collection of morphisms of $\cC$, $A$ is an object of $\cC$, and $\kappa$ is a cardinal. We say that $A$ is \textit{$\kappa$-small relative to $\cD$} if, for all $\kappa$-filtered ordinals $\lambda$ and all $\lambda$-sequences
  \[X_0\to X_1\to\cdots\to X_\beta\to\cdots\]
  such that each map $X_\beta\to X_{\beta+1}$ is in $\cD$ for $\beta+1<\lambda$, the map of sets
  \[\colim_{\beta<\lambda}\cC(A,X_\beta)\to\cC(A,\colim_{\beta<\lambda}X_\beta)\]
  is an isomorphism. We say that $A$ is \textit{small relative to $\cD$} if it is $\kappa$-small relative to $\cD$ for some $\kappa$. We say that $A$ is \textit{small} if it is small relative to $\cC$ itself.
\end{definition}

Recall that given a small category $\cD$ and a functor $F:\cD\to\Set$, we may explicitly construct the colimit of $F$ as the set
\[\colim F:=\(\coprod_{d\in \cD}F(d)\)/\sim,\]
where the equivalence relation $\sim$ is \textbf{generated} by
\[((x\in F(d))\sim(x'\in F(d')))\quad\text{ if }\quad(\exists(f:d\to d')\text{ with }Ff(x)=x').\]
In particular, if $\cD$ is a filtered category then the resulting relation can be described as follows:
\begin{equation}\label{eq1}
  ((x\in F(d))\sim(x'\in F(d')))\quad\text{ iff }\quad(\exists\ d'',\,(f:d\to d''),\,(g:d'\to d'')\text{ with }Ff(x)=Fg(x')).
\end{equation}
given a cone $\eta:F\Rightarrow\underline Y$ under $F$, the unique map $\colim F\to Y$ maps the equivalence class of $x\in F(d)$ to the element $\eta_d(x)\in X$. We will use this characterization of the colimit in the following example.

\begin{example}[Hovey 2.1.5]\label{2.1.5}
  Every set is small. Indeed, if $A$ is a set we claim that $A$ is $|A|$-small. To see this, suppose $\lambda$ is an $|A|$-filtered ordinal, and $X$ is a $\lambda$-sequence of sets. Given $\alpha<\beta<\lambda$, let $\iota_{\alpha,\beta}:X_\alpha\to X_\beta$ denote the induced morphism. We will write $X_\lambda:=\colim_{\beta<\lambda}X_\beta$, and let $\iota:X\Rightarrow X_\lambda$ be the colimit cone, so that given $\beta<\lambda$, $\iota_\beta:X_\beta\to X_\lambda$ is the leg of the colimit cone at $X_\beta$. By composing with the functor $\cC(A,-):\Set\to\Set$, we get another $\lambda$-sequence $\{\cC(X_\beta,A)\}_{\beta<\lambda}$. The cone $\iota$ under $X$ induces a cone $\iota_*$ under $\cC(X_\beta,A)$ with nadir $\cC(A,X_\lambda)$. Let $\eta:\cC(X_\beta,A)\Rightarrow\underline{\colim_{\beta<\lambda}\cC(X_\beta,A)}$ be the colimit cone, and let $\ell:\colim_{\beta<\lambda}\cC(A,X_\lambda)\to\cC(A,X_\lambda)$ be the unique morphism of cones so that the following diagram commutes
  \[\begin{tikzcd}
    {\cC(A,X_0)} && {\cC(A,X_1)} && \cdots && {\cC(A,X_\beta)} && \cdots \\
    \\
    &&&& {\colim_{\beta<\lambda}\cC(A,X_\beta)} \\
    \\
    &&&& {\cC(A,X_\lambda)}
    \arrow["{(\iota_{0,1})_*}", from=1-1, to=1-3]
    \arrow["{(\iota_{1,2})_*}", from=1-3, to=1-5]
    \arrow[from=1-5, to=1-7]
    \arrow["{(\iota_{\beta,\beta+1})_*}", from=1-7, to=1-9]
    \arrow["{\eta_\beta}"', from=1-7, to=3-5]
    \arrow["{\eta_1}", from=1-3, to=3-5]
    \arrow["{\eta_0}"{pos=0.3}, from=1-1, to=3-5]
    \arrow["{(\iota_\beta)_*}", from=1-7, to=5-5]
    \arrow["\ell", dashed, from=3-5, to=5-5]
    \arrow["{(\iota_1)_*}"'{pos=0.4}, from=1-3, to=5-5]
    \arrow["{(\iota_0)_*}"', from=1-1, to=5-5]
  \end{tikzcd}\]
  First, we wish to show that $\ell$ is surjective. Indeed, let $f:A\to X_\lambda$. For each $a\in A$, there exists some $\beta_a\in\lambda$ and some $a'\in X_{\beta_a}$ such that $f(a)=\eta_{\beta_a}(a')$ (see the preceeding discussion). Then let $\gamma:=\sup_{a\in A}\beta_a$. Since $|\{\beta_a\}_{a\in A}|\leq|A|$ and $\lambda$ is $|A|$-filtered, necessarily $\gamma<\lambda$. Now, define $g:A\to X_\gamma$ like so: for $a\in A$, define $g(a):=\iota_{\beta_a,\gamma}(a')$, where $a'\in X_{\beta_a}$ was chosen earlier so that $\iota_{\beta_a}(a')=f(a)$. Then we claim that $\ell(\eta_\gamma(g))=f$. Indeed, as $\ell$ is a morphism of cocones, $\ell\circ\eta=\iota_*$, so that we have
  \[\ell(\eta_\gamma(g))=(\iota_\gamma)_*(g)=\iota_\gamma\circ g,\]
  and given $a\in A$ we have 
  \[\iota_\gamma(g(a))=\iota_\gamma(\iota_{\beta_a,\gamma}(a')).\] 
  By definition of a cone, $\iota_{\gamma}\circ\iota_{\beta_a,\gamma}=\iota_{\beta_a}$, so that
  \[\ell(\eta_{\gamma}(g))(a)=\iota_\gamma(\iota_{\beta_a,\gamma}(a'))=\iota_{\beta_a}(a')=f(a),\]
  so that indeed $\ell(\eta_\gamma(g))=f$.

  It remains to show $\ell$ is injective. Suppose we are given $[f],[g]\in\colim_{\beta<\lambda}\cC(A,X_\beta)$ such that $\ell([f])=\ell([g])$. Then by the preceeding discussion, there exists $\alpha,\beta<\lambda$, $f\in\cC(A,X_\alpha)$, and $g\in\cC(A,X_\beta)$ such that $\eta_\alpha(f)=[f]$ and $\eta_\beta(g)=[g]$. 
  Then since $\ell\circ\eta=\iota_*$, we have
  \[\ell([f])=\ell([g])\implies\iota_\alpha\circ f= (\iota_\alpha)_*(f)=\ell(\eta_\alpha(f))=\ell(\eta_\beta(g))=(\iota_\beta)_*(g)=\iota_\beta\circ g.\]
  %Now, WLOG assume $\alpha\leq\beta$. 
  For each $a\in A$, since $\iota_\alpha(f(a))=\iota_\beta(g(a))$, by \autoref{eq1} there exists $\gamma_a$ with $\alpha,\beta\leq\gamma_a$ such that $\iota_{\alpha,\gamma_a}(f(a))=\iota_{\beta,\gamma_a}(g(a))$. Then let $\gamma:=\sup_{a\in A}\gamma_a$. Since $|\{\gamma_a\}_{a\in A}|\leq|A|$ and $\lambda$ is $|A|$-filtered, necessarily $\gamma<\lambda$. Now, in order to show $[f]=[g]$, by \autoref{eq1} it suffices to show that $(\iota_{\alpha,\gamma})_*(f)=(\iota_{\beta,\gamma})_*(g)$. Indeed, given $a\in A$, we have
  \[(\iota_{\alpha,\gamma})_*(f)(a)=\iota_{\alpha,\gamma}(f(a))=\iota_{\gamma_a,\gamma}\circ\iota_{\alpha,\gamma_a}(f(a))=\iota_{\gamma_a,\gamma}\circ\iota_{\beta,\gamma_a}(g(a))=\iota_{\beta,\gamma}(g(a))=(\iota_{\beta,\gamma})_*(g)(a),\]
  precisely the desired result..
\end{example}

\begin{definition}[Hovey Definition 2.1.7]
  Let $I$ be a class of maps in a category $\cC$.\begin{enumerate}
    \item A map is \textit{$I$-injective} if it has the right lifting property w.r.t.\ every map in $I$. The class of $I$-injective maps is denoted $I\inj$ (or $I\p$).
    \item A map is \textit{$I$-projective} if it has the left lifting property w.r.t.\ every map in $I$. The class of $I$-projective maps is denoted $I\proj$ (or $\p I$).
    \item A map is an \textit{$I$-cofibration} if it has the left lifting property w.r.t.\ every $I$-injective map. The class of $I$-cofibrations is the class $(I\inj)\proj$ and is denoted $I\cof$ (or $\p(I\p)$).
    \item A map is an \textit{$I$-fibration} if it has the right lifting property w.r.t.\ every $I$-projective map. The class of $I$-fibrations is the class $(I\proj)\inj$ and is denoted $I\fib$ (or $(\p I)\p$).
  \end{enumerate}
\end{definition}

\begin{lemma}\label{useful_LP_properties}
  Given classes $A$ and $B$ of maps in a category $\cC$ with $A\sseq B$, $A\sseq {\p(A\p)}$, $A\sseq (\p A)\p$, $(\p(A\p))\p=A\p$, $\p((\p A)\p)={\p A}$, $A\p\spseq B\p$, $\p A\spseq {\p B}$, ${\p(A\p)}\sseq {\p(B\p)}$, and $(\p A)\p\sseq (\p B)\p$.
\end{lemma}
\begin{proof}
  \color{red}TODO.
\end{proof}

\begin{definition}[Hovey Definition 2.1.9]
  Let $I$ be a set of maps in a cocomplete category $\cC$. A \textit{relative $I$-cell complex} is a transfinite composition of pushouts of elements of $I$. That is, if $f:A\to B$ is a relative $I$-cell complex, then there is an ordinal $\lambda$ and a $\lambda$-sequence $X:\lambda\to\cC$ such that $f$ is the composition of $X$ and such that, for each $\beta$ such that $\beta+1<\lambda$, there is a pushout square
  \[\begin{tikzcd}
    {C_\beta} & {X_\beta} \\
    {D_\beta} & {X_{\beta+1}}
    \arrow[from=1-1, to=1-2]
    \arrow[from=1-2, to=2-2]
    \arrow[from=2-1, to=2-2]
    \arrow["\ulcorner"{anchor=center, pos=0.125, rotate=180}, draw=none, from=2-2, to=1-1]
    \arrow["{g_\beta}"', from=1-1, to=2-1]
  \end{tikzcd}\]
  with $g_\beta\in I$. We denote the collection of relative $I$-cell complexes by $I\cell$. We say that $A\in\cC$ is an \textit{$I$-cell complex} if the map $0\to A$ is a relative $I$-cell complex.
\end{definition}

\begin{lemma}\label{I-cell_closed_under_composition_with_isomorphisms}
  Let $\cC$ be a category and $I$ a class of morphisms in $\cC$. Then $I\cell$ is closed under composition with isomorphisms.
\end{lemma}
\begin{proof}
  Suppose that $f:B\to C$ is an element of $I\cell$, and $h:A\to B$ and $g:C\to D$ are isomorphisms in $\cC$. We wish to show $f\circ h$ and $g\circ f$ are also elements of $I\cell$. Since $f\in I\cell$, there exists an ordinal $\lambda$, a $\lambda$-sequence $X$ with $X_0=B$, and a colimit cone $\eta:X\Rightarrow\underline C$, such that $\eta_0=f$. 
  
  First of all, construct a new cone $\eta':X\Rightarrow\underline D$ under $X$ where $\eta'_\beta:=g\circ\eta_\beta$. It is straightforward to verify that $\eta'$ is a colimit cone for $X$ since $\eta$ is a colimit cone and $g$ is an isomorphism. Thus, $g\circ f=g\circ\eta_0=\eta_0'\in I\cell$, as $\eta_0'$ is the composition of a sequence of pushouts of elements of $I$.

  On the other hand, we may construct a new $\lambda$-sequence $X'$ by defining $X'_0=A$, $X_\beta'=X_\beta$ for all $0<\beta<\lambda$, the map $X_0'\to X_\beta'$ for $0<\beta<\lambda$ to be the composition
  \[\begin{tikzcd}
    A & {B=X_0} & {X_\beta},
    \arrow["h", from=1-1, to=1-2]
    \arrow[from=1-2, to=1-3]
  \end{tikzcd}\]
  and the composition $X'_\alpha\to X'_\beta$ to simply be the same map $X_\alpha\to X_\beta$ for $0<\alpha\leq \beta<\lambda$. It is straightforward to verify that defines a $\lambda$-sequence, and that we may define a colimit cone $\eta':X'\Rightarrow\underline C$ by $\eta'_0=\eta_0\circ h=f\circ h$, and $\eta'_\beta=\eta_\beta$ for $0<\beta<\lambda$. Furthermore, clearly for all $1<\beta+1<\lambda$, we have the arrow $X_\beta'\to X_{\beta+1}'$ is a pushout of a map in $I$. Thus, in order to show $f\circ h\in I\cell$, it remains to show that the arrow $A=X_0'\to X_1'=X_1$ is a pushout of a map in $I$. Indeed, we know since $B=X_0\to X_1$ is a pushout of a map $k:P\to Q$ in $I$, and it can be easily verified the diagram on the right is a pushout diagram:
  \[\begin{tikzcd}[row sep=small,column sep=small]
    P && {X_0} && P & {X_0} & {X_0'} \\
    &&& \leadsto &&& {X_0} \\
    Q && {X_1} && Q && {X_1'}
    \arrow[from=1-3, to=3-3]
    \arrow[from=3-1, to=3-3]
    \arrow[from=1-1, to=1-3]
    \arrow["k"', from=1-1, to=3-1]
    \arrow["\ulcorner"{anchor=center, pos=0.125, rotate=180}, draw=none, from=3-3, to=1-1]
    \arrow[from=1-5, to=1-6]
    \arrow["h", from=1-7, to=2-7]
    \arrow[from=2-7, to=3-7]
    \arrow[from=1-5, to=3-5]
    \arrow[from=3-5, to=3-7]
    \arrow["{h^{-1}}", from=1-6, to=1-7]
    \arrow["\ulcorner"{anchor=center, pos=0.125, rotate=180}, draw=none, from=3-7, to=1-5]
  \end{tikzcd}\qedhere\]
\end{proof}

\begin{lemma}[Hovey 2.1.10]\label{2.1.10}
  Suppose $I$ is a class of maps in a category $\cC$ with all small colimits. Then $I\cell\sseq{\p(I\p)}$.
\end{lemma}
\begin{proof}
  \color{red}TODO.
\end{proof}

\begin{theorem}[Small Object Argument, Hovey 2.1.14]\label{2.1.14}
  Suppose $\cC$ is a cocomplete categroy, and $I$ is a set of maps in $\cC$. Suppose the domains of the maps of $I$ are small relative to $I\cell$. Then there is a functorial factorization $(\gamma,\delta)$ on $\cC$ such that for all morphisms $f\in\cC$, the map $\gamma(f)$ is in $I\cell$ and the map $\delta(f)$ is in $I\inj$.
\end{theorem}
\begin{proof}
  \color{red}TODO.
\end{proof}

\begin{corollary}[Hovey 2.1.15]\label{2.1.15}
  Suppose that $I$ is a set of maps in a cocomplete category $\cC$. Suppose as well that the domains of $I$ are small relative to $I\cell$. Then given $f:A\to B$ in $\p(I\p)$, there is a $g:A\to C$ in $I\cell$ such that $f$ is a retract of $g$ by a map which fixes $A$.
\end{corollary}
\begin{proof}
  \color{red}TODO
\end{proof}

\begin{definition}[Hovey Definition 2.1.17]\label{2.1.17}
  Suppose $\cC$ is a model category. We say that $\cC$ is \textit{cofibrantly generated} if there are sets $I$ and $J$ of maps such that:\begin{enumerate}[label=\arabic*.,noitemsep,topsep=0pt]
    \item The domains of the maps of $I$ are small relative to $I\cell$;
    \item The domains of the maps of $J$ are small relative to $J\cell$;
    \item The class of fibrations is $J\p$; and
    \item The class of trivial fibrations is $I\p$.
  \end{enumerate}
  We refer to $I$ as the set of \textit{generating cofibrations} and to $J$ as the set of \textit{generating trivial cofibrations}. A cofibrantly generated model category is \textit{finitely generated} if we can choose the sets $I$ and $J$ above so that the domains and codomains of $I$ and $J$ are finite relative to $I\cell$.
\end{definition}

\begin{proposition}[Hovey Proposition 2.1.18]\label{2.1.18}
  Suppose $\cC$ is a cofibrantly generated model category, with generating cofibrations $I$ and generating trivial fibrations $J$.\begin{enumerate}[label=(\alph*),noitemsep,topsep=0pt]
    \item The cofibrations form the class ${\p(I\p)}$.
    \item Every cofibration is a retract of a relative $I$-cell complex.
    \item The domains of $I$ are small relative to the cofibrations.
    \item The trivial cofibrations form the class ${\p(J\p)}$.
    \item Every trivial cofibration is a retract of a relative $J$-cell complex.
    \item The domains of $J$ are small relative to the trivial cofibrations.
  \end{enumerate}
  If $\cC$ is fibrantly generated, then the domains and codomains of $I$ and $J$ are finite relative to the cofibrations.
\end{proposition}
\begin{proof}
  \color{red}TODO.
\end{proof}

\begin{theorem}[Hovey Theorem 2.1.19]\label{2.1.19}
  Suppose $\cC$ is a complete \& cocomplete category. Suppose $\cW$ is a subcategory of $\cC$, and $I$ and $J$ are sets of maps of $\cC$. Then there is a cofibrantly generated model structure on $\cC$ with $I$ as the set of generating cofibrations, $J$ as the set of generating trivial fibrations, and $\cW$ as the subcategory of weak equivalences if and only if the following conditions are satisfied.\begin{enumerate}[label=\arabic*.,noitemsep,topsep=0pt]
    \item The subcategory $\cW$ has the 2-of-3 property and is closed under retracts.
    \item The domains of $I$ are small relative to $I\cell$.
    \item The domains of $J$ are small relative to $J\cell$.
    \item $J\cell\sseq\cW\cap {\p(I\p)}$.
    \item $I\p\sseq\cW\cap J\p$.
    \item Either $\cW\cap {\p(I\p)}\sseq {\p(J\p)}$ or $\cW\cap J\p\sseq I\p$.
  \end{enumerate}
\end{theorem}
\begin{proof}
  \color{red}TODO.
\end{proof}

\begin{definition}\label{saturated}
  Let $\cC$ be a category and $I$ a collection of morphisms in $\cC$. Then if $I$ is closed under transfinite composition, pushouts, and retracts then we say $I$ is \textit{saturated}.
\end{definition}

\section{Topological Spaces}

An injective map $f:X\to Y$ in $\Top$ is an \textit{inclusion} if $U$ is open in $X$ if and only if there is a $V$ open in $Y$ such that $f^{-1}(V)=U$. If $f$ is a closed inclusion and every point in $Y\setminus f(X)$ is closed, then we call $f$ a \textit{closed $T_1$ inclusion}. We will let $\cT$ denote the class of closed $T_1$ inclusions in $\Top$.

The symbol $D^n$ will denote the unit disk in $\bR^n$, and the symbol $S^{n-1}$ will denote the unit sphere in $\bR^n$, so that we have the boundary inclusions $S^{n-1}\into D^n$. In particular, for $n=0$ we let $D^0=\{0\}$ and $S^{-1}=\emptyset$.

Recall: If $F:\cJ\to\Top$ is a functor, where $\cJ$ is a small category, the limit of $F$ is obtained by taking the limit in the category of sets, and then topologizing it with the \textit{initial topology}, where if $\eta:\underline{\lim F}\Rightarrow F$ is the limit cone, then the open sets in $\lim F$ are precisely the sets of the form $\eta_j^{-1}(U)$ where $j\in\cJ$ and $U\sseq F_j$ is open. Similarly, the colimit of $F$ is obtained by taking the colimit $\colim F$ in the category of sets, and declaring a set $U\sseq\colim F$ to be open if and only if $\vare_j^{-1}(U)$ is open in $F_j$ for all $j\in\cJ$, where $\vare:F\Rightarrow\underline{\colim F}$ is the colimit cone.

\begin{definition}
  A map $f:X\to Y$ in $\Top$ is called a \textit{weak equivalence} if
  \[\pi_n(f,x):\pi_n(X,x)\to\pi_n(Y,f(x))\]
  is an isomorphism for all $n\geq0$ and for all $x\in X$. We will write $\cW$ to refer to the class of all weak equivalences in $\Top$.

  Define the set of maps $I'$ to consist of all the boundary inclusion $S^{n-1}\into D^n$ for all $n\geq0$, and define the set $J$ to consist of all the inclusions $D^n\into D^n\times I$ mapping $x\mapsto(x,0)$ for $n\geq0$. Then a map $f$ will be called a \textit{cofibration} if it is in $I\cof={\p( I'\p )}$, and a \textit{fibration} if it is in $J\inj=J\p$.

  A map in $I'\cell$ is usually called a \textit{relative cell complex}; a relative CW-complex is a special case of a relative cell complex, where, in particular, the cells can be attached in order of their dimension. Note that in particular maps of $J$ are relative CW complexes, hence are relative $I'$-cell complexes. 
  %Thus $J\cof\sseq I'\cof$.
  A fibration is often known as a \textit{Serre fibration} in the literature.
\end{definition}

\begin{theorem}[Hovey Theorem 2.4.19]\label{2.4.19}
  There is a finitely generated model structure on $\Top$ with $I'$ as the set of generating cofibrations, $J$ as the set of generating trivial cofibrations, and the cofibrations, fibrations, and weak equivalences as above. Every object of $\Top$ is fibrant, and the cofibrant objects are retracts of relative cell complexes.
\end{theorem}
\begin{proof}
  We will apply \autoref{2.1.19} to get that there is a cofibrantly generated model structure on $\Top$ with $I'$ as the set of generating cofibrations, $J$ as the set of generating trivial fibrations, and $\cW$ as the subcategory of weak equivalences. The six requirements outlined in the theorem will be verified like so:
  \begin{enumerate}[label=\arabic*.,noitemsep,topsep=0pt]
    \item $\cW$ is a subcategory of $\cC$ which has the 2-of-3 property and is closed under retracts: \autoref{2.4.4}.
    \item The domains of $I'$ are small relative to $I'\cell$: \autoref{domains_of_I'/J_small_rel_I'-cell/J-cell}.
    \item The domains of $J$ are small relative to $J\cell$: \autoref{domains_of_I'/J_small_rel_I'-cell/J-cell}.
    \item $J\cell\sseq\cW\cap {\p( I'\p )}$: In \autoref{2.4.9}, we will show ${\p({J}\p)}\sseq\cW\cap {\p( I'\p )}$, and by \autoref{2.1.10} $J\cell\sseq {\p({J}\p)}$.
    \item $ I'\p \sseq\cW\cap J\p$: \autoref{2.4.10}
    \item $\cW\cap J\p\sseq  I'\p $: \autoref{2.4.12}
  \end{enumerate}
  It will follow by the definition of a cofibrantly generated model structure (\autoref{2.1.17}) that the fibrations in this model structure are given by $J\p$, which is precisely how we defined it. By \autoref{2.1.18}, the class of cofibrations will be given by ${\p( I'\p )}$, which is likewise exactly how we defined them.

  In \autoref{2.4.2}, we will show that compact spaces are finite relative to the class $\cT$ of closed $T_1$ inclusions. Hence, this model structure will be finitely generated, as the domains and codomains of $I'$ and $J$ are all compact, and by the reasoning given above we will have shown $I'\cell\sseq\cT$.
  
  We will show that every object of $\Top$ is fibrant in \autoref{2.4.14}. 
  \color{red}Finally, to see that cofibrant objects are retracts of relative cell complexes, FINISH
\end{proof}

\begin{lemma}[Hovey 2.4.1]\label{2.4.1}
  Every topological space is small relative to the inclusions.
\end{lemma}
\begin{proof}
  As with the case of sets, we claim that every topological space $X$ is $|X|$-small relative to the inclusions. Indeed, suppose $X$ is a $\lambda$-sequence of inclusions in $\Top$. First, we claim that each map $\iota_{\alpha,\beta}:X_\alpha\to X_\beta$ is an inclusion for $\alpha\leq\beta<\lambda$. We do so by presuming $\alpha<\lambda$ fixed and performing transfinite induction on $\beta$. First of all, in the case $\beta=\alpha$, $\iota_{\alpha,\alpha}$ is the identity and therefore clearly an inclusion. Now, suppose that $\iota_{\alpha,\beta}$ is an inclusion, then we wish to show that $\iota_{\alpha,\beta+1}$ is an inclusion. Since $\iota_{\alpha,\beta+1}=\iota_{\beta,\beta+1}\circ\iota_{\alpha,\beta}$ the composition of inclusions, it too is clearly an inclusion. Finally, suppose that $\gamma$ is a limit ordinal, and that the map $\iota_{\alpha,\beta}$ is an inclusion for all $\alpha\leq\beta<\gamma$. We wish to show that the map $\iota_{\alpha,\gamma}$ is an inclusion. First, we claim this map is an injection. Since $\gamma$ is a limit ordinal and $X$ is colimit-preserving, $X_\gamma$ is the colimit of the diagram $X$ restricted to those $X_\beta$ such that $\beta<\gamma$, so that in particular by \autoref{eq1} and the discussion at the beginning of this section, given $a,b\in X_\alpha$, $\iota_{\alpha,\gamma}(a)=\iota_{\alpha,\gamma}(b)$ iff $\iota_{\alpha,\beta}(a)=\iota_{\alpha,\beta}(b)$ for some $\alpha\leq\beta<\gamma$. But we know the map $\iota_{\alpha,\beta}$ is an inclusion, so that if $\iota_{\alpha,\beta}(a)=\iota_{\alpha,\beta}(b)$, then it must have been true $a=b$ in $X_\alpha$. Hence, $\iota_{\alpha,\gamma}$ is injective. Finally, we wish to show that $U\sseq X_\alpha$ is open if and only if there is some $V\sseq X_\gamma$ open such that $\iota_{\alpha,\gamma}^{-1}(V)=U$. The backwards direction is clear as $\iota_{\alpha,\gamma}$ is continuous. Now suppose, $U\sseq X_\alpha$ is open. Then since $\iota_{\alpha,\beta}$ is an inclusion for all $\alpha\leq\beta<\gamma$, for $\alpha\leq\beta$ there exists $V_\beta\sseq X_\beta$ open such that $\iota_{\alpha,\beta}^{-1}(V_\beta)=U$. Now, define
  \[V:=\bigcup_{\alpha\leq\beta<\gamma}\iota_{\beta,\gamma}(V_\beta).\]
  First of all, we claim that $\iota_{\beta,\gamma}^{-1}(V)=V_\beta$ for all $\beta<\gamma$.{\color{red}TODO: FINISH.}

  Now, 
\end{proof}

\begin{proposition}[Hovey 2.4.2]\label{2.4.2}
  Compact topological spaces are finite relative to the class $\cT$ of closed $T_1$ inclusions.
\end{proposition}
\begin{proof}
  
\end{proof}

\begin{proposition}[Hovey 2.4.5 \& 2.4.6]\label{2.4.5-6}
  The class $\cT$ of closed $T_1$ inclusions is saturated.
\end{proposition}
\begin{proof}
  \color{red}TODO.
\end{proof}

\begin{proposition}\label{domains_of_I'/J_small_rel_I'-cell/J-cell}
  The domains of $I'$ (resp.\ $J$) are small relative to $I'\cell$.
\end{proposition}
\begin{proof}
  By \autoref{2.4.1}, every space is small relative to the inclusions, and in particular every space is small relative to the class $\cT$ of closed $T_1$ inclusions. Hence, it suffices to show that $J\cell,I'\cell\sseq\cT$. We showed above in \autoref{2.4.5-6} that $\cT$ is saturated, and clearly every map in $I'$ and $J$ is a closed $T_1$ inclusion, so the desired result follows.
\end{proof}

\begin{lemma}[Hovey Lemma 2.4.4]\label{2.4.4}
  The weak equivalences in $\Top$ are closed under retracts and satisfy 2-of-3 axiom (so that in particular the weak equivalences form a subcategory, as clearly identities are weak equivalences).
\end{lemma}
\begin{proof}
  First we show that weak equivalences satisfy 2-of-3. Let $f:X\to Y$ and $g:Y\to Z$ be continuous functions of topological spaces. 
  
  First of all, suppose $f$ and $g$ are both weak equivalences. Then by functoriality of $\pi_n$, since $\pi_n(f,x)$ and $\pi_n(g,f(x))$ are isomorphisms for all $x\in X$, $\pi_n(g\circ f,x)=\pi_n(g,f(x))\circ\pi_n(f,x)$ is likewise an isomorphism for all $x\in X$, so that $g\circ f$ is a weak equivalence.

  Now, suppose that $g\circ f$ and $g$ are weak equivalences. Pick a point $x\in X$. We wish to show that $\pi_n(f,x):\pi_n(X,x)\to\pi_n(Y,f(x))$ is an isomorphism for all $n\geq0$. We know that $\pi_n(g\circ f,x)$ is an isomorphism, and $\pi_n(g,f(x))$ is an isomorphism, say with inverse, $\varphi$, so that
  \[\varphi\circ\pi_n(g\circ f,x)=\varphi\circ\pi_n(g,f(x))\circ\pi_n(f,x)=\pi_n(f,x)\]
  is an isomorphism, as it is a composition of isomorphisms.

  Now, suppose that $g\circ f$ and $f$ are weak equivalences. Pick a point $y\in Y$. Since $\pi_0(f)$ is an isomorphism, there exists a point $x\in X$ such that $f(x)$ belongs to the path component containing $y$, so that there exists some $\alpha:I\to Y$ with $\alpha(0)=f(x)$ and $\alpha(1)=f(y)$. Then consider the following diagram
  % https://q.uiver.app/?q=WzAsNCxbMCwwLCJcXHBpX24oWSx5KSJdLFsxLDAsIlxccGlfbihaLGcoeSkpIl0sWzAsMSwiXFxwaV9uKFksZih4KSkiXSxbMSwxLCJcXHBpX24oWixnKGYoeCkpKSJdLFswLDEsIlxccGlfbihnLHkpIl0sWzAsMl0sWzIsMywiXFxwaV9uKGcsZih4KSkiLDJdLFsxLDNdXQ==
  \[\begin{tikzcd}
    {\pi_n(Y,y)} & {\pi_n(Z,g(y))} \\
    {\pi_n(Y,f(x))} & {\pi_n(Z,g(f(x)))}
    \arrow["{\pi_n(g,y)}", from=1-1, to=1-2]
    \arrow[from=1-1, to=2-1]
    \arrow["{\pi_n(g,f(x))}", from=2-1, to=2-2]
    \arrow[from=1-2, to=2-2]
  \end{tikzcd}\]
  where the left arrow is the isomorphism given by conjugation by the path $\alpha$, and the right arrow is the isomorphism given by conjugation by the path $g\circ\alpha$. It is tedious yet straightforward to verify that the diagram commutes.
  Furthermore, we know that $\pi_n(f,x)$ and $\pi_n(g\circ f,x)=\pi_n(g,f(x))\circ\pi_n(f,x)$ are isomorphisms for all $n$, so that if we denote the inverse of $\pi_n(f,x)$ by $\varphi$, then
  \[\pi_n(g\circ f,x)\circ\varphi=\pi_n(g,f(x))\circ\pi_n(f,x)\circ\varphi=\pi_n(g,f(x))\]
  is an isomorphism, as it is given as a composition of isomorphisms. Hence, the top arrow must likewise be an isomorphism, precisely the desired result.

  The fact that weak equivalences in $\Top$ are closed under retracts is entirely straightforward and follows from the fact that the functors $\pi_n$ preserve retract diagrams and that the class of isomorphisms in any category is closed under retracts.
\end{proof}

\begin{proposition}[Hovey 2.4.9]\label{2.4.9}
  ${\p({J}\p)}\sseq\cW\cap {\p( I'\p )}$.
\end{proposition}
\begin{proof}
  First, in order to show ${\p(J\p)}\sseq{\p( I'\p )}$, It suffices to show that $J\sseq I'\cell$, as by \autoref{2.1.10} we would have $J\sseq{\p( I'\p )}$, and
  \[J\sseq {\p( I'\p )}\implies{\p(J\p)}\sseq{\p((\p( I'\p ))\p)}={\p( I'\p )},\]
  where the implication and equality both follow from \autoref{useful_LP_properties} which asserts that
  \[A\sseq B\implies {\p(A\p)}\sseq{\p(B\p)}\quad\text{ and }\quad(\p({A}\p))\p=A\p.\]
  Now, to show $J\sseq I'\cell$, first consider the composition $j_n:D^n\into S^n\into D^{n+1}$, where the first map is the pushout
  \[\begin{tikzcd}
    {S^{n-1}} & {D^n} \\
    {D^n} & {S^n}
    \arrow[hook, from=1-1, to=1-2]
    \arrow[hook, from=1-1, to=2-1]
    \arrow[from=2-1, to=2-2]
    \arrow[from=1-2, to=2-2]
    \arrow["\ulcorner"{anchor=center, pos=0.125, rotate=180}, draw=none, from=2-2, to=1-1]
  \end{tikzcd}\]
  obtained by gluing two copies of $D^n$ along their boundary, and the second map map is simply the inclusion $S^n\into D^{n+1}$, which can be written as the pushout
  \[\begin{tikzcd}
    {S^n} & {S^n} \\
    {D^{n+1}} & {D^{n+1}}
    \arrow[Rightarrow, no head, from=1-1, to=1-2]
    \arrow[hook, from=1-1, to=2-1]
    \arrow[Rightarrow, no head, from=2-1, to=2-2]
    \arrow[hook, from=1-2, to=2-2]
    \arrow["\ulcorner"{anchor=center, pos=0.125, rotate=180}, draw=none, from=2-2, to=1-1]
  \end{tikzcd}\]
  It can be seen that $j_n$ includes $D^n$ as a hemisphere of $S^n=\partial D^{n+1}\sseq D^{n+1}$. Note that $D^{n}\times I$ is homeomorphic to $D^{n+1}$ (``smooth out'' the sharp edges of the cylinder) via some homeomorphism $h_n:D^{n+1}\to D^n\times I$, and in particular, we may define $h_n$ so that $h_n(j_n(D^n))= D^n\times\{0\}\sseq D^n\times I$ by squashing the hemisphere $j_n(D^n)$ to be one of the faces of the cylinder $D^n\times I$, in which case $h_n\circ j_n:D^n\to D^n\times I$ is precisely the inclusion $D^n\into D^n\times I$ sending $x\mapsto (x,0)$, and since $j_n\in I'\cell$, $h_n\circ j_n\in I'\cell$ by \autoref{I-cell_closed_under_composition_with_isomorphisms}.

  Now, we claim that $\p(J\p)\sseq\cW$. First note that by \autoref{domains_of_I'/J_small_rel_I'-cell/J-cell} and \autoref{2.1.15}, every map in $\p(J\p)$ is a retract of an element of $J\cell$. Thus, it suffices to find a saturated class $\cS$ of maps in $\Top$ with $J\sseq\cS\sseq\cW$. Indeed, let $\cS$ be the class of \textit{inclusions of a deformation retract}, i.e., those maps $i:A\to B$ such that there exists a homotopy $H:B\times I\to B$ with $H(i(a),t)=i(a)$ for all $a\in A$, $H(b,0)=b$ for all $b\in B$, and $H(b,1)=i(r(b))$ for some map $r:B\to A$. We must complete three steps:
  \begin{enumerate}[listparindent=\parindent,parsep=0pt]
    \item $J\sseq\cS$.
    
    For $n\geq0$, let $j_n:D^n\into D^n\times I$ denote the inclusion of $D^n$ as the subset $D^n\times\{0\}$. Define a deformation retract $H:D^n\times I\times I\to D^n\times I$ by $(x,s,t)\mapsto(x,s(1-t))$. Then indeed we have $H(j_n(x),t)=H(x,0,t)=(x,0)=j_n(x)$ for all $x\in D^n$, $H(x,t,0)=(x,t(1-0))=(x,t)$ for all $(x,t)\in D^n\times I$, and $H(x,t,1)=(x,t(1-1))=(x,0)=j_n(r(x))$ for all $(x,t)\in D^n\times I$, where $r:D^n\times I\to D^n$ is the projection onto time zero sending $(x,t)\mapsto(x,0)$. Thus, indeed $J\sseq\cS$.

    \item $\cS\sseq\cW$.
    
    It suffices to show that if $i:A\to B$ belongs to $\cS$, then $i$ is a homotopy equivalence. Indeed, given $i:A\to B$, let $H:B\times I\to B$ and $r:B\to A$ be a homotopy and retract satisfying the conditions above. Then in particular, $H$ is a homotopy between $\id_B$ (at time $t=0$) and $i\circ r:$ (at time $t=1$), so it remains to show that $r\circ i$ is homotopic to $\id_A$.

    \item $\cW$ is saturated.
  \end{enumerate}
\end{proof}

\begin{proposition}[Hovey 2.4.10]\label{2.4.10}
  $I'\p\sseq\cW\cap J\p$
\end{proposition}
\begin{proof}
  \color{red}TODO.
\end{proof}

\begin{proposition}[Hovey 2.4.12]\label{2.4.12}
  $\cW\cap J\p\sseq I'\p $
\end{proposition}
\begin{proof}
  \color{red}TODO.
\end{proof}

\begin{corollary}[Hovey 2.4.14]\label{2.4.14}
  Every topological space is fibrant, i.e., given a space $X$, the unique map $X\to\ast$ is an element of $J\p$.
\end{corollary}
\begin{proof}
  \color{red}TODO.
\end{proof}

\textbf{Questions:}\begin{enumerate}
  \item Lemma 2.3 help pls (limit ordinal case in transfinite induction)
\end{enumerate}

\end{document}