\documentclass[../main.tex]{subfiles}
\begin{document}
Recall that the simplex category $\Delta$ is the category with objects
\[[n]:=\{0,1,\ldots,n\}\]
for $n\geq0$ and $\Delta([n],[k])$ is the set of maps $f:[n]\to[k]$ such that $x\leq y\implies f(x)\leq f(y)$. We denote by $\Delta_+$ and $\Delta_-$ the subcategories of injective order-preserving maps and surjective order-preserving maps, respectively. Note that every morphism in $\Delta$ may be factored uniquely into a morphism in $\Delta_-$ followed by a morphism in $\Delta_+$. In fact, $\Delta$ is generated by the morphisms $d^i:[n-1]\into[n]\in\Delta_+$ for $n\geq1$ and $0\leq i\leq n$, where the image of $d^i$ does not include $i$, and the morphisms $s^i:[n]\to[n-1]\in\Delta_-$ for $n\geq1$ and $0\leq i\leq n-1$, where $s^i$ identifies $i$ and $i+1$. All the relations among these maps are implied by the \textit{cosimplicial identities}:
\[
  \begin{array}{rcll}
    d^jd^i & = & d^id^{j-1} & (i<j) \\
    s^jd^i & = & d^is^{j-1} & (i<j) \\
    & = & \id & (i=j,j+1) \\
    & = & d^{i-1}s^j & (i>j+1) \\
    s^js^i & = & s^{i-1}s^j & (i>j)
  \end{array}
\]

If $\cC$ is any category, the category of \textit{cosimplicial objects in $\cC$} is the functor category $\cC^\Delta$, and the category of \textit{simplicial objects in $\cC$} is the functor category $\cC^{\Delta^\op}$. Note that these functor categories have whatever colimits and limits exist in $\cC$, taken objectwise. The most important example is when $\cC$ is the category of sets, in which case we denote $\Set^{\Delta^\op}$ by $\SSet$, and refer to $\SSet$ as the \textit{category of simplicial sets}.

If $K$ is a simplicial set, we denote $K[n]$ by $K_n$ and refer to $K_n$ as the set of $n$-simplices of $K$. If $x\in K_n$, the integer $n$ is referred to as the \textit{dimension of $x$}. Dual to the $d^i$ we have the \textit{face maps} $d_i:K_n\to K_{n-1}$ for $n\geq1$ and $0\leq i\leq n$. Dual to the $s^i$ we have the \textit{degeneracy maps} $s_i:K_{n-1}\to K_n$ for $n\geq1$ and $0\leq i\leq n-1$. These maps are subject to the \textit{simplicial identities}
\[
  \begin{array}{rcll}
    d_id_j & = & d_{j-1}d_i & (i<j) \\
    d_is_j & = & s_{j-1}d_i & (i<j) \\
    & = & \id & (i=j,j+1) \\
    & = & s_jd_{i-1} & (i>j+1) \\
    s_is_j & = & s_{j}s_{i-1} & (i>j).
  \end{array}
\]
A simplicial set $K$ is equivalent to a collection of sets $K_n$ and maps $d_i$ and $s_i$ as above satisfying the above simplicial identities. A map of simplicial sets $f:K\to L$ is equivalent to a collection of maps $f_n:K_n\to L_n$ commuting with the face and degeneracy maps.
\end{document}