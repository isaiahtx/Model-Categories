\documentclass[../main.tex]{subfiles}
\begin{document}
Recall that the simplex category $\bDelta$ is the category with objects
\[[n]:=\{0,1,\ldots,n\}\]
for $n\geq0$ and $\bDelta([n],[k])$ is the set of maps $f:[n]\to[k]$ such that $x\leq y\implies f(x)\leq f(y)$. We denote by $\bDelta_+$ and $\bDelta_-$ the subcategories of injective order-preserving maps and surjective order-preserving maps, respectively. Note that every morphism in $\bDelta$ may be factored uniquely into a morphism in $\bDelta_-$ followed by a morphism in $\bDelta_+$. In fact, $\bDelta$ is generated by the morphisms $d^i:[n-1]\into[n]\in\bDelta_+$ for $n\geq1$ and $0\leq i\leq n$, where the image of $d^i$ does not include $i$, and the morphisms $s^i:[n]\to[n-1]\in\bDelta_-$ for $n\geq1$ and $0\leq i\leq n-1$, where $s^i$ identifies $i$ and $i+1$. All the relations among these maps are implied by the \textit{cosimplicial identities}:
\[
  \begin{array}{rcll}
    d^jd^i & = & d^id^{j-1} & (i<j) \\
    s^jd^i & = & d^is^{j-1} & (i<j) \\
    & = & \id & (i=j,j+1) \\
    & = & d^{i-1}s^j & (i>j+1) \\
    s^js^i & = & s^{i-1}s^j & (i>j)
  \end{array}
\]

If $\cC$ is any category, the category of \textit{cosimplicial objects in $\cC$} is the functor category $\cC^\bDelta$, and the category of \textit{simplicial objects in $\cC$} is the functor category $\cC_\bDelta:=\cC^{\bDelta^\op}$. Note that these functor categories have whatever colimits and limits exist in $\cC$, taken objectwise. The most important example is when $\cC$ is the category of sets, in which case we refer to $\Set_\bDelta$ as the \textit{category of simplicial sets}.

If $K$ is a simplicial set, we denote $K([n])$ by $K_n$ and refer to $K_n$ as the set of $n$-simplices of $K$. If $x\in K_n$, the integer $n$ is referred to as the \textit{dimension of $x$}. Dual to the $d^i$ we have the \textit{face maps} $d_i:K_n\to K_{n-1}$ for $n\geq1$ and $0\leq i\leq n$. Dual to the $s^i$ we have the \textit{degeneracy maps} $s_i:K_{n-1}\to K_n$ for $n\geq1$ and $0\leq i\leq n-1$. These maps are subject to the \textit{simplicial identities}
\[
  \begin{array}{rcll}
    d_id_j & = & d_{j-1}d_i & (i<j) \\
    d_is_j & = & s_{j-1}d_i & (i<j) \\
    & = & \id & (i=j,j+1) \\
    & = & s_jd_{i-1} & (i>j+1) \\
    s_is_j & = & s_{j}s_{i-1} & (i>j).
  \end{array}
\]
A simplicial set $K$ is equivalent to a collection of sets $K_n$ and maps $d_i$ and $s_i$ as above satisfying the above simplicial identities. A map of simplicial sets $f:K\to L$ is equivalent to a collection of maps $f_n:K_n\to L_n$ commuting with the face and degeneracy maps.

\begin{definition}
  Let $X$ be a simplicial set. Given $n\geq0$, define the $n$-skeleton $\sk_nX$ of $X$ to be the simplicial set constructed as follows: for $k\leq n$, define $(\sk_nX)_k$ to be $X_k$. Supposing $(\sk_nX)_k$ has been constructed for some $k\geq n$, define $(\sk_nX)_{k+1}$ to be the subset of $X_{k+1}$ containing precisely those $(k+1)$-simplices that can be obtained as degenreacies of the $k$-simplices.

  Define the structure maps for $\sk_nX$ to be the restrictions of the structure maps for $X$.
\end{definition}

\begin{proposition}
  The above definition defines a simplicial set. Furthermore, $X$ is the colimit of the $\sk_nX$'s with the obvious inclusion maps.
\end{proposition}
\begin{proof}
  First, we claim these maps are well-defined, i.e., that given $x\in(\sk_nX)_k$, $0\leq i\leq k$, that $s_ix\in(\sk_nX)_{k+1}$ and $d_ix\in(\sk_nX)_{k-1}$. Clearly by how $\sk_nX$ is defined, given $k\geq0$, $0\leq i\leq k$, and $x\in(\sk_nX)_k$, $s_ix$ clearly belongs to $(\sk_nX)_{k+1}$. On the other hand, suppose we are given $x\in(\sk_nX)_k$ for some $k\geq0$ and some $0\leq i\leq k$. If $k\leq n$, then clearly $d_ix\in(\sk_nX)_{k-1}=X_{k-1}$. Conversely, if $k>n$, then $x=s_jy$ for some $y\in(\sk_nX)_{k-1}$ and $0\leq j\leq k-1$. If $i<j$, then
  \[d_ix=d_is_jy=s_{j-1}d_iy\]
  is a degeneracy as desired. If $i=j$ or $i=j+1$, then 
  \[d_ix=d_is_jy=y\in(\sk_nX)_{k-1}\]
  Finally, if $i>j+1$, then
  \[d_ix=d_is_jy=s_jd_{i-1}y\]
  is a degeneracy of $d_{i-1}y\in(\sk_nX)_{k-2}$.\todo{this is unfinished}
\end{proof}
\end{document}